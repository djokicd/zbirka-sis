\usepackage[math]{cmsrb}
\usepackage{graphicx}
\usepackage[dvipsnames]{xcolor}
\usepackage[T1, T2A]{fontenc}
\usepackage[utf8]{inputenc}
%\usepackage[b5paper, textwidth = 415 pt, textheight = 600 pt, marginparwidth = 5 mm, hoffset = -4 mm]{geometry}
%\usepackage[b5paper, textwidth=415pt, marginparwidth=5mm, inner = 15mm]{geometry}
\usepackage[b5paper, textwidth=415pt, marginparwidth=4mm, marginparsep = 1pt, inner = 20mm]{geometry}
\usepackage[serbianc]{babel}
\usepackage{amsmath, amsfonts}
\usepackage{mathtools}
\usepackage[Symbolsmallscale]{upgreek}
\usepackage[bottom]{footmisc}
\usepackage{hyperref}

%% All math display style
\everymath{\displaystyle}

\usepackage{enumerate}

%% Uredjenje prikaza, poglavlja, podpoglavlja, headeri

\usepackage{fancyhdr}
\pagestyle{fancy}
\fancyhf{} % clear all fields
\fancyhead[LO]{\rightmark}
\fancyhead[RO]{\thepage}
\fancyhead[RE]{\rightmark}
\fancyhead[LE]{\thepage}

\renewcommand{\subsectionmark}[1]{%
  \markright{{\thesubsection.\ #1}}
  }%


\counterwithout{section}{chapter}


\usepackage{multicol}
\usepackage{icomma}
\usepackage{enumitem}
\usepackage{graphicx}
\usepackage{wrapfig}
\usepackage{caption}
\usepackage{subcaption}
\usepackage{tabu}
\usepackage{cancel}

\renewcommand{\figurename}{Слика}
\newcommand{\mr}[1]{{\rm #1}}
\usepackage{calc}

\usepackage{xcolor}


\usepackage{fancybox}

\newcommand{\unit}[1]{\,\left[\mathrm{#1}\right]}
\newcommand{\uu}{{\rm u}}
\newcommand{\jj}{{\rm j}}
\newcommand{\ee}{{\rm e}}
\newcommand{\de}{{\rm d}}
\newcommand{\III}{\text{Ш}}
\newcommand{\const}{{\rm const}}
\newcommand{\rect}{\operatorname{rect}}
\newcommand{\tri}{\operatorname{tri}}
\newcommand{\sinc}{\operatorname{sinc}}
\newcommand{\DS}{\displaystyle}

\newcommand{\DD}{{\rm D}}
\newcommand{\II}{{\rm I}}
\newcommand{\EE}{{\rm E}}

\newcommand{\FS}[1]{\mathcal{FS}\left\{#1\right\}}
\newcommand{\FT}[1]{\mathcal{FT}\left\{#1\right\}}
\newcommand{\IFT}[1]{\mathcal{FT}^{-1}\left\{#1\right\}}
\newcommand{\LT}[1]{\mathcal{L}\left\{#1\right\}}
\newcommand{\ILT}[1]{\mathcal{L}^{-1}\left\{#1\right\}}
\newcommand{\ZT}[1]{\mathcal{Z}\left\{#1\right\}}
\newcommand{\IZT}[1]{\mathcal{Z}^{-1}\left\{#1\right\}}
\newcommand{\sgn}[1]{\operatorname{sgn}{#1}}

\renewcommand{\Re}[1]{\mathbb{R}\mathrm{e}\left\{#1\right\}}
\renewcommand{\Im}[1]{\mathbb{I}\mathrm{m}\left\{#1\right\}} 


\newcounter{tablicecount}
\setcounter{tablicecount}{1}
\counterwithin{tablicecount}{section}
%\newcommand{\redTablice}{\stepcounter{tablicecount}\thesection.\arabic{tablicecount}}
\newcommand{\redTablice}{\refstepcounter{tablicecount}\thetablicecount}

\newcommand{\Svojstvo}{(\refstepcounter{equation}\theequation)} 

\newcommand{\reft}[1]{T.\ref{#1}}
\newcommand{\refs}[1]{S.\ref{#1}}


%\renewcommand{\tablicecount}{\thesection.\arabic{tablicecount}} % adds the section number before your proof counters


\usepackage{draftwatermark}

\SetWatermarkText{{\bf Сигнали и системи у инжењерству \\ \bf Збирка задатака,\bf Радна верзија: \\ \today}}
\SetWatermarkColor[gray]{0.95}
\SetWatermarkFontSize{1.25cm}
\SetWatermarkAngle{60}
\SetWatermarkHorCenter{9cm}

\newlength\sirinaSlike

\raggedbottom

\usepackage{xparse}

% Brojac zadataka


%% RAZMAK IZMEDJU ZADATAKA

\newcommand{\ProblemSep}{5mm}

% RESENJE REZULTAT

\newcommand{\RESENJE}{\par\textsc{\goodbreak\underline{Решење}} \par \indent}
\newcommand{\REZULTAT}{\par\textsc{\goodbreak\underline{Резултат}} \par \indent}


%% BROJACI 

\newcounter{ID} % Brojac zadataka
\newcounter{fid}

\newcommand{\fid}{\stepcounter{fid}\thefid}
\newcommand{\ID}{\theID}

\newcommand{\PID}[1][ ]{\noindent\textbf{\color{blue}\ID.}
\immediate\write\pregledFile{\unexpanded{\\} \ID\unexpanded{ & } \unexpanded{#1} \unexpanded{ & }  } }

\counterwithin{equation}{ID}
\counterwithin{figure}{ID}

%% SLIKA PORED TEKSTA

\NewDocumentEnvironment{slikaDesno}{oomo}
{
    \IfNoValueTF{#1}
        { \def\slika{\includegraphics[scale=1]{#3}} }
        { \def\slika{\includegraphics[scale=#1]{#3}} }
    \setlength\sirinaSlike{\widthof{\slika}}
    \hspace*{-19pt}
    \begin{minipage}[t]{\textwidth - \sirinaSlike - 20pt}
}{
    \end{minipage}
    %
    \hfill
    %
    \begin{minipage}[t]{\sirinaSlike}
        \vspace*{-10pt}
        \begin{center}
            \slika \\[2pt]
            % Слика \ID.\fid.
            \IfNoValueTF{#2}
            {\captionof{figure}{}}
            {\captionof{figure}{#2}}
            \IfNoValueTF{#4}
            {}
            {
                \includegraphics[scale=1]{#4}
                \captionof{figure}{}
            }
        \end{center}
    \end{minipage}
}

%% 

\widowpenalty=300000
\clubpenalty=150

%% CRTICE

\usepackage{environ}
\usepackage{tikz}

\NewEnviron{crtice}{
\par
\noindent
\begin{tikzpicture}
\node[rectangle,minimum width=0.99\textwidth] (m) {\begin{minipage}{0.99\textwidth}\BODY\end{minipage}};
\draw[dashed] (m.south west) -- (m.north west);
\draw[dashed] (m.south east) -- (m.north east);
\end{tikzpicture}
}


%% MARGIN NOTES 

\usepackage{layout}
\usepackage{fourier-orns}
%\usepackage{marginnote}
\newcommand{\mnImportant}{{\marginpar{$\color{red}\clubsuit$}}}
\newcommand{\mnDifficult}{{\marginpar{\textcolor{red}{\warning}}}}
\newcommand{\mnAdvanced}{\marginpar{\textcolor{red}{\noway}}}

%% TOC 

\usepackage{tocloft}
\renewcommand{\cftchapfont}{\normalfont}  % Remove bold for chapter entries
\renewcommand{\cftchappagefont}{\normalfont}  % Remove bold for page numbers
\renewcommand{\cftchapdotsep}{\cftdotsep}  % Add dots to chapter entries
\renewcommand{\cftchapleader}{\cftdotfill{\cftdotsep}}  % Customize dot leaders

%% UNDERLINE

\usepackage{soulutf8}
\usepackage{color}
\makeatletter
\newcommand*{\whiten}[1]{\llap{\textcolor{white}{{\the\SOUL@token}}\hspace{#1pt}}}
\DeclareRobustCommand*\myul{%
    \def\SOUL@everyspace{\underline{\space}\kern\z@}%
    \def\SOUL@everytoken{%
     \setbox0=\hbox{\the\SOUL@token}%
     \ifdim\dp0>\z@
        \raisebox{\dp0}{\underline{\phantom{\the\SOUL@token}}}%
        \whiten{1}\whiten{0}%
        \whiten{-1}\whiten{-2}%
        \llap{\the\SOUL@token}%
     \else
        \underline{\the\SOUL@token}%
     \fi}%
\SOUL@}
\makeatother


%% METODICKI PREGLED
\usepackage{supertabular,booktabs}

\newcolumntype{b}{X}
\newcolumntype{s}{>{\hsize=.25\hsize}X}

\newcommand{\metodickiPregled}{
    %\input{pregled.tex}
}

%\usepackage{refcount}
\newcommand{\refz}[1]{%
    \ref{z:#1}%
    \immediate\write\pregledFile{\getrefnumber{z:#1}}%
}

% Referenca bez upisa u metodicki fajl
\newcommand{\refzz}[1]{%
    \ref{z:#1}%
}

\newcommand{\metod}[1]{\immediate\write\pregledFile{\unexpanded{#1}}}