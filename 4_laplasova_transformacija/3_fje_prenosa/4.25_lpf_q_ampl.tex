\PID \mnImportant
За систем $H(s)$ дефиисан у задатку \ref{z:lpf_q}, за избор параметара 
$Q = 5$ и $\upomega_{\rm N} = 1\unit{\dfrac{krad}{s}}$,
\begin{enumerate}[label=(\alph*)]
    \item одредити граничне учестаности филтра $\upomega_{\rm g}$ такве да је
    $|H(\jj\upomega_{\rm g})| = \dfrac{\max|H(\jj\upomega)|}{\sqrt 2}$.
    \item Скицирати амплититудску фреквенцијску карактеристику датог филтра за параметре из претходне тачке. 
    \item Скицирати фазну фреквенцијску карактеристику. 
\end{enumerate}

\RESENJE 
(а) Амплитудска фреквенцијска карактеристика одређује се израчунавањем модула Лапласове трансформације 
на имагинарној оси, чиме се има 
\begin{equation}
    |H(\jj\upomega)| =
    \left|
    \dfrac{\upomega_{\rm N}^2 }{(\upomega_{\rm N}^2 - \upomega^2) + \jj\upomega\dfrac{\upomega_{\rm N}}{Q}} 
    \right| 
    =
    \dfrac{\upomega_{\rm N}^2}{\sqrt{  
        \left(\upomega_{\rm N}^2 - \upomega^2\right)^2 
        +
        \left(
            \dfrac{\upomega_{\rm N}}{Q} \upomega
        \right)^2
    }}
    \label{eq:\ID.freq}
\end{equation}
За скицирање графика испитаћемо асимптотско понашање дате фреквенцијске карактериситке. 
У случају када је $Q$-фактор довољно велики, тада је 
$\upomega\dfrac{\upomega_{\rm N}}{Q} \ll \upomega_{\rm N}^2 - \upomega^2$, изузев у непосредној околини 
$\upomega_{\rm N} \approx \upomega$, па је у том случају
\begin{equation}
    | H \bigl( \jj(\upomega \not \approx \upomega_{\rm N}) \bigr) |\approx
    \left|
    \dfrac{\upomega_{\rm N}^2}{\upomega_{\rm N}^2 - \upomega^2}
    \right|
    \approx \begin{cases}
        1 & \upomega \ll \upomega_{\rm N} \\[2mm]
        \dfrac{\upomega_{\rm N}^2}{\upomega_0} & \upomega \gg \upomega_{\rm N}.
    \end{cases}
\end{equation} 
У случају непосредне околине $\upomega_{\rm N}\approx \upomega_{0}$, онда је доминантни члан у изразу
\ref{eq:\ID.freq} имагинарни део, па је онда 
\begin{equation}
    H \bigl( \jj(\upomega \approx \upomega_{\rm N}) \bigr) \approx 
    \dfrac{\upomega_{\rm N}^2}{\underbrace{\upomega}_{\mathclap{\approx \upomega_{\rm N}}}\dfrac{\upomega_{\rm N}}{Q} }
    \approx Q.
\end{equation}
Дефинисана граничну учестаност филтра одређује се на основу израза \ref{eq:\ID.freq}, нормирањем учестаности 
по $\upomega_{\rm N}$ увођењем нормиране граничне учестаности $w_{\rm g} = \upomega_{\rm g}/\upomega_{\rm N}$ добија се 
\begin{equation}
    \dfrac{\upomega_{\rm N}^2}{\sqrt{  
        \left(\upomega_{\rm N}^2 - \upomega_{\rm g}^2\right)^2 
        +
        \left(
            \dfrac{\upomega_{\rm N}}{Q} \upomega_{\rm g}
        \right)^2
    }} = \dfrac{1}{\sqrt2} 
    \Rightarrow
    (1 - w_{\rm g}^2)^2 + \left( \dfrac{w_{\rm g}}{Q} \right)^2 = 2
\end{equation}
Добијена биквадратна једначина се решава чиме се добија нормирана гранична учесатност
$
    w_{\rm g} = \dfrac{Q^2 \pm Q \sqrt{2Q^2 + 1}}{Q^2 + 1}
$. 
У случају када је $Q$-фактор довољно велики, ово се асимптотски може изразити као 
\begin{equation}
    w_{\rm g} = \dfrac{Q^2 \pm Q \sqrt{2Q^2 \cancel{+ 1}    }}{{Q^2 \underbrace{\cancel{+ 1}}_{\ll Q^2}  } }
    \approx 1 \pm \dfrac{\sqrt 2}{Q}
\end{equation}
Граничне учестаности филтра су онда $\upomega_{\rm g} \approx \upomega_{\rm N}
\left(  1 \pm \dfrac{\sqrt 2}{Q} \right)$, а ширина пропусног опсега је онда 
$\Delta \upomega = \dfrac{2\upomega_{\rm N}\sqrt 2}{Q}$. Из добијеног резултата може се приметити да са повећањем 
$Q$-фактора фреквенцијска селективност филтра расте, и у погледу резонантног појачања, и у погледу сужавања пропусног 
опсега.

Ради потпуности, на слици \ref{fig:\ID.freq_resp} је приказан скуп различитих нормираних фреквенцијских карактеристика
за различите $Q$-факторе

\begin{figure}
    \centering
    %
    \begin{subfigure}[t]{0.45\textwidth}
        \includegraphics{fig/Q_ampl_approx.pdf}    
        \caption{Асимптотско понашање и скица амплитудске карактеристике филтра.}
        \label{fig:\ID.crtez}
    \end{subfigure}
    %
    \begin{subfigure}[t]{0.45\textwidth}
        \includegraphics{fig/Q_razlicite_ampl_resp.pdf}    
        \caption{Нормирана амплитудска карактерситика филтра за различите $Q$-факторе}
        \label{fig:\ID.freq_resp}
    \end{subfigure}
    \caption{Уз решење задатка.}
\end{figure}