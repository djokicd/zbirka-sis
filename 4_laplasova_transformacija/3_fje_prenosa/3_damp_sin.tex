%{\color{red}$\blacktriangle$} 
\PID \label{z:damp_sin}
Нека је дата Лапласова трансформација сигнала $x(t)$ у облику $\LT{x(t)} = X(s) = \dfrac{1}{s^2 + s + 1}$. Одредити 
сигнал $x(t)$. 

\RESENJE  Полови дате функције преноса су корени карактеристичне једначине 
$p^2 + p + 1 = 0$, који се могу израчунати помоћу обрасца за решења квадратне једначине одакле се налази
$p_{1} = p = \upsigma + \jj\upomega$ и $p_2 = p^\ast$, где су $\upsigma = -\dfrac12$ и $ \upomega = \dfrac{\sqrt{3}}{2}$. 
Растављањем на парцијалне разломке добија се израз облика,
\begin{equation}
    X(s) = \dfrac{A}{s - p} + \dfrac{A^\ast}{s - p^\ast},
\end{equation}
а коефицијент $A$ се може потражити на начин описан у додатку \ref{a:pfd}. На основу тог постука 
је\footnote{Искоришћен је идентитет који важи за комплексне бројеве 
$z - z^\ast = \jj 2 \, \Im{z}$} 
\begin{equation}
    A = \dfrac{1}{\xcancel{(s - p)}(s - p^\ast)} \Bigg|_{s = p} 
      = \dfrac{1}{p - p^\ast} = \dfrac{1}{2 \jj \Im{p} } = \dfrac{1}{\jj\sqrt{3}} = -\jj \dfrac{\sqrt{3}}{3}. \label{\ID.A}
\end{equation}

У наставку, искористимо табличну трансформацију $\LT{e^{at}\uu(t)} = \dfrac{1}{s - a}$, одакле налазимо да је 
\begin{eqnarray}
x(t) &=& \ILT{ \dfrac{A}{s - p} + \dfrac{A^\ast}{s - p^\ast} } = 
       A \ILT{ \dfrac{1}{s - p}} + A^\ast \ILT{ \dfrac{1}{s - p^{\ast}}}  \\[2mm]
     &=&
     A \ee^{pt} + A^* \ee^{p^\ast t}.
\end{eqnarray}
Ради једноставности, изразимо коефицијент $A$ у поларном облику као $A = |A| \ee^{\jj\arg{A}}$, одакле се даље има
\begin{eqnarray}
    x(t) &=&
    |A| \ee^{\jj \arg{A}} \ee^{pt} + |A| \ee^{-\jj \arg{A}} \ee^{p^\ast t} = 
    |A| \ee^{\jj \arg{A}} \ee^{(\upsigma + \jj\upomega)t} + |A| \ee^{-\jj \arg{A}} \ee^{(\upsigma - \jj\upomega) t} = \\
    &=&
    |A| \ee^{\upsigma t} 
    \underbrace{
    \left( 
        \ee^{(\jj(\upomega t + \arg A) } + \ee^{ -(\jj(\upomega t + \arg A)) } 
    \right)}_{ 2\cos(\upomega t + \arg A) },
\end{eqnarray}
одакле коначно закључујемо да је 
\begin{equation}
    x(t) = 2|A| \ee^{\upsigma t} \cos(\upomega t + \arg A)
\end{equation}
У конкретном случају, из \eqref{\ID.A} се има $A = \dfrac{\sqrt 3}{3} \ee^{-\jj\uppi/2}$, па је онда коначно
\begin{equation}
    x(t) = \dfrac{2\sqrt 3}{3} \ee^{-\frac{1}{2}t} \sin \left(\dfrac{\sqrt 3}{2} t\right) \, \uu(t)
\end{equation}