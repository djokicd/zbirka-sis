\begin{slikaDesno}{fig/klatno_1.pdf}[fig/klatno_2.pdf]
    \PID 
    У механичком систему са слике \ID.1 приказано jе инверзно клатно причвршћено
    за ослонац коjи може да се креће дуж $x$ осе. Клатно jе сачињено из кугле масе
    $m$, чиjи jе центар на растоjању $L$ од ослонца, а слободно jе да се креће у равни
    цртежа. Познато jе и $g = |{\bf g}|$.
    \begin{enumerate}[label=(\alph*)]
        \item Одредити функциjу преноса система $H(s)$ чиjи jе улаз тренутни положаj
              ослонца клатна $x = x(t)$ а излаз тренутни угаони отклон клатна 
              $\uptheta = \uptheta(t)$. Сматрати да jе отклон клатна мали тако да jе $\sin \uptheta \approx \uptheta$.
        \item Испитати асимптотску стабилност посматраног система $H(s)$.
        \item У сложеном систему са слике \ID.2 употребљен jе систем $H(s)$ а преносна функциjа другог система jе 
        $C(s) = K$, где jе $K$ константа. Одредити функциjу
        преноса $W(s) = \dfrac{Y(s)}{X(s)}$.
        \item Испитати асимптотску стабилност система $W(s)$ у функциjи параметра $K$.
    \end{enumerate}
\end{slikaDesno}

\REZULTAT

(а) Тражена функција преноса је $H(s) = \dfrac{s^2}{ g - s^2 L}$. (б) Систем је асимптотски нестабилан. 
(в) Функција преноса је $W(s) = \dfrac{K s^2}{g + (K - L) s^2}$.
(г) За $K \leq L$ систем је асимтотски нестабилан, а за $K > L$ је маргинално стабилан. 