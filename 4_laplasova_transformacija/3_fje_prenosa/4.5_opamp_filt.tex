\mnDifficult
\begin{slikaDesno}{fig/opamp_ckt.pdf}
    \PID
    У колу са слике познато jе 
    $R_2 = 10 R_1 = R_3/10$. Функција преноса
    кола, чији улаз jе напон $v_{\rm U}$ а излаз напон 
    $v_{\rm I}$ jе облика
    $
    {H(s) = K\dfrac{s^2}{s^2 + \dfrac{\upomega_0}{Q} s + \upomega_0^2 }}
    $.
    (а) Израчунати параметре $K$ и $Q$,
    и вредности елемената кола $R_1$ и $C_1$ ако су познати $R_2 = 10\unit{k\Omega}$ и 
    $\upomega_0 = 10^5\unit{\dfrac{rad}{s}}$.
    (б) Одредити принудни и устаљени одзив филтра на побуду
    $v_{\rm G}(t) = V_0 \bigl(5 + \ee^{2t} \updelta(t) \bigr)\,\uu(t - \uptau)$, где су $V_0 = 1\unit{V}$ 
    и $\uptau = 2\unit{s}$. 
\end{slikaDesno}

\RESENJE

(а) 
Једносмерно појачање система (DC) одређује се као
$H(s = 0) = 0$. У том случају, капацитивности се понашају као отворене везе. 
Приметимо да је тада отпорност $R_1$ везана редно улазном напонском генератору. 
Односно, да би напонско појачање система било равно нули, мора бити 
$R_1 \to \infty$. 
Са друге стране, 
када jе $\upomega \to \infty$ тада су 
$V^+ = V^- = V_{\rm U}$ за оба операциона појачавача па jе
$VI = 2V_U \Rightarrow K = 2$, због разделника напона $R_2$--$R_2$.

На „+“ прикључку доњег операционог појачавача jе, услед напонског разделника, 
напон $V^+ = \dfrac{V_{\rm I}}2$ , што jе и на „$-$“ прикључку, због НПС.
Тиме jе дефинисана струја кондензатора
као $I = \dfrac{V_{\rm I} - \dfrac{V_{\rm I}}{2}}{R_2} = \dfrac{V_{\rm I}}{2R_2}$.
Напон на излазу доњег операционог појачавача је 
$V_{\rm OP} = \dfrac{V_{\rm I}}{2} \left( 1 - \dfrac{1}{sC_1R_2} \right)$.
Струја
кроз улазни кондензатор jе једнака 
$I_{\rm U}
=
\left(
    V_{\rm U} 
    -
    \dfrac{V_{\rm I}}{2}
\right) sC_1$, одакле се има 
$V_{\rm U} = \dfrac{V_{\rm I}}{2}
\cdot
\dfrac{1 + sC_1R_2 + (sC_1R_2)^2 }{ (sC_1R_2)^2 }$
па се сређивањем добија
\begin{equation}
    H(s) = 2 \dfrac{s^2}{ 
        s^2 + \dfrac{1}{R_2 C_1} s + \left( \dfrac{1}{R_2 C_1} \right)^2
    }.
\end{equation}
Поређењем добијеног облика са обликом из поставке задатка директном идентификацијом 
се установљава да су $Q = 1$ и $C_1 = 1\unit{nF}$

(б)
Пошто jе Хевисаjдова функциjа померена, Делта импулс у нули се анулира, а одзив jе исти као за побуду
$v_{\rm G} = 5 V_0 \, \uu(t - \uptau)$, па се одзив система може потражити као 
\begin{eqnarray}
    V_{\rm I}(s) = 
    V_{\rm G}(s) \cdot H(s)
    =
    \dfrac{5 V_0 \ee^{-s\uptau}}{\cancel{s}} \cdot 
    K\dfrac{s^{\cancel2}}{s^2 + \upomega_0 s + \upomega_0^2 }
\end{eqnarray}
Инверзна Лапласова трансформација може се одредити допуњавањем имениоца до потпуног квадрата
\begin{eqnarray*}
    && V_{\rm I}(s)=5 V_0 \cdot 
    K\dfrac{
        s + \dfrac{\upomega_0}{2} - \dfrac{\upomega_0}{2}
    }{
    \underbrace{s^2 + \upomega_0 s + \left(\dfrac{\upomega_0}{2}\right)^2}_{
        \left(
            s + \upomega_0/2
        \right)^2
    } 
    \underbrace{
    - \left(\dfrac{\upomega_0}{2}\right)^2  + \upomega_0^2 }_{ {3\upomega_0^2}/{4} } 
    }
    \ee^{-s\uptau} = \\
    &=& \hspace*{-2mm} 5KV_0 
        \left(
            \dfrac{
            s + \dfrac{\upomega_0}{2} 
            }{
               \left(s + \dfrac{\upomega_0}2 \right)^2 + \left( \dfrac{\upomega_0 \sqrt 3}{2} \right)^2
            }
            - \dfrac{\cancel \upomega_0}{\cancel 2}
            \cdot
            \dfrac{\cancel 2}{\cancel \upomega_0 \sqrt3}
            \dfrac{
                \dfrac{\upomega_0 \sqrt 3}{2}
            }{
                \left(s + \dfrac{\upomega_0}2 \right)^2 + \left( \dfrac{\upomega_0 \sqrt 3}{2} \right)^2
            }
    \right)
    \ee^{-s\uptau},
\end{eqnarray*}
па се на сличан начин као и у задатку \refz{teleskop} помоћу табличних трансформација 
\reft{T:LT:exp_cos} и \reft{T:LT:exp_sin}, те применом својства кашњења, налази коначни резултат 
\begin{equation}
    v_{\rm I}(t) = 
    10\unit{V}
    \exp\left(- \dfrac{\upomega_0}{2} (t-\uptau)\right)
    \left( 
        \cos\left( \dfrac{\upomega_0\sqrt3}{2}(t-\uptau) \right)
        -
        \dfrac{\sqrt 3}{3}
        \sin\left( \dfrac{\upomega_0\sqrt3}{2}(t-\uptau) \right)
    \right) \uu(t-\uptau).
\end{equation}