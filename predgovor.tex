\section*{Предговор}

Збирка задатака „Сигнали и системи у инжењерству“ намењена je предмету „Сигнали и системи“ на 
Електротехничком факултету Универзитета у Београду на одсеку за Електронику и дигиталне системе. На основу материјала 
са рачунских вежби, као и испитних задатака, Аутор је почео израду збирке задатака која треба да у неком тренутку 
прође и званичну процедуру и рецензију и постане званично наставно средство на предмету. 
Аутор се труди да збирка буде методичка, односно, да се задаци надовезују један на други, сложенији након
једноставнијих. Такође, велики број задатака има тежњу да илуструје примену градива предмета на различитим 
поједностављеним инжењерским проблемима, претежно у електротехници.
Аутор нарочито моли студенте који примете да недостаје неки „степеник“ у овом низу задатака, макар и на свом личном 
примеру, да му то и пренесу. 

Због идеолошких убеђења, Аутор објављује збирку задатака онлајн бесплатно, отвореног кода. 
Студенти могу слободно да користе овај садржај као најажурнији облик материјала за вежбе на 
Предмету, за потребе припремања предиспитних обавеза и испита. 

Изворни код збирке је постављен на \textit{GitHub} где сви заинтересовани могу да пренесу своје замерке или 
предлоге помоћу интегрисаног система \textit{Issues} који прати све досадашње измене и на једном месту чува све 
већ стављене примедбе. Детаљно објашњење овог процеса постоји на интернет адреси на којој се збирка налази:
\begin{center}
    \url{https://github.com/djokicd/zbirka-sis}
\end{center}

Аутор ће бити веома захвалан свима који помогну развој збирке јер ово заједнички
пројекат чији је циљ квалитетнији наставни материјал за један веома прагматичан инжењерски предмет. 
Аутор ће се трудити да одржава стање ажурним у складу са датим примедбама, и захваљује се свима 
који су помогли збирку својим идејама или пријављивањем постојећих грешака. Ова збирка не би исто изгледала без вас. 

Поред неких задатака, на маргинама, постоје нарочите методичке ознаке које служе да нагласе природу тих задатака, то су 
\begin{itemize}[noitemsep, topsep=0pt]
    \item $\color{red}\clubsuit$ -- Задаци који представљају нарочито важне фундаменталне основе, које се често користе у 
    осталим задацима;
    \item \textcolor{red}{\warning} -- Сложенији задаци, у рангу испитних задатака, на које треба обратити нарочиту пажњу; и
    \item \textcolor{red}{\noway} -- Задаци који нису у оквиру градива предмета, али се служе релевантном 
    тематиком. Служе за продубљивање знања, и не спадају у градиво за испите. 
\end{itemize}

\vfill

\noindent
\begin{minipage}{0.2\textwidth}
    \includegraphics[width=\textwidth]{CC_BY-SA.png} 
\end{minipage}
\hfill
\begin{minipage}{0.75\textwidth}
    Збирка се објављује под лиценцом 
    \textit{Creative Commons 4.0 Ауторство - делити под истим условима}.
\end{minipage}
    