\section*{Предговор}

Збирка задатака \textit{„Сигнали и системи у инжењерству“} намењена је студентима предмета 
\textit{„Сигнали и системи“} на Електротехничком факултету Универзитета у Београду, 
на одсеку за Електронику и дигиталне системе. Заснована је на материјалу са рачунских вежби 
и испитних задатака, а у перспективи треба да прође званичну процедуру рецензије и 
постане акредитовано наставно средство. 

Аутор је тежио да збирка има методички ток: једноставнији задаци воде ка сложенијима, 
а значајан број њих илуструје примену градива на поједностављене инжењерске проблеме, 
претежно из области електротехнике. Студенте који примете да у том низу недостаје неки 
„корак“, макар и на основу сопственог искуства, аутор охрабрује да то поделе са њим. 

Вођен личним уверењима, аутор објављује збирку онлајн, бесплатно и отвореног кода. 
Студенти је могу користити као најажурнији материјал за вежбе, припрему предиспитних обавеза 
и самих испита. 

Изворни код збирке постављен је на \textit{GitHub-у}, где сви заинтересовани могу изнети 
своје примедбе и предлоге путем интегрисаног система \textit{Issues}. 
Овај систем прати све измене и на једном месту чува све примедбе. 
Детаљније упутство налази се на интернет адреси збирке:
\begin{center}
    \url{https://github.com/djokicd/zbirka-sis}
\end{center}

Аутор изражава захвалност свима који доприносе развоју збирке. Ово је заједнички пројекат 
чији је циљ да обезбеди квалитетнији наставни материјал за један веома прагматичан инжењерски предмет. 
У складу са добијеним сугестијама збирка ће бити редовно ажурирана. 
Посебна захвалност дугује се онима који су својим идејама или пријављивањем грешака помогли њено унапређење. 

Поред појединих задатака, на маргинама се налазе методичке ознаке које указују на њихов карактер:
\begin{itemize}[noitemsep, topsep=0pt]
    \item $\color{red}\clubsuit$ -- Задаци који представљају кључне фундаменталне основе, често коришћене у другим задацима;
    \item \textcolor{red}{\warning} -- Сложенији задаци, у рангу испитних, на које треба обратити посебну пажњу;
    \item \textcolor{red}{\noway} -- Задаци који превазилазе програм предмета, али су тематски повезани и намењени продубљивању знања 
    (нису део градива за испит).
\end{itemize}


\vfill

\noindent
\begin{minipage}{0.2\textwidth}
    \includegraphics[width=\textwidth]{CC_BY-SA.png} 
\end{minipage}
\hfill
\begin{minipage}{0.75\textwidth}
    Збирка се објављује под лиценцом 
    \textit{Creative Commons 4.0 Ауторство - делити под истим условима}.
\end{minipage}
    