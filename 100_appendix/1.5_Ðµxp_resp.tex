\chapter{Формуле екпоненцијалног одзива} \label{dod:exp_response}

\section*{Мотвација}

У општем случају, решавање нехомогених диференцијалних и диференцних једначина облика 
\begin{eqnarray}
    P(\DD) y(t) = x(t), \qquad\qquad P(\EE)y[n] = x[n],
\end{eqnarray}
подразумева одређивање партикуларног дела решења $y_{\rm p}(t)$ или $y_{\rm p}[n]$. 
Одређивање хомогеног дела решења је формални поступак заснован на структури  коренова карактеристичног полинома, као што је описано
у додатку \ref{dod:diferencne}. Међутим, када је у питању одређивање партикуларног дела, неретко је једини приступ паметно погађање
решења засновано на искуству. У овом додатку, увешћемо формуле које представљају партикуларни део веома широке класе побудних сигнала. 

Под експоненцијалним побудним сигналима подразумевамо све сигнале у природном облику:
\begin{eqnarray}
    x(t) = C \ee^{at}, \qquad\qquad \text{односно}  \quad\quad x[n] = C \uplambda^n,
\end{eqnarray}
где $C, a$, и $\uplambda$ у општем случају могу бити и комплексне константе. 

На овај начин, могу се описати и сигнали, који се можда на први поглед не чине екпоненцијалним у природи. На пример:
\begin{itemize}
    \item \emph{Константни сигнал:} У континуалном случају, ако је $a = 0$ тада је $x(t) = C$; односно, у дискретном случају, 
    ако је $\uplambda = 1$ тада је $x[n] = C$.
    \item \emph{Простопериодични сигнали:} У континуалном случају, ако је $a = \jj\upomega$ и ако је 
    $C = |C| \ee^{\jj\upphi}$, тада је 
    $x(t) = |C| \ee^{\jj\upphi} \ee^{\jj\upomega t} = |C|\cos(\upomega t + \upphi) + \jj |C| \sin(\upomega t + \upphi)$; односно, у дискретном случају, 
    ако је $\uplambda = \ee^{\jj\Upomega}$,  $C = |C| \ee^{\jj\upphi}$, тада је $y(t) =
    |C| \ee^{\jj\upphi} \ee^{\jj\Upomega n} = |C|\cos(\Upomega n + \upphi) + \jj C \sin(\upomega n + \upphi)$.
    Ови резутлати могу се искористити за одређивање одзива на побуде типа $\sin(\upomega t)$, или 
    $\cos(\upomega t)$ како је показано у задатку \ref{z:sin_cos_pobuda}.
    \item \emph{Пригушени експоненцијални сигнали:} Уколико на сличан начин као у претходној тачки, искористимо да је 
    $a = \upsigma + \jj\upomega$ односно $\uplambda = \Sigma \ee^{\jj\Upomega}$, добијамо да су 
    $x(t) = C\ee^{\upsigma t}\cos(\upsigma t)+ \jj C\ee^{\upsigma t}\sin(\upomega t)$, односно
    $x(t) = C\ee^{\Upsigma t}\cos(\Upomega n)+ \jj C\ee^{\Upsigma n}\sin(\Upomega t)$
\end{itemize}

\section{Формуле за континуалне сигнале}

За диференцијалну једначину у облику $P(\DD) y(t) = x(t)$ и експоненцијалну побуду облика 
$x(t) = C \ee^{at}$, партикуларни део ће бити облика који зависи од параметра $a$. Конректно, да ли параметар $a$ припада скупу 
коренова полинома $P(\DD)$. Према томе, разликујемо следеће случајеве
\begin{itemize}
    \item Ако параметар $a$ \emph{није корен} карактеристичног полинома $P(\DD)$, онда је партикуларни део 
    $$
        y_{\rm p}(t) = \dfrac{C\ee^{at}}{P(a)}.
    $$
    \item Ако је параметар $a$ \emph{једноструки корен} карактеристичног полинома $P(\DD)$, онда је партикуларни део 
    $$
        y_{\rm p}(t) = \dfrac{Ct\ee^{at}}{P'(a)}, \qquad\qquad P'(a) = \dfrac{\de P(\DD)}{\de \DD}\bigg\vert_{\DD = a}.
    $$
    \item Ако је параметар $a$ \emph{вишеструки корен} карактеристичног полинома $P(\DD)$, вишеструкости $q$, онда је партикуларни део 
    $$
    y_{\rm p}(t) = \dfrac{Ct^q\ee^{at}}{P^{(q)}(a)}, \qquad\qquad P^{(q)}(a) = \dfrac{\de^q P(\DD)}{\de \DD^q}\bigg\vert_{\DD = a}.
    $$
    Овај случај покрива и једноструки корен када је $q = 1$.
\end{itemize} 

\section{Формуле за дискретне сигнале}

За диференцну једначину у облику $P(\EE) y[n] = x[n]$, и експоненцијалну побуду облика 
$x[n] = C\uplambda^{n}$, партикуларни део ће бити облика који зависи од параметра $\uplambda$. Конректно, да ли параметар $\uplambda$ припада скупу 
коренова полинома $P(\DD)$. Према томе, разликујемо следеће случајеве
\begin{itemize}
    \item Ако параметар $\uplambda$ \emph{није корен} карактеристичног полинома $P(\EE)$, онда је партикуларни део 
    $$
        y_{\rm p}(t) = \dfrac{C \uplambda^n }{P(\uplambda)}.
    $$
    \item Ако је параметар $\uplambda$ \emph{једноструки корен} карактеристичног полинома $P(\EE)$, онда је партикуларни део 
    $$
        y_{\rm p}(t) = \dfrac{C n \uplambda^{n-1} }{P'(\uplambda)}, \qquad\qquad P'(a) = \dfrac{\de P(\EE)}{\de \EE}\bigg\vert_{\EE = \uplambda}.
    $$
    \item Ако је параметар $\uplambda$ \emph{вишеструки корен} карактеристичног полинома $P(\EE)$, вишеструкости $q$, онда је партикуларни део 
    $$
    y_{\rm p}(t) = \dfrac{Cn^{(q)}\uplambda^{n - q} }{P^{(q)}(\uplambda)}, \qquad\qquad P^{(q)}(\uplambda) = \dfrac{\de^q P(\EE)}{\de \EE^q}\bigg\vert_{\EE = \uplambda},
    $$
    где је $n^{(q)} = n(n-1)\cdots(n-q+1)$. Овај случај покрива и једноструки корен када је $q = 1$.
\end{itemize} 
