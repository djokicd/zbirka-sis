
\chapter{Одређивање импулсног одзива континуалних 
\textit{LTI} система} \label{a:impulsni_odziv}

Циљ додатка је да обједини и појасни на примерима
методе за одређивање импулсног одзива континуалних 
\textit{LTI} система. 


\section{Поједностављење опште форме}

Посматрајмо континуални \textit{LTI} систем описан 
диференцијалном једначином облика
\begin{equation}
 P({\rm D}) y(t) = Q({\rm D}) x(t),
 \label{def}
\end{equation}
где су $x(t)$ и $y(t)$ побуда и одзив система редом, а  $P({\rm D})$ и $Q({\rm D})$ произвољни полиноми 
по оператору диференцирања $\rm D$. Импулсни  
одзив овог система $h = h(t)$ представља одзив система на 
јединичну импулсну
побуду $x(t) = \updelta(t)$, односно решење једначине
\begin{equation}
 P({\rm D}) h(t) = Q({\rm D}) \updelta(t).
 \label{phqd}
\end{equation}
Уколико приметимо смену 
\begin{equation}
\boxed{
 h(t) = Q({\rm D}) h_1(t)
} \label{hqh1}
\end{equation}
и заменимо у израз \eqref{phqd}, даље се може писати
\begin{equation}
P({\rm D}) h(t) = Q({\rm D}) \updelta(t) 
\enspace\Leftrightarrow\enspace
P({\rm D}) Q({\rm D}) h_1(t) = Q({\rm D}) \updelta(t) 
\enspace\Leftrightarrow\enspace
Q({\rm D}) P({\rm D}) h_1(t) = Q({\rm D}) \updelta(t). 
\end{equation}
У последњем кораку, начињена је замена редоследа 
примене оператора, $P({\rm D}) Q({\rm D})
\equiv Q({\rm D}) P({\rm D})$, која је 
оправдана на основу линеарности. У последњем 
кораку се може на обе стране применити оператор
$Q^{-1}({\rm D})$, чију егзистенцију овде нећемо
дискутовати, након чега преостаје резултат:
\begin{equation}
\boxed{
P({\rm D}) h_1(t) = \updelta(t). 
} \label{eq2}
\end{equation}

\noindent
\ovalbox{
\begin{minipage}{0.99\textwidth}
На основу претходно изнесеног поступка, могуће је 
одредити импулсни одзив система описаног једначином
\eqref{def} тако што се одреди помоћни одзив 
$h_1(t)$ помоћног система описаног једначином  
\eqref{eq2} a потом трансформацијом добијеног 
помоћног одзива у одзив полазног система помоћу 
\eqref{hqh1}
\end{minipage}
} 

\vspace*{2mm}
Додатно, из практичних 
разлога постоји ограничење у степенима
полинома $\deg P \geq \deg Q$.  
Уколико се импулсни одзив помоћног система запише 
у облику $h_1(t) = g(t)\,{\rm u}(t)$, онда се
могу разликовати случајеви:
\begin{equation}
h(t) = 
\left\{
\begin{array}{ll}
Q(D)\bigl( g(t){\rm u}(t) \bigr), & \deg P = \deg Q \\
Q(D)\bigl( g(t)\bigr)\, {\rm u}(t) , & \deg P > \deg Q \\
\end{array}
\right.
\end{equation}


\section{Одређивање импулсног одзива за
поједностављену форму}

На основу претходне дискусије, потребно је и довољно
одредити решење једначине 
$P({\rm D}) h_1(t) = \updelta(t)$. Препоручена 
метода
за решавање овог проблема је (\textit{i}) 
диференцирањем одскочног одзива (дискутовано на предавањима), али се може користити и 
(\textit{ii}) 
поједностављена
метода уклапања импулса (енг. 
\textit{impulse matching}, дискутовано на вежбама).
\\[1mm]
\indent
Прва метода утемељена је на својству линеарности 
система. Нека је $s_1(t) = O\{ {\rm u}(t) \}$ 
одскочни
одзив посматраног система, онда се диференцирањем 
обе стране изналази
\begin{equation}
\dfrac{{\rm d}s_1(t)}{{\rm d}t} = 
\dfrac{{\rm d}}{{\rm d}t} O\{ {\rm u}(t) \}
= 
 O\left\{ 
 \dfrac{{\rm d}}{{\rm d}t} {\rm u}(t) \right\}
 = O\{\updelta(t)\} = h_1(t),
\end{equation}
при чему је у другом кораку замењен редослед 
примене оператора диференцирања и система због
линеарности. Коначно се има закључак 
$
\boxed{
\dfrac{{\rm d}s_1(t)}{{\rm d}t} =  h_1(t)
},
$ 
односно, \textbf{импулсни одзив се добија диференцирањем  одскочног одзива}. Одређивање одскочног одзива 
представља решавање диференцијалне једначине 
$P({\rm D}) s_1(t) = {\rm u}(t)$. Велика предност 
у решавању на овај начин у односу на директно 
решавање једначине \eqref{eq2} јесте то да у њој 
\myul{нема импулса} што за последицу 
има то да је одскочни одзив непрекидан, тако да су
једнаке преиницијалне и постиницијалне вредности 
за $s_1(t)$. Будући да се одскочни одзив 
одређује за преиницијалне услове равне нули 
то повлачи да су онда и постиницијални услови 
равни нули. 
Одскочни одзив добија се као збир хомогеног и 
партикуларног дела $s_1 = s_{1\rm h} + s_{1\rm p}$. 
Хомогени део се одређује на начин показан на часу.
Партикуларни део
представља устаљени одзив на 
експоненцијалну побуду ${\rm e}^{0\cdot t} 
{\rm u}(t)$ и на основу дискусије са часа вежби 
једнак је $s_{\rm 1p} = \dfrac{1}{P(0)}$.
\\[1mm]
\indent
Друга метода утемељена је на особини линеарне 
независности различитих извода Диракових импулса.
%\begin{crtice} 
Може се показати да једначина облика:
\begin{equation}
a_0 \updelta(t) + 
a_1 \updelta'(t) + 
a_2 \updelta''(t) + 
\cdots
+ a_n \updelta^{(n)}(t) = 0, \quad (n \in \mathbb N)
\end{equation} 
по непознатим коефицијентима $a_0, a_1, 
\ldots, a_n \in \mathbb R$ има \myul{само} тривијално 
решење $a_0 = a_1 = a_2 = \cdots = a_n = 0$. 
У контексту решавања диференцијалне једначине 
облика $P({\rm D}) h_1(t) = \updelta(t)$ то значи да:
(\textit{i}) пошто се са десне стране налази делта 
импулс мора се налазити и са леве стране и 
(\textit{ii}) пошто се са десне стране не налазе 
изводи делта импулса њега не може бити ни са леве 
стране. На основу тога, у изразу 
$P({\rm D}) h_1(t)$ се мора појавити делта импулс
и не сме се појавити његов први извод. Претпоставимо 
да се сабирак $f(t) \updelta(t)$ јавља у $k$-том 
изводу импулсног одзива, $h_1^{(k)}(t)$, у том 
случају се у $(k+1)$-вом изводу импулсног 
одзива мора наћи сабирак облика 
$f'(t) \updelta(t) + {f(t) \updelta'(t)}$.
Односно, да се не би са леве стране појавио извод 
Дираковог импулса, неопходно је да се импулс 
појављује тек у највишем изводу импулсног одзива
који се јавља у једначини а то је $h^{(\deg P)}(t)$,
где је $\deg P$ степен полинома $P$ -- ред 
диференцијалне једначине. Будући да се импулс јавља
у $h^{(\deg P)}(t)$ то се Хевисајдова одскочна 
функција мора јављати у $h^{(\deg P - 1)}(t)$, односно
до прекида долази у $(\deg P - 1)$-вом изводу. Будући
да су интеграли Хевисајдове функције непрекидни, то 
је онда тај и једини извод импулсног одзива који  има 
прекид. На основу тога, ако распишемо 
оператор система као 
$P({\rm D}) = c_0 + c_1{\rm D} + c_2{\rm D}^2 + 
\cdots + c_2{\rm D}^n$ једначина се може писати као
\begin{eqnarray}
& (c_0 + c_1{\rm D} + c_2{\rm D}^2 + 
\cdots + c_2{\rm D}^n) h_1(t) = \updelta (t)  \\
& c_0 h_1(t) +
c_1 h_1'(t) +
c_2 h_1''(t) +
\cdots
+ c_n h_1^{(n)}(t)
  = \updelta (t)
\end{eqnarray}
Ако интегралимо обе стране добијене једначине
$\int_{-\upepsilon}^{+\upepsilon}$ у
произвољно „уским“ границама, $\upepsilon \to 0$, 
приметивши да онда интеграли свих ограничених функција
теже нули преостаје само члан са импулсом:
\begin{align} 
    \DS
&
\cancel{ \DS
c_0 \int_{-\upepsilon}^{+\upepsilon}h_1(t) \,\de t
} +
\cancel{ \DS
c_1 \int_{-\upepsilon}^{+\upepsilon}h_1'(t) \,\de t
} +
%c_2 
%\cancel{ \DS \int_{-\upepsilon}^{+\upepsilon}h_1''(t) \,\de t 
%} +
\cdots
+ c_{n-1}  
\cancel{ \DS \int_{-\upepsilon}^{+\upepsilon}h_1^{(n-1)}(t) \,\de t}
% } + c_n \int_{-\upepsilon}^{+\upepsilon} h_1^{(n)}(t) \,\de t 
  = 
\underbrace{  \DS
  \int_{-\upepsilon}^{+\upepsilon} \updelta (t)\,\de t
}_{=1, \text{по деф.}} \nonumber
\Rightarrow \\[-2mm]
& 
c_n h_1^{(n-1)}(0^+)
-
c_n h_1^{(n-1)}(0^-) = 1 
\end{align} 
Како су преиницијални услови приликом тражења
импулсног одзива равни нули то преостаје 
$h_1^{(n-1)}(0^+) = \dfrac{1}{c_n}$.
%\end{crtice}
\noindent
\textbf{Коначно су непосредно 
познати сви постиницијални услови импулсног одзива
}
\begin{equation}
\boxed{
h_1^{}(0^+) = 0,
\enspace h_1^{'}(0^+) = 0,
\enspace h_1^{''}(0^+) = 0,
\ldots, 
\enspace h_1^{(n-2)}(0^+) = 0,
\enspace h_1^{(n-1)}(0^+) = \dfrac{1}{c_n},
}
\end{equation}
где је $c_n$ коефицијент уз највиши извод у
диференцијалној једначини а $n$ је ред диференцијалне
једначине.

\vspace*{10mm}



\noindent\textbf{Пример 1.} Одредити одзив система 
описаног диференцијалном једначином 
$y''(t) + 3 y'(t) + 2y(t) = x'(t) + 2x(t) $, 
на побуду $x(t) = 2e^t\,{\rm u}(t-2)$. \\[1mm]
%

\noindent\textbf{Решење.} Одзив на дату побуду $x(t)$
може се одредити конволуцијом. За примену конволуције
потребно је прво одредити импулсни одзив система. 
Систем се може записати у облику \eqref{def} за
$P({\rm D}) = {\rm D}^2 + 3 {\rm D} + 2$ и 
$Q({\rm D}) = {\rm D} + 2$. Потражимо импулсни одзив
помоћног система $P({\rm D}) h_1(t) = \updelta(t)$.

%\begin{flushright}
%\begin{minipage}{0.9\textwidth}
\vspace*{2mm}
\noindent
\textbf{1. метода} (\textit{диференцирањем импулсног 
одзива}). Импулсни одзив помоћног система 
тражимо као решење једначине 
\begin{equation}
P({\rm D}) s_1(t) = {\rm u}(t).
\label{eq:step}
\end{equation} Импулсни одзив
има хомогени део одређен коренима полинома $P$ и то 
$\uplambda \in \{-2, -1\}$. Облик хомогеног решења 
је онда $s_{\rm 1,h} = C_1 {\rm e}^{-t} + C_2
{\rm e}^{-2t}$. Партикуларни део је 
$s_{\rm 1,p} = \dfrac{1}{P(0)} = \dfrac{1}{2}$. 
Коефицијенти у општем облику једначине одскочног одзива
%$
%s_1(t) = C_1 {\rm e}^{-t} + C_2
%{\rm e}^{-2t} + \dfrac{1}{2}
%$
налазе се на основу постиницијалних почетних услова 
који су из раније наведених разлога равни нули. 
\begin{eqnarray}
s_1(t) = C_1 {\rm e}^{-t} + C_2
{\rm e}^{-2t} + \dfrac{1}{2} \Rightarrow &
s_1(0) = 0 = C_1 + C_2 + \dfrac{1}{2} \\
s_1'(t) = -C_1 {\rm e}^{-t} - 2 C_2
{\rm e}^{-2t}  \Rightarrow &
s_1'(0) = 0 = -C_1 -2 C_2 .
\end{eqnarray}
Решавањем добијеног система једначина по 
непознатим коефицијентима добија се 
$C_1 = -1$, $C_2 = \dfrac{1}{2}$. Одакле се 
коначно налази одскочни одзив 
$
s_1(t) = 
\left(	
-{\rm e}^{-t} + \dfrac{1}{2}{\rm e}^{-2t} + 
\dfrac{1}{2} \right) {\rm u}(t).
$ 
Диференцирањем добијеног одскочног одзива 
добија се и импулсни одзив помоћног система
$
\boxed{
h_1(t) =
\left(
{\rm e}^{-t}  - {\rm e}^{-2t} 
\right) \, {\rm u}(t)
}
$

\vspace*{2mm}
\noindent
\textbf{2. метода} (\textit{поједностављеном
методом уклапања импулса}) Непосредно се решава 
једначина $P({\rm D}) h_1(t) = \updelta(t)$. 
Општи облик импулсног одзива одређен је коренима 
полинома $P$ на исти начин као у претходној 
методи,
$h_1(t) = C_1 {\rm e}^{-t} + C_2 {\rm e}^{-2t}$.
На основу закључка методе познати су постиницијални почетни 
услови као $h_1(0^+) = 0$, $h_1'(0^+) = 1$. 
Заменом у општи облик добија се:
\begin{eqnarray}
h_1(t) = C_1 {\rm e}^{-t} + C_2
{\rm e}^{-2t}  \Rightarrow &
h_1(0^+) = 0 = C_1 + C_2  \\
h_1'(t) = -C_1 {\rm e}^{-t} -2 C_2
{\rm e}^{-2t}  \Rightarrow &
h_1'(0^+) = 1 = -C_1 - 2 C_2,  \\
\end{eqnarray}
решавањем добијеног система имају се $C_1 = -C_2 = 1$,
одакле је $\boxed{
h_1(t) =
\left(
{\rm e}^{-t}  - {\rm e}^{-2t} 
\right) \, {\rm u}(t)
}$
%\end{minipage}
%\end{flushright}

\vspace*{5mm}
\noindent
Имајући импулсни одзив помоћног система, импулсни
одзив полазног система налази се на основу 
\eqref{hqh1}, одакле је 
\begin{equation}
h(t) = Q({\rm D}) h_1(t) \Rightarrow 
\boxed{ h(t) = 
{\rm e}^{-t}
 \,{\rm u}(t) }. \label{eq:hodh1}
\end{equation}
\noindent
Одзив на побуду може се потражити конволуцијом 
\begin{equation}
y_{\rm p}(t) = h(t) \ast x(t). \label{konv}
\end{equation} 
У овом случају
је најефикасније применити својства конволуције. 
Побудни сигнал се може записати и као
$x(t) = 2{\rm e}^{t {\color{blue} - 2 + 2}}
\,{\rm u}(t-2) = 2{\rm e}^2 {\rm e}^{t-2} {\rm u} (t-2)$.
Приметимо да је побудни сигнал временски померен 
и скалиран у односу на сигнал $x_1 = e^{t} \, {\rm u}(t)$. Да би се 
израчунала конволуција \eqref{konv} згодно је 
искористити ово својство тако да се конволуција 
одговарајућим трансформацијама своди на табличну, 
или једноставнију
конволуцију. На основу таблице је 
$y_{\rm p,1} = h(t) \ast x_1(t) = 
{\rm e}^{-t} {\rm u}(t) \ast {\rm e}^{t} {\rm u}(t)
= \dfrac{{\rm e}^t - {\rm e}^{-t} }{2} 
{\rm u}(t) = {\rm sinh}(t) \, {\rm u}(t).
$
На основу линеарности и временске инваријантности, 
пошто важи $x(t) = 
2{\rm e}^2 x_1(t - 2)$ то је и 
$y_{\rm p}(t) = 
2{\rm e}^2 y_{\rm p,1}(t - 2)$ одакле је 
\begin{equation}
\boxed{
\boxed{
y_{\rm p}(t) = 2{\rm e}^2 {\rm sinh}(t-2) \, {\rm u}(t-2)
}
}
\end{equation}
\begin{flushright}
$\blacksquare$
\end{flushright}
