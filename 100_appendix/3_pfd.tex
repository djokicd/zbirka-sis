
\chapter{Растављање преносне функције на парцијалне разломке} \label{a:pfd}

Од значајног интереса за одређивање различитих инверзних трансформација јесте поступак растављања преносне функције 
облика $H(s) = \dfrac{Q(s)}{P(s)}$, где су $P(s)$ и $Q(s)$ одговарајући полиноми са 
\myul{реалним} коефицијентима, на \textit{парцијалне разломке}. 
У том случају, корене полинома $Q(s)$ називамо \textit{нулама} $z_i$, а корене полинома $P(s)$ \textit{половима} $p_i$, 
те преносне функције. У општем случају се таква функција преноса може представити у облику 
\begin{equation}
    H(s) = k \dfrac{ (s - z_1)(s-z_2)\cdots(s - z_m) }{ (s - p_1)(s - p_2)\cdots(s - p_m) }.
\end{equation}
При чему се $k$ онда назива појачањем система. У даљој дискусији, фактор $k$ не прави разлику па ће се анализирати 
случај када је $k = 1$. Под парцијалним разломцима, подразумевамо суму чланова код којих се у имениоцу налазе 
полиноми који се не могу даље факторисати, или који представљају вишеструке полове функције преноса. У том смислу, 
размотрићемо неколико различитих специјалних случајева. 

\subsection*{Преносна функција са различитим реалним коренима}

Посматрајмо преносну функцију облика 
$H(s) = \dfrac{Q(s)}{(s-p_1)(s-p_2)\cdots(s-p_n)}$, код које су сви $p_i$ различити реални корени, а важи да је 
$\deg Q < n$. Онда се таква функција може представити у облику
\begin{equation}
    H(s) = \dfrac{Q(s)}{(s-p_1)(s-p_2)\cdots(s-p_n)} = \dfrac{A_1}{s - p_1} + \dfrac{A_2}{s - p_2} + \cdots + \dfrac{A_n}{s - p_n},
\end{equation}
за шта је потребно одредити коефицијенте развоја $A_1$, $A_2$, \ldots, $A_n$. Да бисмо израчунали коефицијент 
$A_1$ можемо да обе стране израза помножимо са $(s-p_1)$, ако је $s \neq p_1$ па се онда има резултат
\begin{equation}
    \dfrac{Q(s)}{(s-p_2)\cdots(s-p_n)} = A_1 + \dfrac{(s-p_1)A_2}{s - p_2} + \cdots + \dfrac{(s-p_1)A_n}{s - p_n},
\end{equation}
Уколико сада са обе стране потражимо граничну вредност израза $\lim_{s\to p_1}$, приметимо да је онда 
$s - p_1 \to 0$, док се лева страна израза може једноставно израчунати.\
\begin{equation}
    \underbrace{\dfrac{Q(p_1)}{(p_1-p_2)\cdots(p_1-p_n)}}_{\text{За $s \to p_1$}}
    = A_1 + \dfrac{\cancelto{0}{(s-p_1)}A_2}{s - p_2} + \cdots + \dfrac{\cancelto{0}{(s-p_1)}A_n}{s - p_n} = A_1.
\end{equation}
На основу овог поступка може се формулисати општи сликовити поступак за извлачење коефицијента $A_k$.

\begin{center}
\shadowbox{
\begin{minipage}{0.85\textwidth}
Уколико је  
$$\dfrac{Q(s)}{(s-p_1)\cdots(s-p_k)\cdots(s-p_n)} = \dfrac{A_1}{s - p_1} + \cdots + 
\dfrac{A_k}{s - p_k}  + \cdots + \dfrac{A_n}{s - p_n},$$
за $p_i$ различите реалне вредности, онда се сваки коефицијент $A_k$ може израчунати „прикривањем“ (односно уклањањем) члана $s - p_k$ из израза, 
па заменом $s \mapsto p_k$ у остатак, као 
$$
    A_k = \dfrac{Q(s)}{(s-p_1)\cdots\colorbox{gray!50}{$\xcancel{(s-p_k)}$}\cdots(s-p_n)} \bigg|_{s = p_k}.
$$
Овај поступак основа је Хевисајдовог метода прикривања (енг. \textit{Heaviside cover-up method}).
\end{minipage}
}
\end{center}

\subsection*{Пар комплексно конјугованих полова.}

Размотримо преносну функцију која поред једног реалног пола, $p$, има и један пар комплексно конјугованих полова
$p_{2,3} = \upsigma \pm \jj\upomega$, а која нема нула. Ради дискусије, пођимо од облика растављеног на парцијалне разломке. 
%
\begin{eqnarray}
    H(s) 
    &=& 
    \dfrac{A}{s - p} + 
    \dfrac{B_1}{s - (\upsigma + \jj\upomega)} + 
    \dfrac{B_2}{s - (\upsigma - \jj\upomega)} 
    \\
    &=& 
    \dfrac{A}{s - p} + 
    \dfrac{
        B_1 \bigl( s - (\upsigma - \jj\upomega) \bigr)
        +
        B_2 \bigl( s - (\upsigma + \jj\upomega) \bigr)
     }{s^2 + 2 \upsigma s + \upomega^2 + \upsigma^2}.  \\[2mm] 
     %
     &=&
     \dfrac{A}{s - p} + 
     \dfrac{
         (B_1 + B_2)(s - \upsigma) +   
         (B_1 - B_2)\jj\upomega
      }{s^2 + 2 \upsigma s + \upomega^2 + \upsigma^2}. 
\end{eqnarray}
%
Будући да разматрамо само функције преноса са полиномима са реалним коефицијентима, то мора бити 
$\Im{B_1 + B_2} = 0$ и $\Re{B_1 - B_2} = 0$, па закључујемо да коефицијенти $B_1$ и $B_2$ морају 
имати конјуговано-комплексну симетрију $B = B_1 = B_2^\ast$. Коефицијент $B$ је онда могуће 
одредити на начин који аналоган поступку у претходној тачки
\begin{eqnarray}
    H(s) 
    &=& 
    \dfrac{A}{s - p} + 
    \dfrac{B}{s - (\upsigma + \jj\upomega)} + 
    \dfrac{B^\ast}{s - (\upsigma - \jj\upomega)} \qquad | \times \bigl( s + (\upsigma - \jj\upomega) \bigr)
    \\
    \bigl( s - (\upsigma + \jj\upomega) \bigr)
    H(s) 
    &=& B + 
    \dfrac{
        B^\ast \bigl( s - (\upsigma + \jj\upomega) \bigr)
    }{s - (\upsigma - \jj\upomega)}
    \Rightarrow
    B = \lim_{{s \to (\upsigma + \jj\upomega)}}  \bigl( s - (\upsigma + \jj\upomega) \bigr)
    H(s) 
\end{eqnarray}
На основу тога можемо да установимо правило за издвајање парцијалних разломака, који одговарају 
комплексно конјугованим коренима 

\begin{center}
    \shadowbox{
    \begin{minipage}{0.85\textwidth}
    Уколико је  
    $$\dfrac{Q(s)}{\cdots\bigl(s-(\upsigma + \jj\upomega)\bigr)\bigl(s-(\upsigma - \jj\upomega)\bigr)\cdots} = 
    \cdots + 
    \dfrac{A_k}{s-(\upsigma + \jj\upomega)}  + 
    \dfrac{A_k^*}{s-(\upsigma - \jj\upomega)}  
    \cdots$$
    онда се сваки коефицијент $A_k$ може израчунати „прикривањем“ (односно уклањањем) члана 
    $s-(\upsigma + \jj\upomega)$ из израза, 
    па заменом $s \mapsto p_k$ у остатак, као 
    $$
        A_k = 
        \dfrac{Q(s)}{\cdots
        \colorbox{gray!50}{$\xcancel{
                \bigl(s-(\upsigma + \jj\upomega)\bigr)
        }$}
        \underbrace{
            \bigl(s-(\upsigma - \jj\upomega)\bigr)
        }_{\jj 2 \upomega}
        \cdots}
        \bigg|_{s = \upsigma + \jj\upomega}.
    $$
    Том приликом, по аутоматизму се добија коефицијент $A_k^\ast$ и рачунање није потребно понављати, за други корен.
    \end{minipage}
    }
    \end{center}

Када је на тај начин одређен  коефицијент $A_k$, приметимо да се даље може писати\footnote{Користе се и идентитети 
$z + z^* = 2\Re{z}$ и 
$z - z^* = \jj 2 \Im{z}$, за $z \in \mathbb C$. }
\begin{eqnarray}
    \dfrac{A_k}{s-(\upsigma + \jj\upomega)}  + 
    \dfrac{A_k^*}{s-(\upsigma - \jj\upomega)}  
    &=&  
    \dfrac{ A_k(s-(\upsigma - \jj\upomega)) + A_k^* (s-(\upsigma + \jj\upomega))}{(s-\upsigma)^2 + \upomega^2}  
    \\ 
    &=&
    \dfrac{ (A_k + A_k^*) (s - \upsigma) + (A_k - A_k^*) (-\jj\upomega)   }{(s-\upsigma)^2 + \upomega^2}     
    \\
    &=&
    2\,\dfrac{ \Re{A_k} (s - \upsigma) + \jj \Im{A_k} (-\jj\upomega)   }{(s-\upsigma)^2 + \upomega^2}     
    \\
    &=&
    2\,\dfrac{ \Re{A_k} (s - \upsigma) + \Im{A_k} \upomega   }{(s-\upsigma)^2 + \upomega^2},    
\end{eqnarray}
што може бити погодно имајући у виду табличне резултате \reft{T:LT:exp_sin} и \reft{T:LT:exp_cos}

\subsection*{Поступак за вишеструке полове}

Чланови који одговарају вишеструким половима могу се расписати према 
\begin{equation}
    H(s) = \dfrac{Q(s)}{\cdots(s - p)^n\cdots} = 
    \cdots
    +
    \dfrac{A_1}{s-p} 
    +
    \dfrac{A_2}{(s-p)^2}
    +
    \cdots
    + 
    \dfrac{A_n}{(s-p)^n}
    +
    \cdots
    .
\end{equation}
Коефицијент $A_n$ се може наћи како је показано у ранијим одељцима, а формуле за рачунање коефицијената 
$A_1, \ldots, A_{n-1}$ дате су обрасцем 
$A_k = \lim_{s \to p} \dfrac{1}{(k-1)!} \dfrac{\de^{k-1}}{\de s^{k-1}} \bigl( (s-p)^n H(s) \bigr)$. 
Рачунање резултата на основу овог израза може бити значајно сложеније и од сређивања полинома па њега нећемо користити. 
У форми парцијалних разломака ћемо израчунати коефицијенте испред остатака првог реда, а остале коефицијенте можемо брзо 
наћи променом методе неодређених коефицијената, заменом за неку (било коју) конкретну вредност $s$. Илуструјмо поступак на примеру
$H(s) = \dfrac{1}{(s+1)(s+2)^2}$.

Дати облик се може представити као 
\begin{equation}
    H(s) = \dfrac{A}{s + 1} + \dfrac{B}{s+2} + \dfrac{C}{(s+2)^2}.
\end{equation}
Коефицијенте $A$ и $B$ можемо директно одредити прикривањем као 
\begin{eqnarray}
    A = \dfrac{1}{ \xcancel{(s+1)}(s+2)^2} \bigl|_{s = -1} = 1 \qquad
    B = \dfrac{1}{ {(s+1)}\xcancel{(s+2)^2}} \bigl|_{s = -2} = -1
\end{eqnarray}
На основу ових резултата знамо да је
$ \dfrac{1}{(s+1)(s+2)^2} = \dfrac{1}{s + 1} + \dfrac{B}{s+2} - \dfrac{1}{(s+2)^2}$. Па једини преостали коефицијент 
$B$ можемо наћи заменом $s \not\in \{-1, -2\}$, на пример, $s = 0$ 
\begin{equation}
    \dfrac{1}{4} = 1 + \dfrac{B}{2} - \dfrac{1}{4} \Rightarrow B = -1, 
\end{equation}
чиме се коначно има 
$ \dfrac{1}{(s+1)(s+2)^2} = \dfrac{1}{s + 1} - \dfrac{1}{s+2} - \dfrac{1}{(s+2)^2}$.