\renewcommand{\thechapter}{S}
\setcounter{section}{0}
\chapter{Важна својства трансформација} \label{a:svojstva}


\section*{Конволуција континуалних и дискретних сигнала}
Конволуција континуалних, односно, дискретних сигнала дефинише се као 
\begin{equation}
    x(t) \ast y(t) = \int_{\mathclap{\uptau = -\infty}}^{\infty} x(\uptau) y(t-\uptau) \de \uptau, 
    \qquad
    x[n] \ast y[n] = \sum_{\mathclap{m = -\infty}}^{\infty} x[m] y[n-m] 
\end{equation}
Без обзира на то да ли се односи на континуалне или дискретне сигнале, за конволуцију важе:
\begin{itemize}\itemsep0pt
    \item \emph{Комутативност:} $x \ast y = y \ast x$;
    \item \emph{Асоцијативност:} $(x \ast y) \ast z = x \ast (y \ast z)$;
    \item \emph{Дистрибутивност у односу на сабирање:} $x \ast (y + z) = (x \ast y) + (x \ast z)$;
    \item \emph{Асоцијативност у односу на скаларно множење:} $k \cdot (x \ast y) = (kx) \ast y$; 
    \item \emph{Постојање неутралног елемента} $f \ast \updelta = f$; 
    \item \emph{Комплексна конјукција} $(x \ast y)^\ast = x^{\ast} \ast y^{\ast}$.
\end{itemize}

\subsection*{Кружна конволуција}
Под кружном конволуцијом два периодична сигнала (са периодом $T$ у континуалном случају, или 
периодом $N$ у дискретном), подразумевају се 
\begin{equation}
    x(t) \circledast y(t) = \int_{\mathclap{\uptau = \langle T \rangle }} x(\uptau) y(t-\uptau) \de \uptau, 
    \qquad
    x[n] \circledast y[n] = \sum_{\mathclap{m = \langle N \rangle }} x[m] y[n-m] 
\end{equation}
Том приликом смисао овако написаних граница интеграције јесте интервал дужине периода, односно, од $t_0$ до $t_0 + T$. 
Дуално, смисао граница сумирања јесте такође дужина интервала, односно до $n_0$ до $n_0 + N - 1$. 

\section*{Фуријеови редови континуалних сигнала} \label{d:CTFS}

У општем случају, уколико континуални сигнал $x(t)$, са периодом $T$, задовољава Дирихлеове услове, тада се он може развити у ред облика 
\begin{eqnarray}
    x(t) &=& \sum_{k = -\infty}^{\infty} X[k] \ee^{\jj k \upomega_{\rm F}t } 
    \qquad\qquad\qquad\qquad\qquad\qquad\quad
    {\text{(комплексни облик), или}} \\
         &=&
    A[0] + \sum_{k = 1}^{\infty} A[k] \cos(\upomega_{\rm F} k) + \sum_{k = 0}^{\infty} B[k]  \sin(\upomega_{\rm F} k) 
    \quad \text{(тригонометријски облик)},
\end{eqnarray}
где је $\upomega_{\rm F}$ кружна учестаност основног хармоника. $X[k]$ представљају комплексне а $A[k]$ и $B[k]$ тригонометријске (реалне) 
коефицијенти уколико се разматра реалан сигнал $x(t)$. Комплексни коефицијенти могу се израчунати помоћу корелације са комплексним 
сигналима Фуријеовог базиса $\{ \ee^{\jj k \upomega_{\rm F}t} \}$:
\begin{eqnarray}
    \FS{X[k]} = X[k] = \dfrac{1}{T_{\rm F}}\int_{  \mathclap{\langle T_{\rm F} \rangle } } x(t)\,\ee^{-\jj k \upomega_{\rm F} t} \, \de t 
\end{eqnarray},
а тригонометријски коефицијенти се могу израчунати помоћу корелације са тригонометријским функцијама:
\begin{eqnarray}
     A[0] = \dfrac{1}{T_{\rm F}} \int_{ \langle T_{\rm F} \rangle } x(t) \, \de t ,
    \qquad && 
    A[k > 0] = \dfrac{2}{T_{\rm F}} 
    \int_{\mathclap{\langle T_{\rm F} \rangle}  } 
    x(t) \cos(k \upomega_{\rm F} t ) \\
    && B[k > 0] = \dfrac{2}{T_{\rm F}} 
    \int_{ \mathclap{\langle T_{\rm F} \rangle} } 
    x(t) \sin(k \upomega_{\rm F} t ), \label{eq:ctfs_B_def}
\end{eqnarray}
Између комплексних и тригонометријских 
коефицијената постоје везе
\begin{equation}
    X[k \neq 0] = \dfrac{A[k] - \jj B[k]}{2}, \qquad X[0] = A[0], \label{eq:ctfs_ab_to_x}
\end{equation}
које се могу показати полазећи од Ојлерове формула $\ee^{\jj x} = \cos(x) + \jj\sin(x)$.  
Уколико се разматра реални сигнал $x(t)$, сигнали $A[k]$ и $B[k]$ су реални, па важе и инверзне релације 
\begin{eqnarray}
    A[0] = X[0] && 
    A[k > 0] = 2 \, \Re{X[k]} \qquad
    B[k > 0] = -2 \, \Im{X[k]}. \label{eq:ctfs_x_to_ab}
\end{eqnarray}

Уколико је сигнал $x(t)$ реалан, његов комплексни спектар задовољава комплексно-конјуговану (Хермитову) симетрију 
$X[-k] = X^{\ast}[k]$. \\[1mm]

\noindent 
Парсевалова теорема даје везу између \myul{снаге} континуалног сигнала и коефицијената
његовог развоја у Фуријеов ред:
\begin{equation}
P_x = \dfrac{1}{T} \int_{0}^{T} |x^2(t)|\,\de t = \sum_{k=-\infty}^{\infty} |X[k]|^2. \label{eq:ctfs_parseval}
\end{equation}

Коефицијенти сигнала $x(t)$ (и $y(t)$), чији је развој у Фуријеов ред
$X[k] = \FS{x(t)}$ (и $Y[k] = \FS{y(t)}$),
задовољавају наредне значајне особине: 
\begin{itemize}\itemsep0pt
    \item \emph{Линеарност:}  $\FS{ax(t) + by(t)} = a \FS{x(t)} + b \FS{y(t)}$, $a,b = \const$;
    \item \emph{Померај у времену:} $ \FS{x(t-\uptau)} = \FS{x(t)} \ee^{-\jj k \upomega_{\rm 0} \uptau}$, $\uptau = \const$;
    \item \emph{Инверзија временске осе:} $\FS{x(t)} = X[k] \Rightarrow \FS{x(-t)} = X[-k] $;
    \item \emph{Кружна конволуција\footnote{Односи се на конволуцију, само на заједничком периоду $T_0$ за оба сигнала.}}
    $
    \FS{x(t) \circledast y(t)} = T_0 X[k] Y[k];
    $
    \item \emph{Производ\footnote{Односи се на конволуцију, само на заједничком периоду $T_0$ за оба сигнала.}}
    $
    \FS{x(t) \cdot y(t)} = X[k] \ast Y[k];
    $
    \item \emph{Дејство на извод} $\FS{ \dfrac{\de x(t)}{\de t} } = \jj k \upomega_0 X[k]$;
    \item \emph{Дејство на интеграл} $\FS{ \int_{\uptau = -\infty}^t x(\uptau) \, \de \uptau } = \dfrac{1}{\jj k \upomega_0} X[k]$;
\end{itemize}

\section*{Фуријеови редови дискретних сигнала}

Дискретан сигнал $x[n]$ са периодом $N_{\rm F}$ може се представити преко коефицијената Фуријеовог 
реда дискретног сигнала као
\begin{eqnarray}
    x[n] = \sum_{k = \langle N_{\rm F} \rangle } X[k] \ee^{ \jj k \Upomega_{\rm F} n }, 
\end{eqnarray}
где је $\Upomega_{\rm F} = \dfrac{2\uppi}{N_{\rm F}}$ дискретна кружна учестаност основног хармоника. Коефицијенти 
Фуријеовог реда могу се израчунати према формули 
\begin{equation}
    X[k] = \dfrac{1}{N_{\rm F}} \sum_{n = \langle N_{\rm F} \rangle} x[n] \ee^{-\jj k \Upomega n}, 
    \qquad\qquad\qquad
    k = \langle N_{\rm F} \rangle,
\end{equation}
а у општем случају има их $N_{\rm F}$. 

Коефицијенти сигнала $x[n]$ (и $y[n]$), чији је развој у Фуријеов ред
$X[k] = \FS{x[n]}$ (и $Y[k] = \FS{y[n]}$), на периоду $N_{\rm F}$ задовољавају својства:
\begin{itemize}\itemsep0pt
    \item \emph{Линеарност:} $\FS{a x[n] + b y[n]} = a X[k] + b Y[k], \qquad a,b = \const$ 
    \item \emph{Померај у времену: }
    $ \FS{x[n-M]} = X[k] \ee^{-\jj k \upomega_{\rm 0} M}$, $M = \const$;
    \item \emph{Модулација у времену:}
    $ \FS{\ee^{\jj k_0 \Upomega_{\rm F} n } x[n] } = 
    X[k - k_0]$;
    \item \emph{Инверзија временске осе:}
    $\FS{x[-n]} = X[-k] $;
    \item \emph{Комплексна конјукција:}
    $\FS{x^{\ast}[n] } = X^{\ast} [-k]$;
    \item \emph{Кружна конволуција:}
    $\FS{x[n] \circledast y[n] } = N_{\rm F} X[k] Y[k]$
    \item \emph{Производ:}
    $\FS{x[n] \cdot y[n]} = X[k] \circledast Y[k]$
\end{itemize}



\section*{Фуријеова трансформација континуалног сигнала} \label{d:CTFT}

За континуалан сигнал $x(t)$, у општем случају апериодичан, може се одредити Фуријеова трансформација по дефиницији
\begin{eqnarray}
    X(\jj\upomega) = \FT{x(t)} = \int_{\mathclap{t = -\infty}}^{\infty} x(t) \ee^{-\jj \upomega t} \de t
\end{eqnarray}

Фуријеова трансформација строго је дефинисана за сигнале који су апсолутно интеграбилни, односно, који задовољавају услов 
$\int_{-\infty}^{\infty} |x(t)| \de t < \infty$. Ипак, увођењем Делта импулса, као што је показано у задатку \refz{jexp}, ово 
домен трансформације се може проширити тако да обухвата и све периодичне сигнале за које се може дефинисати 
Фуријеов ред (практично све периодичне сигнале у инжењерству). 

Дакле, за периодичан сигнал $x(t)$ који се може развити у Фуријеов ред $X[k] = \FS{x(t)}$, може се одредити онда Фуријеова трансформација 
као\footnote{Члан „$2\uppi$“ заправо потиче из чињенице да у инжењерској пракси користимо Фуријеову трансформацију чији је 
аргумент \textit{кружна} учестаност, па тај члан проистиче из $\upomega = 2\uppi f$, односно, у интегралу 
$\de \upomega = 2\uppi \, \de f$. }
\begin{equation}
    X(\jj\upomega) = 2\uppi \sum_{k = -\infty}^{\infty} X[k] \updelta( \upomega - k \upomega_{\rm 0} )
    \qquad
    \text{(\textit{Веза Фуријеовог реда и трансформације})}
\end{equation}

\noindent 
Парсевалова теорема даје везу између \myul{енергије} континуалног сигнала и његове
Фуријеове трансформације:
\begin{equation}
W_x = \int_{-\infty}^{\infty} |x^2(t)|\,\de t = \dfrac{1}{2\uppi}\int_{-\infty}^{\infty} |X(\jj\upomega)|^2 \de \upomega. \label{s:ctft:parseval}
\end{equation}

%Квадрат модула Фуријеове трансформације представља спектралну густину снаге сигнала 
%$S_x = |\FT{x(t)}|^2$ (Парсевалова теорема). 

За континуални сигнал $x(t)$ (и $y(t)$), за који постоји Фуријеова трансформација
$X(\jj\upomega) = \FT{x(t)}$ (и $Y(\jj\upomega) = \FT{y(t)}$), важе следећа својства
\begin{itemize}
    \item \emph{Линеарност:} $\FT{ax(t) + by(t)} = a X(\jj\upomega) + b Y(\jj\upomega)$, \qquad $a,b = \const$;
    \item \emph{Померај у времену:} $\FT{x(t-\uptau)} = X(\jj\upomega) \ee^{-\jj\upomega \uptau}$; \qquad $\uptau = \const$
    \item \emph{Модулација у времену:} $\FT{x(t) \ee^{\jj\upalpha t}} = X(\jj(\upomega - \upalpha))$; \qquad $\upalpha = \const$
    \item \emph{Комплексна конјукција:} $\FT{x^{\ast}(t) } = X^{\ast}(-\jj\upomega)$; 
    \item \emph{Инверзија временске осе:} $\FT{x(-t) } = X(-\jj\upomega)$; 
    \item \emph{Скалирање временске осе:} $\FT{x(at)} = \dfrac{1}{|a|} X\left( \dfrac{\upomega}{a} \right)$;
    \item \emph{Конволуција:} $\FT{x(t) \ast y(t)} = X(\jj\upomega) \cdot Y(\jj\upomega)$; 
    \item \emph{Множење:} $\FT{x(t) \cdot y(t)} = \dfrac{1}{2\uppi} X(\jj\upomega) \ast Y(\jj\upomega)$; 
    \item \emph{Производ:} $\FT{x(t) \cdot y(t)} = \dfrac{1}{2\uppi} X(\jj\upomega) \ast Y(\jj\upomega)$;
    \item \emph{Дејство на извод:} $ \FT{ \dfrac{ \de x(t)}{\de t}} = \jj\upomega X(\jj\upomega)$;
    \item \emph{Дејство на интеграл:} $ \FT{ \int_{-\infty}^{t} x(t) \de t} =  
    \left( 
        \dfrac{1}{\jj\upomega} + \updelta(\upomega)
    \right) X(\jj\upomega)$;
    \item \emph{Диференцирање по учестаности:} 
    $
    \FT{ t x(t) } = \jj \dfrac{\de X(\jj\upomega)}{\de \upomega}
    $;
    \item \emph{Дејство на парни део сигнала:}
    $
    \FT{ \operatorname{Ev}x(t)} = \Re{X[k]}
    $
    \item \emph{Дејство на непарни део сигнала:}
    $
    \FT{ \operatorname{Od}x(t)} = -\Im{X[k]}
    $
    
\end{itemize} 


\section*{Унилатерална Лапласова трансформација}

Лапласова трансформација може се сматрати уопштењем Фуријеове трансформације на шири скуп сигнала. 
За континуални сигнал $x(t)$ дефинише се унилатерална Лапласова трансформација
\begin{equation}
    X(s) = \LT{x(t)} = \int_{t = 0}^\infty x(t) \ee^{-st} \de t, 
\end{equation}
где је $s = \upsigma + \jj\upomega$ генералисана учестаност. У општем случају, постоји одређена област конвергенције за 
параметар $s$, међутим, када се разматра унилатерална трансформација код каузалних сигнала и каузалних система област 
конвергенције не утиче на тачност поступка и својстава која се овде наводе, а практично се не мора разматрати. 
Приликом рачунања Лапласове трансформације по дефиницији у том случају, може се сматрати да се резултати наводе за $s$ 
које се налази унутар области конвергенције. 

Инверзна Лапласова трансформација у општем случају се изводи помоћу метода комплексне анализе, за наше потребе, најједноставније је 
применити технику растављања на парцијалне разломке.


За континуални сигнал $x(t)$ (и $y(t)$), за који постоји Лапласова трансформација
$X(s) = \LT{x(t)}$ (и $Y(s) = \LT{y(t)}$), важе следећа својства
\begin{itemize}\itemsep0pt
    \item \emph{Линеарност:} $\LT{ax(t) + by(t)} = a X(s) + b Y(s)$, \qquad $a,b = \const$;
    \item \emph{Скалирање временске осе:} $\LT{x(at)} = \dfrac{1}{a} F\left(\dfrac{s}{a}\right)$, \qquad $a \in \mathbb R$, $a > 0$;
    \item \emph{Померај у времену:} $\LT{x(t-\uptau} = X(s)\ee^{-s\uptau} \qquad \uptau = \const$; 
    \item \emph{Померај у фреквенцији:} $\LT{ \ee^{-\upsigma t} x(t) } = X(s - \upsigma)$; \\
    \item \emph{Утицај на извод:} $ \LT{ \dfrac{\de x(t)}{\de t} } = s X(s) - x(0^-) $;
    \item \emph{Утицај на интеграл:} $ \LT{ \int_0^t x(\uptau) \de \uptau } = \dfrac{1}{s} X(s)$;
    %\item \emph{Диференцирање у комплексном домену:} $\LT{t^n x(t)} = (-1)^n \dfrac{\de^n F}{\de s^2}$, \qquad $n\in\mathbb N_0$.
    %\item \emph{Интеграл у комплексном домену:} $\LT{ \dfrac{x(t)}{t} } = \int_{s}^{\infty F(z) \de z} $, \qquad $n\in\mathbb N_0$.
    \item \emph{Конволуција: } $\LT{x(t) \ast y(t)} = X(s) \cdot Y(s)$; 
    \item \emph{Множење:} $\LT{ x(t) \cdot y(t) }  = \dfrac{1}{\jj2\uppi} X(s) \ast Y(s)$; 
\end{itemize}

Слика периодичног сигнала $x(t)$, чији је основни период $T$, може се одредити као 
$\LT{x(t)} = \dfrac{1}{1 - \ee^{-sT}} \int_0^T x(t) \ee^{-s\uptau}$.

Помоћу наведеног правила о утицају на извод, могу се добити и формуле за више изводе, нпр.
\begin{eqnarray}
    && \LT{ \dfrac{\de^2 x(t)}{\de t^2} } = s^2 X(s) - s x(0^-) - x'(0^-) \\
    && \LT{ \dfrac{\de^3 x(t)}{\de t^3} } = s^3 X(s) - s^2 x(0^-) - s x'(0^-) - x''(0^-) \\
    && \LT{ \dfrac{\de^n x(t)}{\de t^n} } = s^n X(s) - s^{n-1} x(0^-) - s^{n-2} x'(0^-) - \cdots - x^{(n-1)} (0^-) 
\end{eqnarray}


\section*{Унилатерална $\mathcal{Z}$-трансформација}

$\mathcal{Z}$-трансформација је уопштење дискретне Фуријеове трансформације. За дискретан сигнал $x[n]$ може се одредити
његова $\mathcal{Z}$-трансформација помоћу израза 
\begin{equation}
        X(z) = \sum_{m = 0}^{\infty} x[n] z^{-n}.
\end{equation}
На сличан начин као и Лапласова трансформација, унилатерална $\mathcal{Z}$-трансформација односи се на каузалне сигнале 
који делују на каузалне системе, када се може занемарити област конвергенције.

Инверзна $\mathcal{Z}$-трансформација у општем случају се изводи помоћу метода комплексне анализе. 
Слично као и за Лапласову трансформацију, за наше потребе, најједноставније је 
применити технику растављања на парцијалне разломке.

За дискретни сигнал $x[n]$ (и $y[n]$), за који постоји $\mathcal{Z}$-трансформација
$X(z) = \ZT{x[n]}$ (и $Y(z) = \ZT{y[n]}$), важе следећа својства
\begin{itemize}
    \item \emph{Линеарност:} $\ZT{ax[n] + by[n]} = a X(z) + b Y(z)$, \qquad $a,b = \const$;
    \item \emph{Предикција у времену} : $\ZT{x[n+1]} = z X(z) - z x[0]$;
    \item \emph{Модулација у времену:} $\ZT{ a^n x[n] } = X \left(\dfrac{z}{a}\right)$, \qquad $a = \const$;
    \item \emph{Комплексна конјукција:} $\ZT{ x^{\ast}[n] } = X^{\ast}(z^{\ast})$;
    \item \emph{Извод по фреквенцији:} $\ZT{n x[n]} = -z \dfrac{\de X(z) }{\de z}$;
    \item \emph{Конволуција} $\ZT{x[n]  \ast y[n]} = X(z) \cdot Y(z)$; 
\end{itemize}