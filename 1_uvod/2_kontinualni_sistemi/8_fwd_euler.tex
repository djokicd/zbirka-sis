\begin{slikaDesno}{fig/fwr_euler.pdf}    
    {\color{red}*}\PID 
    У колу са слике познато jе $R = 1\unit{k\Omega}$ и 
    $C = 100\unit{nF}$. Посматра се континуални систем, чиjи jе jедини улаз струjа идеалног струjног генератора
    $i_{\rm g} = i_{\rm g}(t)$, а jедини излаз напон $v_{\rm I} = v_{\rm I} (t)$.
    (a) Одредити диференциjалну jедначину тог система, у облику 
    $P(\DD) v_{\rm I} = Q(\DD) i_{\rm g}$, где jе $\DD = \dfrac{\de}{\de t}$.
    (б) Испитати асимтотску стабилност датог система. 
\end{slikaDesno}
Тако одређен систем потребно jе симулирати на дигиталном рачунару,
зарад чега jе неопходно дати систем дискретизовати у времену. Дискретизациjа
се обавља заменом оператора диференцирања у времену скалираним оператором
диференце унапред 
$\dfrac{\de}{\de  t} \mapsto \dfrac{\Delta}{T}$, 
а дискретизовани систем онда апроксимира еквивалентна диференцна jедначина, 
$P\left( 
    \dfrac{\Delta}{T}
\right) \hat v_{\rm I}[n] 
=
Q\left( 
    \dfrac{\Delta}{T}
\right) \hat i_{\rm g}[n]$, 
по низовима $\hat v_{\rm I}[n] = v_{\rm I}(nT)$ и $\hat i_{\rm g}[n] = i_{\rm g}(nT)$, где jе $T > 0$ 
период дискретизациjе. У зависности од параметра $T$ (в)
испитати стабилност дискретизованог система у асимптотском смислу. 
\vspace*{2mm}

\textsc{\underline{Резултат:}}
(а) Диференциjална jедначина система jе $R i_{\rm g} = (\uptau \DD + 1) v_{\rm I}$, где jе 
$\uptau = 200\unit{\upmu s}$.
(б) Систем jе асимптотски стабилан.
(в) 
Дискретизовани систем jе
$
\left\{
\begin{matrix}
    \text{стабилан} \\
    \text{гранично стабилан} \\
    \text{нестабилан} \\
\end{matrix}
\right\}
$ 
за
$
\left\{
\begin{matrix}
    T < 200\unit{\upmu s} \\
    T = 200\unit{\upmu s} \\
    T > 200\unit{\upmu s}
\end{matrix}
\right\}
$.

