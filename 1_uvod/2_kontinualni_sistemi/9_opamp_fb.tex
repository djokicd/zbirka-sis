\mnDifficult
\begin{slikaDesno}{fig/OP_fb.pdf}
\PID У колу са слике познато је 
$R_1 = 1\unit{k\Omega}$, 
$R_2 = R_3 = R_4 = 2\unit{k\Omega}$ и
$C = 100\unit{\upmu F}$ а операциони појачавачи
и прекидач $\Uppi$ су идеални. Прекидач 
$\Uppi$ је \myul{отворен} а у колу \myul{нема} 
почетне енергије. Једини улаз посматраног система је улазни напон, $v_{\rm U}$ 
а једини излаз је напон $v_{\rm I}$.

\begin{enumerate}[label=(\alph*)]

    \item Одредити диференцијалну једначину тог система, а затим решавањем у временском домену 
    одредити одзив датог система на одскочну побуду \linebreak
    $v_{\rm U}^{\text{(а)}} = 1\unit{V} \, {\rm u}(t)$. 
    
    \item Полазећи од резултата претходне тачке, применом својстава \textit{LTI}
    система, одредити одзив на импулсну побуду 
    $v_{\rm U}^{\text{(б)}} = 100\unit{mWb}\,\updelta(t)$. 
    
    \item Методом по избору, испитати асимптотску и \textit{BIBO} стабилност датог система.
    
    \item Прекидач $\Uppi$ се \myul{затвори}. Израчунати отпорност $R_{\rm X}$ 
    тако да добијени систем у целини
    буде \myul{маргинално стабилан}. За тако добијену вредност одредити 
    \myul{резонантан одзив} на побуду 
    облика $v_{\rm U}^{\text{(г)}} = 1\unit{V} {\rm e}^{-at} \,{\rm u}(t)$ за одговарајућу 
    вредност параметра $a$.

\end{enumerate} 

\end{slikaDesno}


\textsc{\underline{Резултат:}} (а) Диференцијална једначина система је 
$\uptau \dfrac{\de v_{\rm I}}{\de  t} + v_{\rm I} = -a v_{\rm U}$, 
где су $\uptau = R_2 C = 0,2\unit{s}$ и $a = -\dfrac{R_2}{R_1} = -2$. Одзив на одскочну побуду 
је $v_{\rm I}^{\text{(a)}} = -2\unit{V} \bigl( 1 - {\rm e}^{-t/\uptau}  \bigr) \, {\rm u}(t)$.
(б) Пошто важи веза 
$v_{\rm U}^{\text{(б)}} = 100\unit{ms}\, \dfrac{\de v_{\rm U}^{\text{(a)}}}{\de t}$
применом својства \textit{LTI} система има се да је 
$v_{\rm I}^{\text{(б)}} = 100\unit{ms} \, \dfrac{\de v_{\rm I}^{\text{(a)}}}{\de t}$ односно
$v_{\rm I}^{\text{(б)}} = -1\unit{V} \, {\rm e}^{-t/\uptau} \, {\rm u}(t)$. 
(в) Систем јесте и асимптотски и \textit{BIBO} стабилан. (г) Треба да буде 
$R_{\rm X} = R_2 = 2\unit{k\Omega}$ у том случају резонантни одзив наступа за $a = 0$ а дат је
изразом $v_{\rm I}^{\text{(г)}} = -\dfrac{1\unit V}{R_1 C} t \, {\rm u}(t) = 
- 10\unit{\dfrac{V}{s}} \, t \, {\rm u}(t)$.

