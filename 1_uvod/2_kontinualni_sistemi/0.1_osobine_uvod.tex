\PID
За следеће системе испитати да ли су стабилни
у \textit{BIBO} смислу,
линеарни, временски инваријантни, 
са меморијом и каузални: 
\begin{multicols}{3}
\begin{enumerate}[label=(\alph*)]
\item $y(t) = \sum_{k = 0}^{\infty} x(t-kT)$, 
\item $y(t) = \sqrt{2} x(t)$,
\item $y(t) = t \left( x(t-1) \right)^2$,
\item $y(t) = \int_{\uptau = -\infty}^{t}
\hspace{-0.99em}
x(\uptau) \sin(\uptau) \,\de\tau$,
\item $y(t) = \dfrac{\de x(t)}{\de t}
$,
\item $y(t) =t {\rm e}^{x(t) - t} \, \uu(t)$,
\end{enumerate}
\end{multicols}\noindent
где је $y(t) = {\rm O}\{x(t)\}$ одзив посматраног система.
\\[2mm]

\RESENJE 
Поновимо особине система и њихове дефиниције
\begin{itemize}
    \item Систем је \textit{линеаран} уколико је \textit{адитиван} и \textit{хомоген}. Систем је адитиван уколико је 
    одзив на збир два улаза једнак збиру одзива на сваки од улаза појединачно, 
    ${{\rm O}\{x_1(t) + x_2(t)\} = {\rm O}\{x_1(t)\} + {\rm O}\{x_2(t)\}}$.
    Систем је хомоген ако је
    одзив на умножак улаза и константе једнак производу одзива и константе, односно
    ${\rm O}\{kx(t)\} = k {\rm O}\{x(t)\}$, $k = \rm const$. У општем случају, систем је линеаран уколико важи принцип 
    суперпозиције 
    \begin{equation}
        {{\rm O}\{ax_1(t) + bx_2(t)\} = a{\rm O}\{x_1(t)\} + b{\rm O}\{x_2(t)\}}, \qquad \qquad \forall a,b = \rm const.
    \end{equation}
    \item Систем је \textit{стабилан у \textit{BIBO} смислу} (енг. \textit{Bounded Input Bounded Output}) 
    уколико сваки ограничен улаз доводи до ограниченог излаза. Односно, уколико је побуда ограничена са $B_{\rm x} \geq 0$,
    тако да је $|x(t)| < B_{\rm x}, \forall t$, онда је систем стабилан у BIBO смислу ако је и одзив ограничен, 
    односно постоји $B_{\rm y}$ такво да је $|y(t)| < B_{\rm y}, \forall t$. Предикатском логиком ово се може записати 
    и као 
    \begin{equation}
        (\exists B_{\rm x})(\forall t)( |x(t)| < B_x ) \Rightarrow
        (\exists B_{\rm y})(\forall t)( |{\rm O}\{x(t)\}| < B_y ).
    \end{equation}
    Важно је нагласити да је дати исказ импликација,
    односно код \textit{BIBO} стабилних система може се десити да неограничена побуда доводи до неограниченог одзива 
    или да неограничена побуда доводи до ограниченог одзива. Такође, често је могуће показати да систем није 
    стабилан одређивањем контрапримера, односно, испитивањем исказа контрапозиције.

    \item Систем је \textit{стационаран} (тј. временски инваријантан/непроменљив) уколико транслација побуде у времену 
    доводи до исте транслације одзива у времену: 
    \begin{equation}
        {y(t) = {\rm O}\{x(t)\} \Rightarrow y(t - \uptau) = {\rm O}\{x(t - \uptau)\}}.
    \end{equation}
    Стационарност система се често може испитати уверавањем да ли систем на неки начин препознаје апсолутно време - односно 
    да ли његов одзив експлицитно зависи од времена.
    \item За систем се каже да \textit{нема меморију} уколико тренутна вредност одзива зависи само од тренутне вредности улаза, односно, нема 
    особину \textit{меморисања} претходних вредности улаза. Иначе, систем је са меморијом. 

    \item За систем се каже да је  \textit{каузалан} уколико тренутна вредност одзива не зависи од будућих вредности улаза. 
    Сваки систем који постоји у природи, или који може да се реализује, је каузалан.
\end{itemize}

\begin{enumerate}[label=(\alph*)]
    \item
    \begin{itemize}
    \item Линеарност испитујемо испитивањем суперпозиције
    ${\rm O}\{ a x_1(t) + b x_2(t) \} = \sum_{k = 0}^{\infty} ax_1(t-kT) + bx_2(t-kT) = 
    a \sum_{k = 0}^{\infty} x_1(t-kT) + b  \sum_{k = 0}^{\infty} x_2(t-kT) = a {\rm O}\{x_1(t)\} + b {\rm O}\{x_2(t)\}$,
    што значи да систем јесте линеаран.  \hfill $\checkmark$

    \item Претпоставимо да је ограничена побуда $x(t) = 1$, онда је одзив $y(t) = \sum_{k = 0}^{\infty} 1 \to \infty$, 
    па је одзив неограничен па систем није стабилан у \textit{BIBO} смислу. \hfill $\times$

    \item Пошто чланови суме за $k > 0$ за вредност одзива у тренутку $t$ користе вредност побуде у тренуцима $t - kT < t$
    (односно у прошлости), систем је са меморијом. \hfill $\checkmark$
    
    \item Пошто су сви чланови суме такви да је $t - kT \leq t$, систем је каузалан. Обратити пажњу да је ово 
    последица одабира доње границе сумирања.  \hfill $\checkmark$
    \end{itemize}

    \item
    \begin{itemize}
        \item Испитивањем суперпозиције, ${\rm O}\{ a x_1(t) + b x_2(t) \} = \sqrt2 (ax_1(t) + bx_2(t)) = 
        a \sqrt 2 x_1(t) + b \sqrt 2 x_2(t)
        = a {\rm O}\{x_1(t)\} + b {\rm O}\{x_2(t)\}$, закључује се да је систем линеаран. \hfill $\checkmark$

        \item Пошто је систем множење константом, ограниченост побуде имплицира ограниченост одзива,
        $B_{y} = \sqrt 2 B_{\rm x}$, па је систем стабилан у \textit{BIBO} смислу. \hfill $\checkmark$

        \item Пошто транслација побуде доводи до транслације одзива, систем је стационаран. \hfill $\checkmark$

        \item Пошто је одзив у тренутку $t$ пропорционалан вредности побуде у тренутку $t$, систем нема меморију. \hfill $\times$
        
        \item Пошто систем за рачунање тренутне вредности одзива „види“ само тренутну вредност побуде, он је каузалан. \hfill $\checkmark$
    \end{itemize} 

    \item 
    \begin{itemize}
        \item Линеарност система не важи, будући да постоји квадрирање сигнала, што је нити адитивна нити хомогена операција.
        \hfill $\times$

        \item Размотримо сигнал $x(t) = 1$, тада је $y(t) = t$, па је $y(\infty)\to\infty$ одзив неограничен, па 
        систем није стабилан у \textit{BIBO} смислу. \hfill $\times$

        \item Одзив у тренутку $t$ зависи од тренутка  $t-1$ тако да је дати систем са меморијом. \hfill $\checkmark$
        
        \item Пошто одзив у тренутку $t$ не зависи од будућности, систем је каузалан. \hfill $\checkmark$
    \end{itemize}
\end{enumerate}

Примери (г), (д) и (ђ) се читаоцу остављају за вежбу. Коначни резултат приказан је табеларно. 

\begin{center}
\newcommand{\cm}{\checkmark}
\begin{tabular}{r|cccccc}
    & (a) & (б) & (в) & (г) & (д) & (ђ) \\
     \hline
    Линеаран               
    & \cm & \cm &    & \cm & \cm &    \\ 
    Временски инваријантан 
    & \cm    & \cm &    &     &  \cm   &    \\ 
    Са меморијом 
    & \cm &     &\cm & \cm &   \cm  & \\
    Стабилан 
    &     & \cm  &  & &  & \cm \\
    Каузалан 
    & \cm & \cm & \cm & \cm & & \cm \\
    \end{tabular}
\end{center}    
