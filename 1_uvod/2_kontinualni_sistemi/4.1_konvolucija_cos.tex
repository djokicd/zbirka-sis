\mnDifficult\PID \label{z:konv_sin}
Посматра се систем првог реда чији је импулсни одзив дат изразом 
$h(t) = \ee^{-at} \uu(t)$, где је $a>0$ позната константа. 
Одредити принудни одзив овог система на простопериодичну побуду 
облика (а) $x(t) = \cos(\upomega_0 t)\uu(t)$, односно (б) 
$x(t) = \sin(\upomega_0 t) \uu(t)$.\\[2mm]

\textsc{\myul{Решење:}}
(а) Слична идеја се може искористити као у задатку \ref{z:sin_cos_pobuda}, потраживањем одзива на комплексну 
експоненцијалну побуду $\underline{x}(t) = \cos(\upomega_0 t) + \jj\sin(\upomega_0 t) = \ee^{\jj\upomega_0 t}$,
помоћу конволуције
$
    \underline{y}(t) = \underline{x}(t) \ast h(t) = \ee^{\jj \upomega_0 t} \ast \ee^{-at}  
$
Добијена конволуција експоненцијалних сигнала одређују се као у задатку \ref{z:exp_konv}б, одакле се има 
$\ee^{\jj \upomega_0 t} \ast  \ee^{-at} = 
\dfrac{\ee^{\jj \upomega_0 t} -  \ee^{-at}}{\jj\upomega_0 + a}$, па се онда појединачни одзиви простопериодичне 
побуде одређују растављањем овог израза до реалног и имагинарног дела. Израз у имениоцу запишимо у поларном
облику\footnote{Користи се растављање комплексног броја
у поларни облик $a + \jj b = \sqrt{a^2 + b^2}\,\exp(\jj \arctg{{b/a}})$, за $a > 0$.} као 
$\jj\upomega_0 + a = \sqrt{ \upomega_0^2 + a^2 } \exp\left( \jj \uppsi \right)$, где је 
$\uppsi = \arctg \dfrac{\upomega_0}{a}$. Заменом и даљим сређивањем добија се коначно

\begin{align}\hspace*{-5em}
    \underline{y}(t) =&
    \dfrac{\ee^{\jj \upomega_0 t} -  \ee^{-at}}{\jj\upomega_0 + a}
    =
    \dfrac{\ee^{\jj \upomega_0 t} -  \ee^{-at}}{\sqrt{ \upomega_0^2 + a^2 } \ee^{-\jj \uppsi} }
    = \dfrac{
    \cos\left( \upomega_0 t +  \uppsi \right)
    + 
    \jj
    \sin\left( \upomega_0 t +  \uppsi \right)
    - \ee^{-at} \exp\left( -\jj \uppsi \right)
    }{ \sqrt{ \upomega_0^2 + a^2 } } \\
    =&
    \underbrace{
    \dfrac{
    \cos\left( \upomega_0 t +  \uppsi \right)
    - \ee^{-at} \cos(\uppsi)
    }{ \sqrt{ \upomega_0^2 + a^2 } }
    }_{\text{Реални део, Одзив на $\cos$}}
    + \jj
    \underbrace{
    \dfrac{
    \sin\left( \upomega_0 t +  \uppsi \right)
    + \ee^{-at} \sin(\uppsi)
    }{ \sqrt{ \upomega_0^2 + a^2 } }
    }_{\text{Имагинарни део, Одзив на $\sin$}}, \qquad\uppsi = \arctg \dfrac{\upomega_0}{a}.
\end{align}

Уколико за тиме постоји потреба, члан уз $\ee^{-at}$ може се даље расписати применом тригонометријских идентитета
$\cos(\arctg x) = \dfrac{1}{\sqrt{1 + x^2}}$ и $\sin(\arctg x) = \dfrac{x}{\sqrt{1 + x^2}}$. Преостало сређивање 
израза препушта се читаоцу.

