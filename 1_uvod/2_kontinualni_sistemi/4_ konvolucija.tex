\subsubsection{\textit{Конволуција континуалних сигнала}}
\PID
Дати су сигнали 
$x = x(t)$ који се доводе на улаз система чији је импулсни одзив 
дат изразом $h = h(t)$. Одредити принудни одзив у случајевима:
%\begin{multicols}{2}
\begin{enumerate}
\item[(а)] $x(t) = {\rm u}(t)$, $h(t) = {\updelta}(t - T)$, 
$T \in \mathbb R$
\item[(б)] $x(t) = {\rm e}^{-at} \,{\rm u}(t)$, 
$h(t) = {\rm e}^{-bt}\,{\rm u}(t)$, где су 
$a,b\in\mathbb R^+_0$  и 
$a \neq b$
\item[(в)] $x(t) = {t}^{k} \,{\rm u}(t)$,
$h(t) = {\rm u}(t)$, где је $k \neq -1$.
\item[(г)] $x(t) = \uu(t) - \uu(t-T)$,
$h(t) = x(t)$, где је $T \in \mathbb R^+$.
\end{enumerate}

\textsc{\myul{Решење:}}
Принудни одзив система (одзив система на побуду) одређен је конволуцијом 
побуде $x(t)$ и импулсног одзива $h(t)$, што је одређено конволуционим интегралом
\begin{equation}
    y(t) = x(t) \ast h(t) = \int_{-\infty}^{\infty} x(\uptau) h(t - \uptau )\, \de \uptau. \label{eq:\ID.1}
\end{equation}


(а) Заменом датих сигнала у \eqref{eq:\ID.1} добија се 
$y(t) = 
\int_{-\infty}^{\infty} \uu(\uptau) \underbrace{\updelta(t - \uptau - T)}_{\uptau = t - T}  \, \de \uptau
$. Овај интеграл је исказ својства еквиваленције Дираковог импуса, па је 
$y(t) = \uu(t - T)$. Такође, приметимо да је систем чији је импулсни одзив
$\updelta(t - T)$ систем за кашњење за време $T$, што је конзистентно са 
добијеним резултатом.


(б)  Заменом датих сигнала у \eqref{eq:\ID.1} добија се 
$y(t) = 
\int_{-\infty}^{\infty} \ee^{- a \uptau} \uu(\uptau) \, \ee^{-b(t - \uptau)} \uu(t-\uptau)  \, \de \uptau
$. У овом интегралу, Хевисајдова одскочна функција намеће границе интегације, будући да је 
$\uu(\uptau)\cdot\uu(t-\uptau) = \begin{cases}
    1, & 0 < \uptau < t \\
    0, & \text{иначе}.
\end{cases}$. Одатле се има, да је подинтегрална величина једнака нули за $t<0$, па је 
$y(t < 0) = 0$, док је за $t>0$ одзив једнак
$y(t) = 
\int_{0}^{t} \ee^{- a \uptau}\, \ee^{-b(t - \uptau)}   \, \de \uptau
= 
\int_{0}^{t} \ee^{-bt} \ee^{(b - a) \uptau}   \, \de \uptau
= \dfrac{\ee^{-bt} - \ee^{-at}}{b - a}
$. Узимајући у обзир оба резултата (и за $t<0$ и за $t>0$),  коначно се може записати
$y(t) = \dfrac{\ee^{-bt} - \ee^{-at}}{b - a} \, \uu(t)$.

Читаоцу се препоручује да тачку понови у случају када је $a=b$. 

(в) Слично као у претходној тачки, и у овом случају Хевисајдова одскочна функција намеће границе интеграције, па се има 
$y(t) = 
\int_{-\infty}^{\infty} \uptau^k \uu(\uptau) \, \uu(t-\uptau)  \, \de \uptau = 
\int_{0}^{t} \uptau^k \, \de \uptau = \dfrac{t^{k+1}}{k+1} \, \uu(t)$.
Такође, приметимо да, пошто је $\int_{-\infty}^t \updelta(\uptau) \, \de \uptau = \uu(t)$, то онда систем 
чији је импулсни одзив $\uu(t)$ мора представљати интегратор. 


(г) Тачка се оставља читаоцу за вежбу, коначан резултат је 
$y(t) = T \operatorname{tri}\left( \dfrac tT - 1  \right)$.



