\PID\mnImportant
Нека је дат континуалан систем диференцијалном једначином облика 
${\dfrac{\de y}{\de t} - ay = x}$, где су $x=x(t)$ и $y=y(t)$ побуда и одзив тога система редом, а $a$ је позната
реална константа. Одредити одзив на експоненцијалну побуду облика 
${\rm e}^{bt} \, \uu(t)$, (а) ако је $b \neq a$ и (б) $b = a$. \\[5mm]

\RESENJE  За $t < 0$ је одзив на побуду једнак нули, док се 
за $t > 0$ решава диференцијална једначина 
${y'(t) - ay(t)  = {\rm e}^{bt}}$. У општем случају, решење се састоји из 
\textit{хомогеног} и \textit{партикуларног} дела\footnote{Овом приликом треба да нагласити 
да, код стабилних система, хомогени део решења одговара \textit{сопственом} одзиву система на почетне услове
док партикуларни део представља \textit{устаљени} одзив система за задату побуду. Ово тумачење онда оправдава
да партикуларни део треба тражити сигналима учестаности побуде (јер линеарни системи не могу да мењају учестаност 
улазног сигнала, нпр. кола простопериодичних струја), а да резонтатни сигнали доводе 
до алгебарске дегенерациије решења -- множење са $t$.}. Хомогени део се налази одређивањем корена 
карактеристичног полинома $P(\uplambda) = \uplambda - a$ па је $\lambda_0 = a$. Постоји само 
једна карактеристична функција па је облик хомогеног дела одзива 
$y_{\rm h}(t) = A \ee^{at}$, за произвољну вредност константе $A$. Партикуларни део експоненцијалне побуде се тражи у 
експоненцијалном облику\footnote{То је природно за очекивати, будући да је експоненцијална функција
једина сразмерна своме изводу.}, па је $y_{\rm p} = B\ee^{bt}$, заменом у полазну једначину добија се:
\begin{equation}
    Bb\cancel{\ee^{bt}} - aB\cancel{\ee^{bt}} = \cancel{\ee^{bt}} \Rightarrow
    B = \dfrac{1}{b - a}.
\end{equation}
Комплетан облик одзива је онда облика $y(t) = A\ee^{at} + \dfrac{1}{b-a} \ee^{bt}$. Будући да је
побуда ограничена то је одзив непрекидан па је $y(0^+) = 0$ одакле се налази константа $A$, 
па је $A = \dfrac{1}{a - b}$, коначно се добија да је одзив на тражену побуду:
\begin{equation}
    y(t) = \dfrac{\ee^{bt} - \ee^{at}}{b - a}.
    \label{eq:\ID.1}
\end{equation}

(б) 
Проблем дељења нулом када је $b=a$ се може решити тражењем граничне вредности. 
Узмимо да је $b = a + \upepsilon$ 
и заменимо у резултат \ref{eq:\ID.1}, одатле се сређивањем даље има
\begin{equation}
    y(t) = \dfrac{\ee^{(a + \upepsilon)t} - \ee^{at}}{\cancel{a} + \upepsilon - \cancel{a}} = 
    \ee^{at} \dfrac{\ee^{\upepsilon t} - 1}{\upepsilon}.
\end{equation}
Будући да је $\DS \lim_{\upepsilon\to 0} \dfrac{\ee^{\upepsilon t} - 1}{\upepsilon} = t$ одзив
у траженом случају ће бити коначно 
$\DS y(t) = t \ee^{at}$.

У општем случају, генерализацијом поступка који је дат у овом задатку, показује 
се да је партикуларни део решења диференцијалне једначине 
$P(\DD) y(t) = x(t)$, за експоненцијалну побуду $x(t) = \ee^{at} \uu(t)$, дат 
у облику $y_{\rm p}(t) = \dfrac{\ee^{at}}{P(a)}$, ако је $P(a) \neq 0$.

Уколико је $a$ једноструки корен полинома $P$, онда је партикуларни део у облику
$y_{\rm p} = \dfrac{t \ee^{at}}{P'(a)}$, а кажемо да побуда „погађа“ резонансу система 
првог реда. 

Ако је $a$ вишеструки корен полинома $P$, вишеструкости $s$, 
онда је партикуларни део у облику $ y_{\rm p} = \dfrac{t^s \ee^{at}}{P^{(s)}(a)}$. Ове формуле 
и дискретне системе дате су у додатку \ref{dod:exp_response}.
