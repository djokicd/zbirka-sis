\PID
Нека је систем описан диференцијалном једначином у облику
$\DS
P({\rm D})\,y(t) = x(t)$, где су $x(t)$ и $y(t)$ побуда и одзив тог система редом, а 
$P({\rm D})$ је оператор дат полиномом са реалним коефицијентима по оператору диференцирања 
${\rm D} = \dfrac{\de }{\de t}$. Полазећи од формуле одзива за експоненцијалну побуду, 
облика $y_{\rm p} = \dfrac{ \ee^{at} }{P(a)}$, одредити 
партикуларни део одзива на нерезонантну побуду када је она простопериодична, облика 
(а) $x(t) = \cos(\upomega_0 t)$ и 
(б) $x(t) = \sin(\upomega_0 t)$.

\underline{\sc Решење:} Приметимо да је 
$\ee^{\jj\upomega_0 t} = \cos(\upomega_0 t) + \jj \sin(\upomega_0 t)$.
Мотивисани том примедбом, 
размотримо побуду комплексним сигналом облика: 
$\underline{x}(t) = x_{\rm r}(t) + \jj x_{\rm i}(t)$. Заменом је одзив на такву 
побуду у облику: $\underline{y}(t) = P(\DD) \underline{x} = P(\DD) x_{\rm r}(t) + \jj P(\DD) x_{\rm i}(t)$, односно, може се 
тврдити да је $\Re{\underline{y}(t)} = \Re{ P(\DD) \underline{x}}$, односно, 
$\Im{ \underline{y}(t)} = \Im{ P(\DD) \underline{x}}$. Другим речима, реални део побуде побуђује само реални део озива, док 
имагинарни део побуде побуђује само имагинарни део одзива. 

На овом резултату можемо да темељимо поступак одређивања одзива на побуде облика $\cos(\upomega_0 t)$ и $\sin(\upomega_0 t)$, полазећи
од одзива на експоненцијалну побуду  $\ee^{\underline s t}$, $\underline s = \jj\upomega$, користећи резултат да је 
$\underline{y}_{\rm p}(t) =  \dfrac{\ee^{\underline s t}}{P(\underline s)} =  \dfrac{\ee^{\jj\upomega_0 t}}{P(\jj\upomega_0)}$, 
одакле се налазе партикуларни делови одзива за побуде облика $\cos(\upomega_0 t)$ и $\sin(\upomega_0 t)$ као: 
\begin{align}
    &y_{\rm p}^{(\cos)}(t) 
    = \Re{\dfrac{\ee^{\jj\upomega_0 t}}{P(\jj\upomega_0)}} 
    = \Re{\dfrac{\ee^{\jj\upomega_0 t}}{|P(\jj\upomega_0)| \ee^{ \jj \arg P(\jj\upomega_0) } }} 
    = \dfrac{\cos\bigl(\upomega_0 t - \arg P(\jj\upomega_0)\bigr)}{|P(\jj\upomega_0)|}; \text{и} \label{eq:cos_pobuda} \\[2mm]
    &y_{\rm p}^{(\sin)}(t) 
    = \Im{\dfrac{\ee^{\jj\upomega_0 t}}{P(\jj\upomega_0)}} 
    = \Im{\dfrac{\ee^{\jj\upomega_0 t}}{|P(\jj\upomega_0)| \ee^{ \jj \arg P(\jj\upomega_0) } }} 
    = \dfrac{\sin\bigl(\upomega_0 t - \arg P(\jj\upomega_0)\bigr)}{|P(\jj\upomega_0)|}, \label{eq:sin_pobuda} 
\end{align}
којом приликом је расписан карактеристични полином у комплексном поларном облику, 
$P(\jj\upomega_0) = |P(\jj\upomega_0)| \ee^{ \jj \arg P(\jj\upomega_0) }$, а коришћено је и 
$\ee^{\jj\upomega_0 t} / \ee^{ \jj \arg P(\jj\upomega_0)} = \ee^{\jj\bigl(\upomega_0 t - \arg P(\jj\upomega_0)\bigr)}$. 