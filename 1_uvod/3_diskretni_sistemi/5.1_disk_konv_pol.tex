\mnAdvanced \PID
Дати су дискретни низови $a[k]$ за $0 \leq k \leq N$ и $b[k]$ за $0 \leq k \leq M$. Нека су дефинисани полиноми 
помоћу тих низова као $P(x) = \sum_{k = 0}^{N} a[k] x^k$ и $Q(x) = \sum_{k = 0}^{N} b[k] x^k$. Одредити 
коефицијенте полинома $R(x) = P(x) \cdot Q(x)$.

\RESENJE

\noindent 

У нади да мотивишемо општи поступак, спроведимо процедуру множења два полинома другог степена: 
\begin{eqnarray}
    (a_0 + a_1 x + a_2 x^2) (b_0 + b_1 x + b_2 x^2) &=& \hspace*{-5mm}
    \begin{matrix}
        & & a_0b_0\, x^0  &+& a_0b_1\, x^1 &+& a_0b_2\, x^2 \\
        &+& a_1b_0\, x^1  &+& a_1b_1\, x^2 &+& \boxed{a_1b_2}\, x^3 \\
        &+& a_2b_0\, x^2  &+& \boxed{a_2b_1}\, x^3 &+& a_2b_2\, x^4 \\
    \end{matrix} 
\end{eqnarray}
Приметимо да се коефицијенти испред $x^n$, добијају сабирањем свих производа коефицијената чији се индекси сабирају до $n$ 
(уоквирени случај $n=3$)
Са тим у виду, поступак се може уопштити.

Изразимо полином преко сличног низа $R(x) = \sum_{k=0}^{L} c[k] x^k$, онда се може писати
\begin{equation}
    R(x) = \sum_{k = 0}^{N} a[k] x^k \cdot  \sum_{m = 0}^{N} b[m] x^m = \sum_{k = 0}^{N}  \sum_{m = 0}^{N} a[k]b[m] x^{k+m}
\end{equation}
Приметимо онда, да се се коефицијент испред $x^n$ добија у свим сабирцима где је $k+m=n$. Односно, то се може записати као 
\begin{eqnarray}
    R(x) = \sum_{n = 0}^{N+M} \underbrace{\left( \sum_{k + m = n} a[k]b[m] \right)}_{c[n]} x^n.
\end{eqnarray}
Облик коефицијента можемо да препознамо као интерпретацију \textit{дискретне конволуције}, на начин како је описано у задатку 
\ref{zad:kov_tablica}, односно, коначно можемо изразити 
\begin{equation}
    c[k] = a[k] \ast b[k].
\end{equation}