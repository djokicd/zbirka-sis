\begin{slikaDesno}{fig/sc.pdf}
    {\color{red}*}\PID
    У колу са слике познато jе $C = C_{\rm s} = 100\unit{pF}$. 
    У почетном тренутку оба кондензатора су неоптерећена.
Прекидачи су идеални и затвараjу се наизменично,
краткотраjно, и то прво прекидач $\Uppi_1$ па прекидач
$\Uppi_2$. Напон генератора $v_{\rm U}$ не мења вредност осим када
jе прекидач $\Uppi_2$ затворен.
\end{slikaDesno}
\begin{enumerate}[label=(\alph*)]
    \item Одредити диференцну jедначину система чиjи jе
    jедини улаз напон побудног генератора $v_U[k]$ а jедини излаз напон 
    $v_{\rm I}[k]$ одређени након $k > 0$ затварања прекидача $\Uppi_1$.
    \item Одредити одзив добиjеног система на побуду $v_{\rm U}[k] = V_0 \uu[k]$ и 
    испитати његову асимптотску стабилност.
    \item[({\color{red}**}в)] Одредити отпорност отпорника $R$ који може да замени 
    структуру десно од кондензатора $C$ тако да се одбирци одскочног 
    одзива континуалног система одговарају добијеном дискретном одзиву у тренуцима 
    одабирања. У том случају, усвојити да се прекидачи затварају периодично, 
    учестаношћу $f$.
\end{enumerate}
\vspace*{2mm}

\textsc{\underline{Резултат:}}
(а) $v_{\rm I}[k] = \dfrac{C}{C + C_{\rm s}}( v_{\rm U}[k-1] - v_{\rm I}[k-1] )$.
(б) $v_{\rm I}[k] = V_0( 1 - (1/2)^n ) \uu[n]$. 
({\color{red}**}в) Треба да буде $R = \dfrac{1}{fC_{\rm s}}$