\PID
Нека је дат каузалан дискретан систем описан диференцном једначином 
$$y[n+2] - 5 y[n+1] + 6 y[n] = x[n],$$ где је побуда система 
дата изразом (а) $x[n] = 3\cdot 4^{-n} \uu[n]$, (б) $x[n] = 2 \cdot 3^n \uu[n]$. Одредити одзив датог система на побуду.     

\RESENJE 
Дата диференцијална једначина може се записати у облику 
$P(\EE) y[n] = x[n]$, где је $P(\EE) = \EE^2 - 5\EE + 6$. 
Одређивање одзива система налази се решавањем дате диференцне једначина. Пошто се тражи одзив система на побуду, 
сматрамо да су одговарајући помоћни услови пре дејства побуде дати као $y[-1] = y[-2] = 0$ (јер побуда делује од 
тренутка $n=0$). Одређивањем корена полинома $P(\EE)$, нпр. обрасцем за решења квадратне једначине, добијају се 
једноструки корени $\uplambda_1 = 3$ и $\uplambda_2 = 2$, па је хомогени део решења у општем облику:
\begin{eqnarray}
    y_{\rm h} = C_1 \cdot 3^n + C_2 \cdot 2^n.
\end{eqnarray}

Помоћни услови могу се одредити рекурентно, на сличан начин као у задатку \ref{z:diferencna_resi}, размотримо дату једначину за $n = 0$ и $n = 1$, па је 
\begin{eqnarray}
    n=0 &\Rightarrow& y[2] - \cancelto{0}{5 y[1]} + \cancelto{0}{6 y[0]} = x[0] \Rightarrow y[2] = x[0] \label{\ID.pom1}
    \\ 
    n=1 &\Rightarrow& y[3] - 5y[2] + \cancelto{0}{6y[1]} = x[1] \Rightarrow y[3] = x[1] + 5 y[2] = x[1] + 5 x[0]. \label{\ID.pom2}
\end{eqnarray}
На основу тих услова, биће могуће одредити потребне коефицијенте, када је познат побудни сигнал $x[n]$ и када се одреди партикуларни део одзива.



(a) Партикуларни део може се одредити на основу формуле дате у додатку \ref{dod:exp_response}, на основу чега је 
\begin{eqnarray}
    y_{\rm p}[n] = \dfrac{3 \cdot \left( \dfrac14 \right)^n }{ P\left( \dfrac14 \right) } = \dfrac{48}{77} \cdot \left( \dfrac14 \right)^n.
\end{eqnarray}
Одавде је облик одзива, за који треба наћи одговарајуће константе, на побуду дат изразом 
\begin{equation}
    y[n] = C_1 \cdot 3^n + C_2 \cdot 2^n +  \dfrac{48}{77} \cdot \left( \dfrac14 \right)^n.
\end{equation}
Помоћне услове рачунамо на основу \eqref{\ID.pom1} и \eqref{\ID.pom2} па је онда 
\begin{equation}
    y[2] = x[0] = 3, \qquad y[3] = x[1] + 5x[0] = \dfrac{63}{4}   
\end{equation}
па се решавањем система једначина на основу помоћних услова налазе $C_1 = \dfrac{12}{11}$ и $C_2 = -\dfrac{12}{7}$.

(б) Пошто је у овом случају вредност $a = 3$ једноструки корен полинома $P(\DD)$, побуђен је резонантни 
одзив система. Партикуларни део у том случају је дат формулом из додатка 
\ref{dod:exp_response},
\begin{eqnarray}
    y_{\rm p}[n] &=& \dfrac{ 2 n \cdot 3^{n-1} }{ P'(3) }, \qquad P'(\EE) = 2\EE - 5 \Rightarrow P'(3) = 1 \\
                 &=& \dfrac{2}{3}n \cdot 3^n,
\end{eqnarray}
па је на сличан начин као у претходној тачки укупан облик одзива 
\begin{eqnarray}
    y[n] = C_1 \cdot 3^n + C_2 \cdot 2^n + \dfrac{2}{3}n \cdot 3^{n}.
\end{eqnarray}
На основу истих помоћних услова као у претходној тачки, решавањем система једначина одређују се коефицијенти 
$C_1 = - \dfrac{9}{4}$, и $C_2 = \dfrac{45}{16}$.
