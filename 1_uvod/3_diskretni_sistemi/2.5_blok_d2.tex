\begin{slikaDesno}{fig/blok_D2.pdf}
    \PID У дискретном систему са слике употребљени су идеални блокови за кашњење,
    суматор и поjачавач непознатог поjачања $a \in \mathbb R$. 
    Једини улаз посматраног
    система jе дискретан сигнал $x = x[n]$, а једини излаз jе дискретан сигнал $y = y[n]$.
\end{slikaDesno}
\begin{enumerate}[label=(\alph*)]
    \item Написати диференцну једначину датог система.
    \item Испитати \myul{асимптотску} стабилност датог система у зависности од параметра $a$.
    \item Израчунати вредност параметра $a$ за коју jе систем маргинално стабилан.
    \item Одредити израз за импулсни одзив датог система, за произвољну вредност параметра $a$.
\end{enumerate}

\RESENJE
(а) Диференцна једначина система одређује се помоћу датог блок дијаграма. На „$+$“ улаз блока за сумирње 
доводи се сигнал $x$ док се на његов „$-$“ улаз доводи закашњени излазни сигнал $\DD y$. Након што 
тај сигнал прође кроз појачање и кашњење добија се излазни сигнал одакле се може писати 
једначина $a \DD ( x - \DD y) = y$, чијим се сређивањем налази диференцна једначина 
\begin{equation}
    (1 + a \DD^2) y = a \DD x. \label{\ID.difeq}
\end{equation}

(б) Уколико се добијена диференцна једначину \eqref{\ID.difeq} помножи са $\EE^2$ са обе стране добија се 
$(\EE^2 + a)y = a \EE x$, па је карактеристични полином дате диференцне једначине дат у облику 
$P(\uplambda) = \uplambda^2 + a$, а његови корени су 
$\uplambda_{12} = \pm \sqrt{-a}$. Да би дискретан систем био асимтотски стабилан, потребно је да се сви његови 
корени $\uplambda$, налазе унутар јединичног круга $|\uplambda| < 1$. Уколико постоје само 
једноструки корени на јединичној 
кружници $|\uplambda| = 1$ систем ће бити маргинално стабилан. 
Дакле, систем ће бити асимтотски стабилан за $|a| < 1$, маргинално стабилан за $|a| = 1$ док ће иначе 
систем бити асимтотски нестабилан.

(в) Систем ће бити у асимтотски стабилан у два случаја, односно за $a^{(1)} = 1$ и $a^{(2)} = -1$.  

(г) Импулсни одзив потребно је одредити у два случаја, $a \in \{1,-1\}$ према резултату претходне тачке. 
Рекурзивно, за диференцну једначину \eqref{\ID.difeq}, за $x[n] = \updelta[n]$ и $y[n] = h[n]$, може се писати 
\begin{eqnarray}
    n = 1 &\Rightarrow& h[1] + a \cancelto{0}{h[-1]} = a \cancelto{1}{\updelta[0]} 
    \Rightarrow h[1] = a, \\
    n = 2 &\Rightarrow& h[2] + a \cancelto{0}{h[0]} = a \cancelto{0}{\updelta[1]} 
    \Rightarrow h[2] = 0, \\
    n = 3 &\Rightarrow& h[3] + a h[1] = a \cancelto{0}{\updelta[2]} 
    \Rightarrow h[3] = - a h[1] = - a^2 \\
    n = 4 &\Rightarrow& h[4] + a h[2] = a \cancelto{0}{\updelta[3]} 
    \Rightarrow h[4] = 0 \\
    n = 5 &\Rightarrow& h[5] + a h[3] = a \cancelto{0}{\updelta[4]} 
    \Rightarrow h[5] = - a h[3] = a^3 \\
\end{eqnarray}
На основу добијених међурезултата, индуктивно се може закључити о општем резултату за импулсни одзив у облику 
\begin{equation}
    h[n] = 
    \begin{cases}
        0 &, n = 2m, m \in \mathbb N_0, \\
        (-1)^m a^{m} &, n = 2m + 1, m \in \mathbb N_0 
    \end{cases}.
\end{equation}
Читаоцу се препоручује да исти резултат изведе помоћу поступка са кореновима карактеристичног полинома.  
Заинтересованом читаоцу се саветује и да испита како се могу одредити 
решења диференцне једначине $P(\EE^2) y[n] = Q(\EE^2) x[n]$ полазећи од решења диференцне  
једначине $P(\EE) y[n] = Q(\EE) x[n]$. Односно, на који начин се замена $\EE \mapsto \EE^2$ 
одражава на импулсни одзив система. 