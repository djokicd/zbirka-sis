\noindent
\PID \label{z:parnost} \mnImportant
Одредити парну и непарну компоненту 
континуалних сигнала $x =x(t)$ за:
\begin{multicols}{3}
\begin{enumerate}[label=(\alph*)]
\item $x(t) = {\rm e}^{kt}$; и
\item $x(t) = {\rm e}^{{\rm j}\upomega_0 t}$,
\end{enumerate}
\end{multicols}
\noindent
где су $k$ и $\upomega_0$ познате реалне константе. \\[2mm]

\RESENJE  
Сваки континуални сигнал $x(t)$ може се, на \textit{јединствен} начин, представити преко његове парне и непарне компоненте, $x_{\rm e}(t) = {\rm Ev}\,x(t)$ и 
$x_{\rm o}(t) = {\rm Od}\,x(t)$ редом, као 
\begin{eqnarray}
    x(t) = x_{\rm e}(t) + x_{\rm o}(t).
    \label{eq:\ID.1}
\end{eqnarray}
Парна и непарна компонента сигнала могу се одредити
разматрањем израза за $x(-t)$ као и његове парне и непарне компоненте. Наиме,
\begin{equation}
    x(-t) = x_{\rm e}(-t) + x_{\rm o}(-t) \Rightarrow x(-t) = x_{\rm e}(t) - x_{\rm o}(t),
    \label{eq:\ID.2}
\end{equation}
при чему су искоришћени $x_{\rm e}(-t) = x_{\rm e}(t)$ и $x_{\rm o}(t) = -x_{\rm o}(-t)$, 
према дефиницији парне и непарне компоненте. Одатле се онда из система једначина 
\eqref{eq:\ID.1} и \eqref{eq:\ID.2} налазе изрази за парну и напарну компоненту сигнала датог као $x(t)$:
\begin{eqnarray}
    & x_{\rm e}(t) = \dfrac{ x(t) + x(-t) }{2},& \text{ и } \\
    & x_{\rm o}(t) = \dfrac{ x(t) - x(-t) }{2}.&
    \label{eq:\ID.3}
\end{eqnarray}


На основу добијеног резултата \eqref{eq:\ID.3}, онда се могу непосредно одредити парна и непарна компонента датих израза
\begin{enumerate}[label=(\alph*)]
    \item ${\rm Ev} \{ {\rm e}^{kt} \} = \dfrac{ {\rm e}^{kt} + {\rm e}^{-kt} }{2} = \cosh(kx)$, 
    и ${\rm Od}\{{\rm e}^{kt}\} =  \dfrac{ {\rm e}^{kt} - {\rm e}^{-kt} }{2} = \sinh(kx)$;
    \item ${\rm Ev}\{ {\rm e}^{{\rm j}\upomega_0 t}\} = \dfrac{ {\rm e}^{{\rm j}\upomega_0 t} + {\rm e}^{-{\rm j}\upomega_0 t} }{2} = \cos(\upomega_0 t)$,
    и ${\rm Od} \{ {\rm e}^{{\rm j}\upomega_0 t} \} =  \dfrac{ {\rm e}^{{\rm j}\upomega_0 t} - {\rm e}^{-{\rm j}\upomega_0 t} }{2} =\jj \sin(\upomega_0 t)$;
\end{enumerate}







