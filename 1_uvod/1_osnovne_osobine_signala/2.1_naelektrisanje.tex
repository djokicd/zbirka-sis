\noindent
\begin{slikaDesno}[0.833]{fig/Q.pdf}\noindent
\PID 
Танка нит дужине $a$ хомогено је наелектрисана укупним 
наелектрисањем $Q$ постављена је дуж $x$-осе, као на слици. 
Аналитички описати (а) функцију подужне густине 
наелектрисања $Q'(x)$. Одредити и (б) исту функцију за
$a \to 0$. 
\end{slikaDesno} \\[2mm]

\RESENJE 
(а) Подужна густина наелектрисања ван области нити је онда 
    $Q'(|x| > a/2) = 0$ док је у области нити константна, $Q'(|x| < a/2) = Q/a$.
    Ово се може записати као\footnote{ Јединична правоугаона функција дефинише се као \vspace*{1mm}
    $\rect(x) = \begin{cases}
        1, & |x| < \frac12 \\[3mm]
        1/2 & |x| = \frac{1}{2} \\[3mm]
        0, & |x| > \frac12 \end{cases}$, 
        па се правоугаона функција у интервалу $\displaystyle \left[-\frac{a}{2}, \frac{a}{2}\right]$ може дефинисати као
        $\rect\left(\dfrac{x}{a}\right)$.}
    \begin{equation}
        Q'(x) = \dfrac{Q}{a} \rect\left(\dfrac{x}{a}\right).
    \end{equation}
    
(б) Када је $a \to 0$, нит суштински постаје тачка, па је подужна густина наелектрисања
    $Q'(x) = Q \updelta(x)$, где је $\updelta(x)$ Дираков делта импулс. Такође, ово се може третирати као да је 
    $\updelta_a(x) = \dfrac{1}{a} \rect\left(\dfrac{x}{a}\right)$ језгро делта импулса, па у граничном процесу 
    \vspace*{1mm}
    када је $a \to 0$ постаје $\updelta(x) = \lim_{a \to 0} \updelta_a(x)$.

Слично као у задатку \ref{ID:capID}, димензија мере делта функције је укупно наелектрисање, будући да је мера
самог импулса дефинисаног у просторном домену димензије реципрочне дужине. 
У електромагнетици, тачкаста наелектрисања се третирају као делта импулси густине наелектрисања у простору, чиме се 
    омогућава аналитичко решавање проблема са тачкастим наелектрисањима у простору.