\noindent
\PID 
Одредити парну и непарну компоненту 
континуалног простопериодичног сигнала, облика 
$x(t) = \sin \left(\upomega_0 t + \uptheta \right)$, где су $\upomega_0$ и $\uptheta$ позанте константе. 
\\[2mm]

\textsc{\underline{Решење}}: Решење се може потражити поступком описаним у задатку 
\ref{z:parnost}, ипак, у случају овог задатка, али и разних других сродних примера, резултат се може пронаћи 
\textit{идентификацијом} парног и непарног дела сигнала, трансформацијом полазног израза. Применимо израз за 
синус збира\footnote{Синус збира углова 
$\sin(\upalpha + \upbeta) = \sin\upalpha \cos\upbeta + \cos\upalpha \sin\upbeta$} чиме се добија 
\begin{equation}
    x(t) = \sin \left(\upomega_0 t + \uptheta \right) 
    = \underbrace{\cos \uptheta}_{\rm const} \cdot \sin \upomega_0 t + \underbrace{\sin \uptheta}_{\rm const} \cdot \cos \upomega_0 t.
\end{equation}
Пошто је познато да су $\sin(\upomega_0 t)$ и $\cos(\upomega_0 t)$ непаран и паран сигнал редом, а знамо да се 
сигнал $x(t)$ може на јединствен начин представити као збир његових парних и непарних компоненти, онда 
морају бити
\begin{eqnarray}
    {\rm Od} \{ x(t) \} = \cos \uptheta \cdot \sin \upomega_0 t,& \text{ и } \\
    {\rm Ev} \{ x(t) \}  = \sin \uptheta \cdot \cos \upomega_0 t.&
\end{eqnarray}

Овакав поступак идентификације компоненти сигнала, често може брже довести до резултата од поступка описаног у задаку 
\ref{z:parnost}.
