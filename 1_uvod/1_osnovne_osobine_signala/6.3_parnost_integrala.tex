\PID 
Применом својстава парних и непарних 
сигнала
израчунати вредности 
одређених интеграла 
\begin{multicols}{2}
\begin{enumerate}[label=(\alph*)]
\item
$\displaystyle 
I_1 = 
\int_{-\uppi/2}^{\uppi/2}
\dfrac{\cos t}{1 + {\rm e}^{\sin 2t}}
 \de t;
$

\item
$\displaystyle
I_2 = \int\limits_{-\uppi/3}^{\uppi/3}
\dfrac{1 - t + 2t^3 - t^5 + 2t^7}{\cos^2 (t)}\,{\rm d} t$.
\end{enumerate}
\end{multicols}

\RESENJE 
Пошто су границе интеграла парне, може се користити својство да је 
\begin{equation}
\displaystyle \int_{-a}^{a} f(t) \de t = 2\int_{0}^{a} {\rm Ev}\{f(t)\} \de t, \label{eq:\ID.1}
\end{equation}
односно, потребно је потражити парне компоненте датих подинтегралних величина. 

(а) Парна компоненте подинтегралне величине налази се применом особина парности простопериодичних функција, 
применом поступка из задатка \ref{z:parnost}, као 
\begin{eqnarray}
    {\rm Ev}\left\{  
        \dfrac{\cos t}{1 + {\rm e}^{\sin 2t}}
    \right\}
    &= \dfrac{
        \dfrac{\cos t}{1 + {\rm e}^{\sin 2t}} + \dfrac{\cos (-t)}{1 + {\rm e}^{\sin 2(-t)}}
    }{2}
    =
    \dfrac{
        \dfrac{\cos t}{1 + {\rm e}^{\sin 2t}} + \dfrac{\cos t}{1 + {\rm e}^{-\sin 2t}}
    }{2}
    =  \\
    &= \dfrac{ \dfrac{\cos(t) \cancel{( 2 + \ee^{\sin 2t} + \ee^{-\sin 2t} )} } { \cancel{2 + \ee^{\sin 2t} + \ee^{-\sin 2t}} } }{2}
    = \dfrac{1}{2} \cos(t).
\end{eqnarray}
Заменом добијеног резултата у \eqref{eq:\ID.1}, коначно се налази, $I_1 = 1$.

(б) Сличним поступком се налази резултат $I_2 = 2\sqrt 3$.
