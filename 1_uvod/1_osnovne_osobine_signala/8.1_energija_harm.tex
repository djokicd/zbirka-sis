\noindent 
\PID\mnImportant
Извести израз за снагу сигнала: 
$$
x(t) = a_0 + a_1\cos(\upomega_0 t) +
a_2 \cos(2\upomega_0 t) + \cdots 
+ a_n \cos(n\upomega_0 t), \qquad (n\in\mathbb N)
$$
где су $a_0$, $a_1, a_2, \ldots, a_n$ познате реалне
константе.
\\[2mm]

\RESENJE 
Периоди сигнала $x(t)$ су $T_0 = \dfrac{2\uppi}{\upomega_0}$ и $T_k = \dfrac{T_0}{k}$, где је $k \in \{2,3,\ldots,n\}$, 
па је укупни период сигнала $x(t)$ једнак $T_0$. Снага сигнала $x(t)$ може се изразити као 
\begin{equation}
P_x = \dfrac{1}{T_0}\int_{T_0} x^2(t) \, dt =  \dfrac{1}{T_0} \int_{T_0} \left( a_0 + a_1\cos(\upomega_0 t) +
a_2 \cos(2\upomega_0 t) + \cdots
+ a_n \cos(n\upomega_0 t) \right)^2 \, \de t \\
\end{equation}
Подинтегрална величина се може поделити на чланове у две различите врсте, квадрат сваког члана у изразу, и производ различитих
чланова у изразу, чиме се добија израз
\begin{eqnarray}
T_0 P_x = \int_{T_0} a_0^2 \, \de t + \int_{T_0} a_1^2\cos^2(\upomega_0 t) \, \de t   \label{eq:\ID.1}
+ \int_{T_0} a_2^2 \cos^2(2\upomega_0 t) \, \de t + \cdots + \int_{T_0} a_n^2 \cos^2(n\upomega_0 t) \, \de t + \\
+ 2a_0a_1\int_{T_0} \cos(\upomega_0 t) \, \de t + 
2a_0a_2\int_{T_0} \cos(2\upomega_0 t) \, \de t + \cdots + \label{eq:\ID.2}
2a_0a_n\int_{T_0} \cos(n\upomega_0 t) \, \de t + \\
+ 2a_1a_2\int_{T_0} \cos(\upomega_0 t)\cos(2\upomega_0 t) \, \de t + \cdots + 
2a_{n-1}a_n\int_{T_0} \cos((n-1)\upomega_0 t)\cos(n\upomega_0 t) \, \de t + \cdots   \label{eq:\ID.3}
\end{eqnarray}
Интеграли у изразу \eqref{eq:\ID.1} су квадрати простопериодичних величина\footnote{
За сигнале $\cos^2(n\upomega t)$, са периодом $T$, средња вредност је $\dfrac{1}{2}$ па је 
$\int_{T_0} \cos^2(n\upomega t) \de t = \dfrac{T}{2}$.
}, па је њихова вредност једнака $T_0/2$. У реду \eqref{eq:\ID.2} су интеграли 
простопериодичних функција па су они сви равни нули, док су у реду \eqref{eq:\ID.3} сви интеграли 
типа\footnote{
    Доказ овог става следи из чињенице да су сигнали $\cos(n \upomega_0 t)$ и $\cos(m \upomega_0 t)$
    ортогонални за $m \neq n$ у смислу дефинисаном у задатку \ref{z:zbirP}a. Овај став је од 
    нарочитог интереса за увођење Фуријеових редова.
}
$\displaystyle\int_T \cos(n\upomega t) \cos(m\upomega t) \de t$, за $n \neq m$ равни нули. На основу тога, 
коначно се има
\begin{eqnarray}
    P_x = a_0^2 + \dfrac{ a_1^2 + a_2^2 + \cdots + a_n^2 }{2}
\end{eqnarray}

Резултат се може добити и уопштењем става из задатка \ref{z:zbirP}. Будући да су сабирци израза у задатку облика 
да је, интеграл на периоду производа свака два од њих раван нули, то је снага тог сигнала једнака збиру снага сваког 
од сигнала понаособ. Сигнал $a_0$ је константни сигнал, па је његова снага једнака квадрату амплитуде 
$a_0^2$, док је снага сигнала $a_n \cos(n\upomega_0 t)$, $n \in \mathbb N$, једнака $a_n^2/2$ (квадрат ефективне вредности
$a_n/\sqrt 2$). Сабирањем таквих снага, добија се исти коначан резултат.
 