\noindent\mnImportant
\begin{slikaDesno}[0.833]{fig/C.pdf}\noindent
\PID \label{ID:capID}
У колу са слике познато је 
$C = 1 \unit{\upmu F}$. 
Идеалан прекидач П је отворен, а кондензатор је оптерећен количином
наелектрисања $Q = 1 \unit{\upmu C}$. У тренутку $t_0 = 0$ затвара се прекидач. 
Одредити $v_{C} = v_{C}(t)$        
и $i_{\Uppi} = i_{\Uppi}(t)$, за $-\infty < t < \infty$. 
\end{slikaDesno}

\REZULTAT 
Према услову задатка је 
\vspace{1mm}
$v_C(t < 0) = \dfrac{Q}{C} = 1\unit{V}$. Према карактеристици идеалног прекидача
након затварања прекидача је $v_C(t > 0) = 0$, обједињено ово се може записати у условном облику као 
$
v_C(t) = 
\begin{cases}
        1\unit{V},&  t < 0 \\
        0,        &  t > 0
\end{cases}.
$ Добијени израз се може записати и помоћу Хевисајдове одскочне функције као 
\begin{equation}
v_{\rm C} = 1\unit{V} ( 1 - \uu(t) ). 
\end{equation}
Тако записан израз је нарочито користан за одређивање тражене струје, пошто је струја кондензатора, 
за референтни смер усклађен са напоном, 
константне капацитивности дата изразом $i_C = C\dfrac{\de v_C}{\de t}$, она се може потражити као
$
i_C = C \dfrac{\de v_{\rm C}}{\de t} = - 1\unit{\upmu C} \dfrac{\de \uu(t)}{\de t}  = 
- 1\unit{\upmu C} \, \updelta(t). 
$
Пошто је добијена струја кондензатора у супротном референтном смеру од струје прекидача, коначан 
резултат за струју прекидача је 
$i_{\Uppi}(t) = 1\unit{\upmu C} \, \updelta(t)$.

Важно је прокоментарисати две ствари у вези са овим резултатом. Прво, физички смисао Дираковог импулса може
се потражити у свим појавама које трају веома кратко а које имају коначан утицај. У овом случају, 
за „бесконачно кратко“ време кроз прекидач протекне целокупно наелектрисање кондензатора. Са друге 
стране, приметимо да је димензија сигнала $\delta(t)$ заправо $\unit{s^{-1}}$, па је мера Дираковог импулса
који одговара струји заправо количина наелектрисања. Ово је конзистентно са дефиниционим својством
Дираковог импулса према $\DS \int_{-\infty}^{\infty} 
\underbrace{\updelta(t)}_{\unit{s^{-1}}} \, 
\underbrace{\de t}_{\unit s} = \underbrace{1}_{[\cdot]}$.