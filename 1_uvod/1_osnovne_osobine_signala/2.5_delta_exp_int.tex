\PID 
Нека је дат
континуалан сигнал
$
\DS x(t) = {\rm e}^{\upsigma t} \,
%\sum_{k = -\infty}^\infty \updelta(t - kT)$ 
\text{Ш}_T(t)
\,\uu(t + \upepsilon)$, где је  $0 < \upepsilon < T$.
(а) Одредити услов које треба да задовољава параметар $\upsigma\in\mathbb R$ тако да интеграл
$\DS \int_{-\infty}^{\infty} \hspace*{-0.5em} x(t)\,\de t$ конвергира, 
и у том случају (б) израчунати  тај интеграл.
\vspace{5mm}

\RESENJE  (а) Дата поворка делта импулса се може расписати по дефиницији, а 
затим се може применити особина еквиваленције\footnote{Особина \textit{еквиваленције} је 
$x(t) \updelta(t - t_0) = x(t_0) \updelta(x - t_0)$. } делта импулса према
\begin{equation}
    x(t) = {\rm e}^{\upsigma t} \,
    \underbrace{ \sum_{k = -\infty}^\infty \updelta(t - kT) }_{\text{Ш}_T(t)}
    \,\uu(t + \upepsilon) = 
    \sum_{k = -\infty}^\infty \underbrace{ {\rm e}^{\upsigma t} \updelta(t - kT) }_{\text{особина екв.}}
    \,\uu(t + \upepsilon)
    =
    \sum_{k = -\infty}^\infty {\rm e}^{\upsigma kT} \updelta(t - kT)   
    \,\uu(t + \upepsilon).
\end{equation}
Интеграл датог израза се онда може израчунати заменом редоследа интеграције и 
сумирања\footnote{Строго оправдање замене интеграла и суме је сложено. Ипак, у инжењерским применама, 
користимо то без оправдања будући да се \textit{патолошки} случајеви где то није оправдано у пракси 
практично не јављају. Практично сматрамо да је $\sum\int \equiv \int\sum$.} према поступку:
\begin{equation}
    \int_{-\infty}^{\infty} \hspace*{-0.5em} x(t)\,\de t 
    = 
    \underbrace{
    \int_{-\infty}^{\infty}
    \sum_{k = -\infty}^\infty}_{\text{замена}} {\rm e}^{\upsigma kT} \updelta(t - kT)   
    \,\uu(t + \upepsilon)
    \,\de t 
    =
    \sum_{k = -\infty}^\infty
    \underbrace{{\rm e}^{\upsigma kT} }_{\const}
    \int_{-\infty}^{\infty} 
    \updelta(t - kT)   
    \,\uu(t + \upepsilon)
    \,\de t.
    \label{eq:\ID.2}
\end{equation} 
У последњем написаном интегралу, члан $\uu(t + \upepsilon)$ ограничава интеграл са леве стране 
чиме се онда интеграл решава провером да ли Делта импулс $\updelta(t - kT)$ постоји у домену 
инетеграције $t \in (-\upepsilon, \infty)$. као 
\begin{equation} 
    \int_{-\infty}^{\infty} 
    \updelta(t - kT)   
    \,\uu(t + \upepsilon)
    \,\de t = 
    \int_{-\upepsilon}^{\infty} \updelta(t - kT) \de t = 
    \begin{cases}
        0 &, k < 0 \\
        1 &, k \geq 0
    \end{cases}
    = \uu[k]\text{\quad(дискретан јединични низ)}
\end{equation}
Сменом добијеног резултата у \eqref{eq:\ID.2} добија се геометријски 
ред\footnote{Сума геометријског реда је облика
$\DS \sum_{k = 0}^{\infty} q^k = \dfrac{1}{1-q}$, под условом да је $|q| < 1$. }
ограничен са леве стране 
\begin{equation}
    \sum_{k = -\infty}^{\infty} {\rm e}^{\upsigma k T} \uu[k]
    =
    \sum_{k = 0}^{\infty} (\ee^{\upsigma T})^k 
    = 
    \dfrac{1}{1 - \ee^{\upsigma T}},\quad \text{под условом конвергенције: } \bigl|\ee^{\upsigma T}\bigr|< 1.
    \label{eq:\ID.3}
\end{equation}
Добијени услов конвергенције последица је суме геометријског реда и даје услов па је
тражени услов $\upsigma < 0$, а тражени интеграл дат је резултатом у изразу \eqref{eq:\ID.3}.   

Читаоцу се препоручује да понови задатак у случају да је $\upsigma \in \mathbb C$.
\newpage