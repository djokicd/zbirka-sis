\begin{slikaDesno}{fig/po.pdf}
\PID 
Фреквенцијска карактеристика идеалног филтра пропусника опсега учестаности је 
$H(\jj\upomega) = 
\begin{cases}
    1 &, \upomega_0 < |\upomega| < 3\upomega_0 \\
    0 &, \text{иначе}
\end{cases}.$ Одредити импулсни одзив тог филтра. 
\end{slikaDesno}

\RESENJE 

Фреквенцијска карактеристика састоји се од два правоугаона прозора полуширине $\upomega_0$ центриране на кружним учестаностима 
$\pm 2\upomega_0$, што се може записати као 
\begin{eqnarray}
    H(\jj\upomega) = \rect_{\upomega_0} (\upomega - 2\upomega_0) + \rect_{\upomega_0} (\upomega + 2\upomega_0).
\end{eqnarray}
Добијена померања у учестаности могу се изразити конволуцијом са одговарајућим Дираковим импулсима одакле се може писати 
\begin{eqnarray}
    H(\jj\upomega) = \rect_{\upomega_0} (\upomega) \ast 
    \bigl(
        \updelta(\upomega - 2\upomega_0) + \updelta(\upomega + 2\upomega_0)
    \bigr).
\end{eqnarray}
Инверзна Фуријеова трансформација одређује се применом теореме о конволуцији по учестаности\footnote{
   $\FT{x(t) \cdot y(t)} = \dfrac{1}{2\uppi} X(\jj\upomega) \ast Y(\jj\upomega)$
}, одакле се има
\begin{eqnarray}
    h(t) &=& 2\uppi
    \underbrace{\IFT{ \rect_{\upomega_0}(\upomega) }}_{
        \text{Задатак \refz{lpf} }
    } \cdot 
    \underbrace{\IFT{ \updelta(\upomega - 2\upomega_0) + \updelta(\upomega + 2\upomega_0) }}_{
        \text{Таблица, \reft{T:ctft:cos}: } \frac{1}{\uppi} \cos(2\upomega_0 t)
    } \\
    &=& 2 \cancel{\uppi}
    \dfrac{\upomega_0}{\cancel{\uppi}} \sinc \left( \frac{\upomega_0 t}{\uppi} \right) 
    \cdot 
    \dfrac{1}{\uppi} \cos(2\upomega_0 t)
    = 
    \dfrac{2 \upomega_0}{\uppi} \cos(2\upomega_0 t) \sinc \left( \frac{\upomega_0 t}{\uppi} \right),
\end{eqnarray}
што је и требало одредити. 