\begin{slikaDesno}{fig/rect_pulse_0_T.pdf}
\PID 
\label{ID:rect_pulse_spectrum}
На слици \ID.1 приказан је правоугаони импулс јединичне амплитуде 
ширине $T$, са почетком у нули. Одредити Фуријеову трансформацију тог сигнала. \\

\hspace{4mm}
\textsc{\underline{Решење}:} Дати сигнал се може записати у облику 
$x(t) = \uu(t) - \uu(t - T)$. Применом особине померања у времену Фуријеове трансформације\footnotemark
и табличног резултата $\mathcal{FT}\{\uu(t)\} = \dfrac{1}{\jj\upomega} + \uppi\updelta(\upomega)$,
има се резултат
$X(\jj\upomega) = \dfrac{1}{\jj\upomega} + \uppi \updelta(\upomega) - 
        \dfrac{\ee^{-\jj\upomega T}}{\jj\upomega} - \uppi \ee^{-\jj\upomega T}\updelta(\upomega) 
        =\dfrac{1 - \ee^{-\jj\upomega T} }{\jj\upomega} + 
        \cancelto{0}{
        \uppi(\underbrace{1 - \ee^{-\jj\upomega T}}_{=0 \text{ за } \upomega = 0}) \updelta(\upomega)
        }$, где је у последњем кораку примењено својство еквиваленције Дираковог импулса. 
Коначно је $X(\jj\upomega) = \dfrac{1 - \ee^{-\jj\upomega T} }{\jj\upomega}$
\end{slikaDesno}
\footnotetext{Својство померања ФТ је $\mathcal{FT}\{x(t-T)\} = \mathcal{FT}\{x(t)\}\cdot\ee^{-\jj\upomega T}$.}


