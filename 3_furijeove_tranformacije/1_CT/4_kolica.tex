\begin{slikaDesno}{fig/kolica.pdf}
    \PID
    У механичком систему са слике колица масе $m$, са точковима занемарљиве масе, 
    везана су за непокретан зид опругом коефицијента крутости $k$ и амортизером 
    коефицијента пригушена $b$. Побуда посматраног система jе алгебарски
    интензитет принудне силе 
    ${\bf F} (t) = F \cdot {\bf i}_x$. Одзив система $x(t)$ jе отклон колица 
    у односу на равнотежни положај $x_0 = 0$.
\end{slikaDesno}
\begin{enumerate}[label=(\alph*)]
    \item Одредити  функцију
    преноса овог система. 
    \item Одредити кружну учестаност $\upomega$ простопериодичне побуде
    $F(t) = F_0 \cos(\upomega t)$ тако да амплитуда осцилација колица буде максимална
    применом Фуријеове транформације.
\end{enumerate}

\REZULTAT
(а) Функција преноса посматраног система је 
$H(\jj\upomega) = \dfrac{X(\jj\upomega)}{F(\jj\upomega)} = 
\dfrac{1}{k - \upomega^2 m + \jj\upomega b}$.

(б) Учестаност при којој се остварује максимална амплитуда осциловања колица је
$\upomega_{\rm m} = 
\sqrt{
    \dfrac{k}{m}
    -
    \dfrac{b^2}{2m}
}$.