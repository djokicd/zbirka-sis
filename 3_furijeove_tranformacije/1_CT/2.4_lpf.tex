\begin{slikaDesno}{fig/nf.pdf}
\PID \label{z:lpf}
Фреквенцијска карактеристика идеалног филтра пропусника ниских учестаности је 
$H(\jj\upomega) = 
\begin{cases}
    1 &, |\upomega| < \upomega_0 \\
    0 &, \text{иначе}
\end{cases}.$ Одредити импулсни одзив тог филтра. 
\end{slikaDesno}

\RESENJE 

Дата функција преноса може се записати као правоугаони прозор полуширине 
$\upomega_0$, у облику $\rect_{\upomega_0}(\upomega)$.
Такав правоугаони прозор може се изразити помоћу јединичног правоугаоног прозора коначно
$\rect_{\upomega_0}(\upomega) = \rect \left( \dfrac{\upomega}{2\upomega_0} \right)$.
Помоћу табличне трансформације \reft{T:ctft:sinc} и особине скалирања аргумента\footnote{
    $\FT{x(at)} = \dfrac{1}{|a|} X\left( \dfrac{\upomega}{a} \right)$;
} може се писати
\begin{eqnarray}
    && \FT{ \sinc(t) } = \mathrm{rect}\left(\frac{\upomega}{2\uppi}\right)
    \Rightarrow
    \FT{ a \sinc(at) } = \mathrm{rect}\left(\frac{\upomega}{2\uppi a}\right) \\ 
    && a = \dfrac{\upomega_0}{\pi} \Rightarrow
    \FT{ \dfrac{\upomega_0}{\uppi} \sinc \left( \dfrac{\upomega_0 t}{\uppi} \right) } = \mathrm{rect}\left(\frac{\upomega}{2 \upomega_0}\right) 
    =   \rect_{\upomega_0} (\upomega) \\ 
    && 
    h(t) 
    = 
    \IFT{ \rect_{\upomega_0} (\upomega)  } = \dfrac{\upomega_0}{\uppi} \sinc \left( \dfrac{\upomega_0 t}{\uppi} \right),
\end{eqnarray}
што је требало и одредити. 
