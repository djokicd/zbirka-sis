\mnDifficult
\begin{slikaDesno}[0.833]{fig/rc_nf_filt.pdf}
\PID 
У електричном колу са слике 
приказан је нископропусни филтар
(НФ)
за који је познато
$R = {60}\unit{\Omega}$,  и 
${C = \dfrac{5}{3}\unit{mF}}
\vspace*{1mm}
$, а 
који је оптерећен резистивним потрошачем
отпорности
$R_{\rm p} = 20\unit{\Omega}
\vspace*{1mm}$. На улазу филтра прикључен 
је идеални напонски генератор чији 
је напон облика 
${v_{\rm U}(t) = V_0 
{\rm e}^{-t/\uptau}\,\uu(t)}$, где су
$V_0 = 64\unit{V}$ и 
$\uptau = 25\unit{ms}$. У почетном тренутку
кондензатор је неоптерећен.
\end{slikaDesno}

\begin{enumerate}[label=(\alph*)]
    \item Одредити преносну функцију система 
    $H(s) = \dfrac{V_{\rm I}(s)}{V_{\rm U}(s)}$, $s = \jj\upomega$.
    \item  
    Израчунати граничну учестаност 
    оптерећеног нископропусног
    филтра  $\upomega_{\rm f}$, као 
    $|H({\rm j}\upomega_{\rm f})| = \dfrac{1}{\sqrt 2} 
    \cdot |H({\rm j}\upomega)|_{\rm max}$.
    \item
    Применом Парсевалове теореме израчунати
    укупну енергију предату потрошачу 
    у интервалу времена $0 \leq t < \infty$
\end{enumerate} 

\textsc{\underline{Резултат}:} 
Задатак се може решити применом Тевененове теореме, у односу
на кондензатор. 
Том приликом је $v_{\rm T} = \upalpha_{\rm T} v_{\rm U}$ и
$R_{\rm T} = R\,||\,R_{\rm p} 
= 15\unit{\Omega}$, где је 
$\upalpha_{\rm T} = \dfrac{R_{\rm p}}{R + R_{\rm p}}
= \dfrac{1}{4}$. Преносна функција је 
$H(s) = \dfrac{\upalpha_{\rm T}}{
R_{\rm T}Cs + 1} =  
\dfrac{ \dfrac{R_{\rm p}}{R + R_{\rm p}} }{
(R\,||\,R_{\rm p})Cs + 1
} =
\dfrac{1}{4} \cdot
\dfrac{1}{  (25\unit{ms}) s + 1  }$. 
(б) Гранична учестаност филтра је 
$\upomega_{\rm f} = 40\unit{\dfrac{rad}{s}}$.
(в) Укупна енергија ослобођена на потрошачу 
налази се Парсеваловом теоремом и износи
$\displaystyle
W_{\rm p} = 
\dfrac{1}{R_{\rm p}}
\dfrac{1}{\uppi} 
\int_0^{\infty}
|H(\jj\upomega)V_{\rm U}(\jj\upomega)|^2\,
\de\upomega = 
\dfrac{V_0^2 \upalpha_{\rm T}^2 \uptau^2}{
2(R_{\rm T}C + \uptau)R_{\rm p}} = 
128\unit{mJ}
$.