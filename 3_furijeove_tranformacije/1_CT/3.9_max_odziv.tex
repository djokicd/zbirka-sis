\noindent
\begin{slikaDesno}{fig/max_odziv.pdf}
    \PID 
    У систему са слике користе се идеални диференцијатори и множачи познатим 
    константама $a$, $b$, и $1/c$. 
    Одредити учестаност $\upomega_{\rm m}$ при којој се остварује максимална 
    вредност амплитудске фреквенцијске карактеристике датог 
    $\upomega_{\rm m} = \arg \max |H(\jj\upomega)|$.     
\end{slikaDesno}

\RESENJE

На основу резултата задатка \refz{dif_pf} се блок диференцијатора у фреквенцијском домену може заменити множењем са 
$\jj\upomega$. Тако се праћењем тока сигнала добија 
$((\jj\upomega)^2 a + (\jj \upomega) b  + c) Y(\jj\upomega) = X(\jj\upomega)$, односно, функција преноса целог система је
\begin{equation}
    H(\jj\upomega) = 
    \dfrac{Y(\jj\upomega)}{X(\jj\upomega)} = \dfrac{1}{  \jj \upomega b  + c - \upomega^2 a}
\end{equation}

Учестаност при којој се постиже максимална вредност амплитудске карактеристике се онда може потражити као 
\begin{equation}
    \upomega_{\rm m} = \arg \max |H(\jj\upomega)| = \arg \max |H(\jj\upomega)|^2 = \arg \max \dfrac{1}{ (\upomega b)^2 + ( c - \upomega^2 a )^2 }.
\end{equation}
Ради једноставности поступка, максимум се може добити и одређивањем минимума реципрочне вредности, помоћу првог извода, 
\begin{eqnarray}
    \upomega_{\rm m} = \arg \min \left(   (\upomega b)^2 + ( c - \upomega^2 a )^2  \right) 
    &\Rightarrow&
    \dfrac{\de }{\de \upomega} \left(   (\upomega b)^2 + ( c - \upomega^2 a )^2  \right) = 0 \\
    &\Rightarrow&
    \cancel{2\upomega} b^2 - \cancel{2\upomega} \, 2 (c - \upomega^2 a) = 0 \\
    &\Rightarrow&
    \upomega_{\rm m} = \pm \sqrt{ \dfrac{c}{a} - \dfrac{b^2}{2a}  }
\end{eqnarray}