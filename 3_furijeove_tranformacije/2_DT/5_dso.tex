\begin{slikaDesno}{fig/dso_plot.pdf}
    \PID 
    Дигиталним осцилоскопом снима се континуалан простопериодичан
    напон, $v = v(t)$, непознате амплитуде $V_{\rm m}$ и непознате учестаности $f_0$. 
    Сигнал jе снимљен учестаношћу одабирања $f_{\rm s} = 0,5\unit{GHz} > 2 f_0$ , чиме jе у
    мемориjи осцилоскопа
    добиjен дискретан представник мереног сигнала коначног трајања
    $x[n] = \begin{cases}
        \dfrac{v(nT_{\rm s})}{V_0},& -N \leq n \leq N \\
        0,& \text{иначе} 
    \end{cases}$, 
    где је $V_0 = 10\unit{mV}$, а $2N+1 \in \mathbb N$ је непозната дужина снимка. 
    У обради добиjеног сигнала на рачунару израчуната jе Фуриjеова
    трансформациjа дискретног сигнала $x[n]$.  Амплитудски спектар jе приказан у опсегу
    дискретних кружних учестаности $0 \leq \Omega \leq \uppi$ на слици. 
\end{slikaDesno}
\begin{enumerate}[label=(\alph*)]
    \item Проценити учестаност $f_0$ улазног сигнала, 
    \item Проценити дужину снимка $2N + 1$.
    \item Проценити амплитуду $V_{\rm m}$ улазног сигнала.
\end{enumerate}

\textsc{\underline{Решење}:} Нека је облик побудног напона $v(t) = V_{\rm m} \cos(\upomega t)$.
Приметимо да се акција одсецања снимка коначног трајања може имитирати множењем са 
прозором коначне дужине у виду $x[n] = \underbrace{ \dfrac{V_{\rm m}}{V_{\rm 0}} }_{X_{\rm m}} 
\cos( \underbrace{\upomega T_{\rm s}}_{\Omega_0} n)
\cdot \rect_N[n]$. Фуријеова трансформација датог сигнала може се добити помоћу табличних 
резултата
\begin{eqnarray}
    \FT{ \cos(\Omega_0 n) }  &=&  \uppi \sum_{k = -\infty}^{\infty} \bigl( 
        \updelta(\Omega - \Omega_0 + 2k\uppi) +  \updelta(\Omega + \Omega_0 + 2k\uppi) 
    \bigr) 
    \\
    \FT{ \rect_N [n] } &=& \dfrac{\sin( \Upomega(N + 0,5) ) }{ \sin(\Upomega/2) },
    \quad\quad\quad\quad\text{(Видети и зададтак \ref{z:dtft_rec})}
\end{eqnarray}
применом правила о производу дискретне Фуријеове трансформације 
$\FT{x[n] \cdot y[n]} = \FT{x[n]} \circledast \FT{y[n]}$. Кружну конволуцију можемо одредити по дефиницији 
на основу чега се добија 
\begin{eqnarray}
    X(\jj\Omega) &=& X_{\rm m} \int_{\upxi = -\uppi}^{\uppi} 
    \bigl( \updelta(\Omega - \Omega_0) +  \updelta(\Omega + \Omega_0)  \bigr)
    \dfrac{\sin( (\Omega - \upxi)(N + 0,5) ) }{ \sin((\Omega - \upxi)/2) } \de \upxi  \\
    &=&
    X_{\rm m} \dfrac{\sin( (\Omega - \Omega_0)(N + 0,5) ) }{ \sin((\Omega - \Omega_0)/2) }
    +
    X_{\rm m} \dfrac{\sin( (\Omega + \Omega_0)(N + 0,5) ) }{ \sin((\Omega + \Omega_0)/2) } \label{\ID.sh}
\end{eqnarray}

(а) Приметимо сада да је синусна функција померила спектар правоуганог импулса у десно и у лево за 
$\Omega_0$, израз \eqref{\ID.sh}. Одатле, природно је очекивати да се максимум спектра сада налази на дискретној учестаности 
$\Omega_0$. На основу тога, са графика очитавамо резултат $\Omega_0 = \dfrac{2\uppi}{5}$ на основу чега можемо да 
се има $f = \dfrac{\upomega}{2\uppi} = \dfrac{\Omega}{2\uppi T_{\rm s}} = \dfrac{ \dfrac{2\uppi}{5} }{2 \uppi T_{\rm s}}
= \dfrac{1}{5 T_{\rm s}} = 100 \unit{MHz}$. 

(б) Претпоставимо, на основу облика спектра, да су спектри правоугаоних прозора довољно размакнути тако да 
за  $\Omega \approx +\Omega_0$ важи $X(\jj\Omega) \approx X_{\rm m} \dfrac{\sin( (\Omega - \Omega_0)(N + 0,5) ) }{ \sin((\Omega - \Omega_0)/2) }$, 
(односно, занемаримо лик који постоји у негативним учестаностима). На основу тога, можемо очекивати нуле спектра, 
$\Omega_z$,  
на местима где је $(\Omega_z - \Omega_0)(N + 0,5) \approx \uppi$, односно је 
$ N \approx \dfrac{\uppi}{\Omega_z - \Omega_0} - 0,5$. Са графика очитавамо да је 
${\Omega_z - \Omega_0} \approx \dfrac{\uppi}{10}$, па на крају закључујемо да је $N \approx 9,5$, па заокружујемо на 
ближу непарну вредност $N = 9$, па можемо да се проени да је снимак дугачак $\approx 19$ одбирака. 

(в) Познајући резултате претходних тачака, размотримо максималну вредност функције у тачки $\Omega \approx \Omega_0$. 
На том месту је 
$X(\jj\Omega_0) \approx \lim_{\Omega \to \Omega_0} X_{\rm m} \dfrac{\sin( (\Omega - \Omega_0)(N + 0,5) ) }{ \sin((\Omega - \Omega_0)/2) }
= X_{\rm m} 2(N + 0,5) \approx 19 X_{\rm m}$. Са графика очитавамо да је максимална вредност $10$ па се онда има 
$X_{\rm m} \approx \dfrac{10}{19} \approx 0,52$. Одавде закључујемо да је $V_{\rm m} = V_0 \cdot X_{\rm m} 
\approx 52\unit{mV}$.