\PID 
Применом $\mathcal{Z}$-трансформациjе одредити сопствени одзив система описаног диференцном jедначином
$$
    y[n+2] - 5 y[n + 1] + 6 y[n] = x[n],
$$
где су $x[n]$ и $y[n]$ улаз и излаз тог система редом, а дати су помоћни услови $y[0] = y[1] = 1$.

\RESENJE

Приликом тражења сопственог одзива, побуда се анулира па је $x[n] = 0$.
Користе се особине предикције у времену $\mathcal{Z}$-трансформациjе\footnote{
    $\ZT{x[n+1]} = z X(z) - z x[0]$
}, на основу којих се дата диференцна једначина може превести у комплексан домен 
\begin{eqnarray}
    &&
    \ZT{y[n+1]} = z Y(z) - z y[0] 
    \\
    &&
    \ZT{y[n+2]} = z \ZT{y[n+1]}  - z y[1] 
    \Rightarrow
    \ZT{y[n+2]} = z^2 Y(z) - z^2 y[0]  - z y[1] 
\end{eqnarray}
Заменом у полазну диференцну једначину, након уређивања се добија 
\begin{eqnarray}
    && 
    \ZT{y[n+2]} - 5 \ZT{y[n + 1]} + 6 \ZT{y[n]} = 0 \\
    && 
    z^2 Y(z) - z^2 y[0]  - z y[1] 
    -5z Y(z) + 5z y[0]
    + 6 Y(z) = 0 \\
\end{eqnarray}
Даљим сређивањем одређује се $\mathcal{Z}$-трансформација сопственог одзива као
\begin{eqnarray}
    && (z^2 - 5z + 6) Y(z) = z^2 \cancelto{1}{y[0]} - 5z \cancelto{1}{y[0]} + z \cancelto{1}{y[1]}  \\
    && Y(z) = \dfrac{z^2 - 4z}{z^2 - 5z + 6}
\end{eqnarray}
За растављање на парцијалне разломке степен полинома у бројиоцу мора бити нижи него у имениоцу. У општем случају овај проблем 
могуће је решити дељењем полинома, међутим, у овом случају практичније је извући заједнички члан „$z$“ па извести парцијалне разломке 
над остатком помоћу поступка из \ref{a:pfd}
\begin{eqnarray}
    Y(z) = z \dfrac{z - 4}{z^2 - 5z + 6} = \dfrac{2z}{z - 2} - \dfrac{z}{z - 3},
\end{eqnarray}
па се коришћењем табличне транформације \reft{T:z:exp} добија коначни резултат 
\begin{equation}
    y[n] = \IZT{Y(z)} = (2 \cdot 2^n - 3^n) \uu[n].
\end{equation}
