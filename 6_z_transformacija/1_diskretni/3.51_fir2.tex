
\begin{slikaDesno}{fig/fir_2.pdf}
    \PID У FIR филтру приказаном на слици употребљени су идеални блокови за кашњење, идеални множачи константом и идеални сабирачи.
    Коефицијенти множача представљају биномне коефицијенте $a_k = \binom{n}{k}$, за $k = 0,1,2,\ldots,n$, где је $n$ познати број 
    употребљених блокова за кашњење. 
\end{slikaDesno}

\REZULTAT
Тражена функција преноса је 
$H(z) = (1 + z^{-1})^n$, искористити Њутнову биномну формулу 
$(a + b)^n = \sum_{k = 0}^{n} \binom{n}{k} a^k b^{n-k}$. 

Нагласимо да се добијени филтар назива још и биномни филтар. Пошто биномни коефицијенти апроксимирају Гаусову расподелу, овакав филтар се може 
користити за ефикасну хардверску реализацију Гаусовског НФ филтра.