
\begin{slikaDesno}{fig/fir_2.pdf}
    \PID У FIR филтру приказаном на слици употребљени су идеални блокови за кашњење, идеални множачи константом и идеални сабирачи.
    Познато је да функција преноса има једну $m$-тоструку нулу у $z = -1$, а да је појачање константног сигнала $H_0 = 1$. 
    Одредити коефицијенте множача $a_k$, за $k = 0,1,2,\ldots,n$, где је $n$ број који треба том 
    приликом одредити.
\end{slikaDesno}

\REZULTAT
Треба да буде $n=m$ а тражени коефицијенти су $a_k = \dfrac{1}{2^m} \binom{n}{k}$.
Нагласимо да се добијени филтар назива још и биномни филтар. Пошто биномни коефицијенти апроксимирају Гаусову расподелу, овакав филтар се може 
користити за ефикасну хардверску реализацију Гаусовског НФ филтра.