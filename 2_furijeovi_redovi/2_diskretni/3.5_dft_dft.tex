\PID
Нека је дат дискретан сигнал $x[n]$, чија је основна периода $N$. Полазећи од тог сигнала, два пута је одређен развој у Фуријеов ред на 
основном периоду $y[n] = \FS{\FS{x[n]}}$. Одредити $y[n]$. 

\RESENJE
По дефиницији развоја у дискретан Фуријеов ред, може се писати 
\begin{equation}
    \FS{x[n]}[k] = X[k] = \dfrac{1}{N} \sum_{n = 0}^{N-1} x[n] \, \ee^{-\jj k \Upomega_{\rm F} n}, \qquad\qquad \Omega_{\rm F} = \dfrac{2\uppi}{N},
    \label{\ID.1}
\end{equation}
па се поново на основу дефиниције има 
\begin{eqnarray}
    \FS{X[k]}[m] = \dfrac{1}{N} \sum_{k = 0}^{N-1} X[k] \, \ee^{-\jj m \Upomega_{\rm F} k}. \label{\ID.2}
\end{eqnarray}
Заменом резултата \eqref{\ID.1} у \eqref{\ID.2} и сређивањем израза заменом редоследа сумирања добија се 
\begin{eqnarray}
    \FS{\FS{x[n]}} 
    &=&
    \dfrac{1}{N^2} 
    \sum_{k = 0}^{N-1} 
    \underbrace{ \sum_{n = 0}^{N-1} x[n] \, \ee^{-\jj k \Upomega_{\rm F} n} }_{X[k]}
    \, \ee^{-\jj m \Upomega_{\rm F} k}
    \\ 
    &=& 
    \dfrac{1}{N^2} 
    \sum_{n = 0}^{N-1}
    \sum_{k = 0}^{N-1} 
    x[n] \, \ee^{-\jj k \Upomega_{\rm F} n} 
    \, \ee^{-\jj m \Upomega_{\rm F} k} \\
    &=& 
    \dfrac{1}{N^2} 
    \sum_{n = 0}^{N-1}
    x[n]
    \sum_{k = 0}^{N-1} 
    \ee^{-\jj k \Upomega_{\rm F} n} 
    \ee^{-\jj m \Upomega_{\rm F} k}
    \\
    &=& 
    \dfrac{1}{N^2} 
    \sum_{n = 0}^{N-1}
    x[n]
    \sum_{k = 0}^{N-1} 
    \ee^{-\jj k \Upomega_{\rm F} (n + m)} \label{\ID.3}
\end{eqnarray}
Добијена сума у последњем кораку представља суму геометријске прогресије па се на сличан начин као и у задатку
\refz{dirak_povorka} налази
\begin{equation}
    \sum_{k = 0}^{N-1} 
    \ee^{-\jj k \Upomega_{\rm F} (n + m)}  
    = 
    \begin{cases}
        N &, n = -m \\
        0 &, n \neq -m
    \end{cases},
\end{equation}
одакле се заменом у \eqref{\ID.3} коначно добија\footnote{Добијени резултат 
представља множење и инверзију временске осе. У многим софтверским алатима, развој у Фуријеов ред се обавља 
применом Брзе Фуријеове трансформације (енг. \textit{FFT -- Fast Fourier Transofrm}) која је један од најважнијих алгоритама савременог
доба са веома широким спектром примена у науци и инжењерству. 
У различитим имплементацијама, коефицијент испред суме, који је по конвенцији 
на овом курсу усвојен да је $\dfrac{1}{N_{\rm F}}$,  не мора бити такав. 
На пример, у програмском пакету \texttt{GNU Octave} функција \texttt{fft} рачуна 
овај ред без тог коефицијента. У том случају, константа која множи резултат ће бити другачија, 
али фундаментална је ствар, независна од ове конвенције, појава инверзије временске осе.} $\FS{\FS{x[n]}} = \dfrac{x[-n]}{N}$. 

