\textbf{\color{red}*}\PID 
Нека су дати дискретни сигнали \vspace{2mm}

\hfill
$x[n] = 1 - \cos\left( \dfrac{2\uppi n}{3}  \right)$,
\hfill
$y[n] = \sin^2\left( \dfrac{2\uppi n}{8} + \dfrac{\uppi}{3} \right)$,
\hfill 
$z[n] = x[n] \cdot y[n]$, 
\hfill
и
\hfill
$w[n] = x[n] \circledast y[n]$
\hfill 
\,

\begin{enumerate}[label=(\alph*)]
\item 
Одредити основни период сигнала $x[n]$ и његов
развој у Фуријеов ред на његовом основном периоду $X[k]$.
\item 
Одредити основни период сигнала $y[n]$ и његов
развој у Фуријеов ред на његовом основном периоду $Y[k]$.
\item 
Применом резултата претходних тачака, и одговарајућих особина 
Фуријеових редова дискретних сигнала, 
одредити основни период сигнала $z[n]$ и његов
развој у Фуријеов ред на његовом основном периоду $Z[k]$.
\end{enumerate}

\textsc{\myul{Решење:}}
Простопериодична компонента сигнале је кружне учестаности $\Omega = \dfrac{2\uppi}{3}$, па је одговарајући основни период 
тога сигнала $N_x = \dfrac{2\uppi}{\Omega} = 3$. Дати сигнал се може расписати као 
$
    x[n] = 1 - \dfrac{2\uppi n}{3} = 1 - \dfrac{z + z^{-1}}{2} = - \dfrac{1}{2}z^{-1} + 1 - \dfrac{1}{2}z,
$
где је $z = \ee^{\jj\Upomega_x n}$. Будући да је развој у Фуријеов ред заправо степени ред по $z$, 
као 
$x[n] = \sum_{\langle N_x \rangle} X[k]\,z^k$, идентификацијом налазимо основни период спектра, чији је период $N_x = 3$, као 
\begin{equation}
    X[k] = -\dfrac{1}{2}\updelta[k+1] + \updelta[k] -\dfrac{1}{2}\updelta[k-1].
\end{equation}

(б) На сличан начин као у претходној тачки, приметимо да је $\sin^2(\Omega_0 n) = \dfrac{1 - \cos(2\Upomega_0 n)}{2}$.
Пошто фаза не улази  у разматрање, а важи $\Omega_0 = \dfrac{2\uppi}{8}$, период овог сигнала износи 
$N_{y} = \dfrac{2\uppi}{2\Upomega_y} = 4$. Дати сигнал се онда може расписати као  
\begin{equation}
    x[n] = \dfrac{1 - \cos(\Omega_0 n)}{2} = \dfrac{1}{2}\left( 1 - \dfrac{z + z^{-1}}{2} \right)
         = -\dfrac{1}{4}z^{-1} + \dfrac{1}{2} - \dfrac{1}{4}z,
\end{equation}
где је $z = \ee^{\jj\Upomega_y n}$. На основу тога је основни период спектра овог сигнала 
\begin{equation}
    Y[k] = -\dfrac{1}{4}\updelta[k - 1] + \dfrac{1}{2}\updelta[k] - \dfrac{1}{4}\updelta[k - 1] + \underbrace{0\cdot \updelta[k-2]}_{\mathclap{\text{\small Наглашавање периода}}}.
\end{equation}

(в) Период сигнала $z[n]$ је $N_{z} = \operatorname{NZS}(N_x, N_y) = 12$. Применом својства производа дискретног Фуријеовог реда 
може се писати $ \FS{x[n]\cdot y[n]} = X[k] \circledast Y[k]$, којом приликом се Фуријеови редови морају рачунати на једнаком броју 
одбирака. У овом случају, спектре треба рачунати на $N_z = 12$ одбирака. Односно, спектре рачунате у тачкама (а) и (б) треба проширити на ту дужину. 
Подсетимо да се спектар сигнала рачуна на $M$ основних периода уметањем $M-1$ нула у спектар израчунат на основом периоду. Тако се добијају 
спектри сигнала израчунати на периоду 12, дати у таблици \\
\begin{tabular}{c|ccccccccccccc}
            & 0 & 1 & 2 & 3 & 4 & 5 & 6 & 7 & 8 & 9 & 10 & 11 \\
    $X_{12}[k]$ & $X[0]$ & 0 & 0 & 0 & $X[1]$ & 0 & 0 & 0 & $X[2]$ & 0 & 0 & 0 \\
    $Y_{12}[k]$ & $Y[0]$ & 0 & 0 & $Y[1]$ & 0 & 0 & $Y[2]$ & 0 & 0 & $Y[3]$ & 0 & 0.  \\
\end{tabular} 

\noindent
Спектар сигнала $z[n]$ се онда налази као 
\begin{equation}
    Z[k] = \sum_{i + j \equiv k ({\rm mod}\, 12)} X_{12}[i] Y_{12}[j].
\end{equation}
Директним израчунавањем одређује се коначни резултат.