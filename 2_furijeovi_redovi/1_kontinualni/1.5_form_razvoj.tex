\PID Поновити задатак \refz{snaga_complex} применом формалног развоја у Фуријеов ред. 

\RESENJE

Формални развој у Фуријеов ред подразумева развој датог сигнала у степени ред облика 
$\sum_{k = -\infty}^{\infty} X[k] z^{k}$, где је $z = \ee^{\jj\upomega t}$, изражавањем датог израза по 
$z$ па формалним сређивањем израза и развојем у редове помоћу познатих развоја. У овом случају, може се искористити $\cos(\upomega t) = \dfrac{\ee^{\jj\upomega t} + \ee^{-\jj\upomega t}}{2} 
= \dfrac{z + z^{-1}}{2}$ па се након замене и сређивања у дати израз за струју добија 
\begin{equation}
    i(t) = I_0 \dfrac{0,75}{1,25 - \dfrac{z + z^{-1}}{2} } 
         = I_0 \dfrac{1,5}{2,5 - z - z^{-1}}  
         \cdot 
         \dfrac{z}{z}
         = I_0 \dfrac{1,5 z}{ -z^2 + 2,5 z - 1 } 
         =  -I_0 \dfrac{1,5 z}{ \left( z - \dfrac{1}{2} \right) (z - 2) } \label{eq:\ID.i}
\end{equation}
Добијени израз може се раставити на парцијалне разломке по параметру $z$, одакле се применом техника описаних 
у додатку \ref{a:pfd} добија 
$
    i(t) = -I_0 \left( \dfrac{1}{\dfrac{1}{2}z - 1} - \dfrac{1}{2z - 1} \right). 
$ Леви члан у добијеном изразу може се развити директно у геометријски ред\footnote{Израз за суму геометријског реда 
$\sum_{k = 0}^{\infty} q^k = \dfrac{1}{1 - q}$, $|q| < 1$ } по параметру $z$, будући да је 
$\left\lvert\dfrac{1}{2} z\right\rvert = \dfrac{1}{2} \cancelto{1}{|z|}  < 1$, док је 
за други члан није могуће извести исто јер није задовољен услов $|2z| < 1$. Да би се и он развио потребно је приметити
$
\dfrac{1}{2z - 1} = \dfrac{1}{2}z^{-1} \dfrac{1}{1 - \dfrac{1}{2}z^{-1} }
$, па је њега могуће развити у геометријски ред по $\dfrac{1}{2}z^{-1}$. Имајући наведено у виду, даље се може 
заменом у \eqref{eq:\ID.i} писати
\begin{eqnarray}
    i(t) &=& -I_0 \left(
        \dfrac{1}{\dfrac{1}{2}z - 1} - \dfrac{1}{2}z^{-1} \dfrac{1}{1 - \dfrac{1}{2}z^{-1} }    
    \right) \\[2mm]
    &=& I_0 \left(
      \sum_{k = 0}^{\infty} \left(\dfrac{1}{2}z \right)^k + \dfrac{1}{2} z^{-1} \sum_{k = 0}^{\infty} \left(\dfrac{1}{2} z^{-1} \right)^k     
    \right) \\[2mm]
    &=&
    I_0 \left(
        \sum_{k = 0}^{\infty} \left(\dfrac{1}{2} \right)^k z^k + \dfrac{1}{2} z^{-1} \sum_{k = -\infty}^{0} \left(\dfrac{1}{2}\right)^{-k} z^{k}     
    \right)
    \\[2mm] &=&
    I_0 \left(
        \sum_{k = 0}^{\infty} \left(\dfrac{1}{2} \right)^k z^k + \sum_{k = -\infty}^{0} \left(\dfrac{1}{2}\right)^{-(k-1)} z^{k-1}     
    \right) \\[2mm]
        &=&
    I_0 \left(
        \sum_{k = 0}^{\infty} \left(\dfrac{1}{2} \right)^k z^k + \sum_{k = -\infty}^{-1} \left(\dfrac{1}{2}\right)^{-k} z^k     
    \right)
\end{eqnarray}
Коначним обједињавањем добијених сума добија се 
$i(t) = \sum_{ k = -\infty }^{\infty} \underbrace{ I_0 \left(\dfrac{1}{2}\right)^{|k|}}_{I[k]} z^k$. Снага пријемника 
онда је одређена као 
$P_{\rm p} = R_{\rm p} \overline{i^2}$, а снага сигнала $i(t)$ 
може се израчунати онда помоћу Парсевалове теореме 
\begin{equation}
    \overline{i^2} = \sum_{k = -\infty}^{\infty} |I[k]|^2 = 
    |I[0]|^2 + 2 \sum_{k = 1}^{\infty} |I[k]|^2 =   
    I_0^2 \left( 1 + 2 \sum_{k = 1}^{\infty} \left(\dfrac{1}{2} \right)^{2k} \right) = \dfrac{5}{3} \unit{mA^2},
\end{equation}
па се коначно израчунава $R_{\rm p} = 5\unit{mW}$. 

Дати сигнал припада нарочитој групи сигнала чији
Фуријеови редови образују различите геометријске низове, а који су доступни и кроз таблицу 
\reft{T:frs}