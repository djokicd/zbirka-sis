\noindent\begin{slikaDesno}{fig/I0tri.pdf}
\PID  Нека је дата струја облика униполарне поворке симетричних троугаоних импулса, периода $T$ и амплитуде $I_0$, 
на свом основном периоду изразом
$
i(t) = \left| 2\dfrac{I_0}{T} t \right|
$ као што је илустровано на слици. Одредити развој овога напона у његов тригонометријски Фуријеов ред.
\end{slikaDesno}

\RESENJE
Задатак се може решити полазећи од дефиниције што се препушта заинтересованом читаоцу. Ефикасније 
решење може се добити применом својства о изводу, односно интегралу, сигнала.
Приметимо да се диференцирањем дате струје добија 
$\dfrac{\de i}{\de t} = \dfrac{2I_0}{T} \sgn(t)$, што је облик који одговара сигналу добијеном у 
задатку \ref{z:pravougani_po_def}. Уколико тај сигнал означимо са $v = v(t)$, као у том задатку, 
онда можемо изразити 
$
\dfrac{\de i}{\de t} = \dfrac{2 I_0}{T} \dfrac{v}{V_0},
$
односно, водећи рачуна о средњој вредности струје, можемо писати
\begin{equation}
    i(t) = \dfrac{I_0}{2} + \dfrac{2 I_0}{T V_0} \int_{-\infty}^t v(\uptau) \de \uptau.
\end{equation}
Применом особине интеграљења Фуријеовог реда континуалног сигнала\footnote{
$\FS{ \int_{\uptau = -\infty}^t x(\uptau) \, \de \uptau } = \dfrac{1}{\jj k \upomega_0} X[k]$}
даље се могу одредити комплексни коефицијенти Фуријеовог реда струје $i(t)$ као 
\begin{eqnarray}
    I[k] = \dfrac{I_0}{2}\updelta[k] + \dfrac{2 I_0}{T V_0} \dfrac{1}{\jj k\upomega_0} V[k].
    \label{eq:\ID.2}
\end{eqnarray}
Комплексни коефицијенти напона $V[k]$ могу се изразити на основу релације \ref{eq:ctfs_ab_to_x} из додатка
$V[k] = \dfrac{B_v[k] - \jj A[k]}{2} = \dfrac{1}{2} B_v[k] = 
\dfrac{2V_0}{k\uppi} \cdot \begin{cases}
    1, k\text{ непарно} \\
    0, k\text{ парно} \\
\end{cases}.$
Заменом тако добијеног резултата
у \ref{eq:\ID.2} даље се добија:
\begin{align}
    I[k] &= \dfrac{I_0}{2} \updelta[k] + \dfrac{2 I_0}{T \cancel{V_0}} \dfrac{1}{\jj k \upomega_0} 
    \dfrac{2\cancel{V_0}}{k\uppi} n[k] = 
    \dfrac{I_0}{2} \updelta[k]  - \jj
    \dfrac{2 I_0}{k^2 \uppi^2} 
    \cdot \begin{cases}
        1, k\text{ непарно} \\
        0, k\text{ парно} \\
    \end{cases},
\end{align}
при чему је искоришћено $\upomega_0 T = 2\uppi$. Трансформисање у тригонометријске коефициjенте може се 
обавити помоћу релације \eqref{eq:ctfs_x_to_ab} чиме се има 
\begin{align}
    & A_i[0] = \dfrac{I_0}{2}, \\
    & A_i[k>0] = 0,  \\
    & B_i[k>0] = \dfrac{4 I_0}{k^2 \uppi^2} \cdot 
    \begin{cases}
        1, k\text{ непарно} \\
        0, k\text{ парно} \\
    \end{cases}.
\end{align}