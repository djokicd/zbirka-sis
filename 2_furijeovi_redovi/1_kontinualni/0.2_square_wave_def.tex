\noindent\begin{slikaDesno}{fig/V0unipolar_vi.pdf}
\PID \label{z:pravougani_po_def}  Нека је дат напон облика биполарне поворке симетричних правоугаоних импулса, периода $T$ и амплитуде $V_0$, 
на свом основном периоду изразом
$
v(t) = 
\begin{cases}
    -V_0 &, -\dfrac{T}{2} < t < 0 \\[2mm]
    V_0  &, 0 < t < \dfrac{T}{2}
\end{cases},
$ као што је илустровано на слици. Одредити развој овога напона у његов тригонометријски Фуријеов ред.
\end{slikaDesno}
%
\RESENJE
\textbf{I начин}
У општем случају, тригонометријски облик Фуријеовог реда дат је изразом \eqref{eq:trig_ctfs}. Нарочито, члан 
$A[0]$ представља средњу вредност сигнала, док чланови $A[k]$ и $B[k]$ представљају парне односно непарне 
компоненте тог сигнала респективно. Приметимо да је дати сигнал непаран, па су на основу тога средња вредност 
тог сигнала као и његове парне компоненте сви једнаки нули, $A[k] = 0, k \in \mathbb N_0$. Преостаје да се 
израчунају непарне компоненте што се може обавити по дефиницији према изразу \eqref{eq:ctfs_B_def} из додатка: 
\begin{eqnarray}
    B[k] = \dfrac{2}{T}\int_{-T/2}^{T/2} x(t) \sin(k\upomega_0 t) \de t, 
\end{eqnarray}
где су замењени $T_{\rm F} = T$, односно $\upomega_{\rm F} = \upomega_0 = \dfrac{2\uppi}{T}$. Будући да је 
подинтегрална величина парна функција даље се  може писати\footnote{Користи се таблични 
интеграл $\int \sin(x)\,\de x = -\cos(x) + C$.}
\begin{align}
    B[k] &= \dfrac{4}{T} \int_0^{T/2} V_0 \sin(k\upomega_0 t) \de t 
         = \dfrac{4}{T} \int_{t = 0}^{T/2} V_0 \sin(k\upomega_0 t) \dfrac{ \de(k \upomega_0 t) }{k \upomega_0} 
         \\
         &= \dfrac{4V_0}{k \underbrace{\upomega_0 T}_{2\uppi}} \left( 1 - \cos\left( k \underbrace{\upomega_0 \dfrac{T}{2}}_{\pi} \right) \right)
         = \dfrac{2V_0}{k\uppi} ( 1 - (-1)^k )
\end{align}
Резултат се може дискутовати раздвајањем на парне и непарне хармонике будући да је 
$1 - (-1)^k = 
\begin{cases}
    2 &, k \text{ непарно} \\
    0 &, k \text{ парно}
\end{cases}$. \vspace*{1mm}
Коначно закључујемо да дати сигнал има само непарне хармонике, и то 
\begin{equation}
B[k] = 
\begin{cases}
    \dfrac{4V_0}{k\uppi} &, k \text{ непарно} \\[2mm]
    0 &, k \text{ парно} 
\end{cases}.\label{eq:\ID.resenje1}
\end{equation}

\textbf{II начин} Задатак се може решити и трансформацијама табличног резултата 
\reft{T:ctfs:rect}.
Дефинишимо први прототип сигнала $p_1(t)$ (сигнал који ће се у више корака трансформисати 
до циљаног облика) као табличну поворка импулса, 
са параметрима $\dfrac{w}{T_0} = 0$ из таблице, чиме се добија одговарајући спектар прототипа у првом 
кораку
\begin{equation}
    P_1[k] = \dfrac{1}{2} \sinc\left(\dfrac{k}{2}\right) = 
    \begin{cases}
    \dfrac{1}{k\uppi} \sin\left(\dfrac{k\uppi}{2}\right), & k\neq 0 \\[2mm]
    \dfrac{1}{2}, & k =  0
    \end{cases}.
\end{equation}
Добијени сигнал има средњу вредност која се добија заменом $k=0$ која износи $\dfrac{1}{2}$. 
Други прототип сигнала добија се транслирањем сигнала по вредности $p_2(t) = p_1(t) - \dfrac{1}{2}$.
Иста операција се у спектру одражава на измену DC вредности за 
$k=0$, односно је $P_2[k] = P_1[k] - \FS{\dfrac{1}{2}} = P_1[k] - \dfrac{1}{2}\updelta[k]$, чиме се добија 
\begin{equation}
    P_2[k] = \dfrac{1}{k\uppi} \sin\left(\dfrac{k\uppi}{2}\right).
\end{equation}
Коначно, други прототип треба скалирати одговарајућом амплитудом
и закаснити за четвртину периода у времену да би се добио спектар тражене поворке, 
$v(t) = V_0p_2\left(t - \dfrac{T}{4}\right)$, па се применом својстава \refs{s:ctfs:lin} и \refs{s:ctfs:delay} добија 
\begin{equation}
V[k] = 2V_0 \dfrac{1}{k\uppi} \sin\left(\dfrac{k\uppi}{2}\right) \underbrace{\ee^{-\jj k \upomega_0 \frac{T}{4}}}_{ (-\jj)^k }
= (-\jj)^k 2V_0 \dfrac{1}{k\uppi} \sin\left(\dfrac{k\uppi}{2}\right).
\end{equation}
Резултат се може поједноставити примедбом да су оба низа $(-\jj)^k$ као и $\sin\left(\dfrac{k\uppi}{2}\right)$, 
4-периодична, па се њихов производ најједноставније може одредити члан по члан
\begin{eqnarray}
    \begin{matrix}
        k =  & 0 & 1 & 2 & 3 \\[2mm] \hline 
        \sin\left(\dfrac{k\uppi}{2}\right) & 0 & 1 & 0 & -1 \\[3mm]
        (-\jj)^k & 1 & -\jj & -1 & \jj \\ \hline
        \cdot & 0 & -\jj & 0 & -\jj \\
    \end{matrix},
\end{eqnarray}
односно закључујемо да је резултат $-\jj$ само за непарне $k$. Па је 
\begin{equation}
    V[k] = -\jj \dfrac{2V_0}{k\uppi} \begin{cases}
        1, & k\text{ непарно} \\
        0, & k\text{ парно}
    \end{cases}.
\end{equation}
Применом релација за трансформацију комплекснијх коефицијената у тригонометријске 
\ref{eq:ctfs_x_to_ab}, добијају се коефицијенти у тригонометријском облику 
\begin{eqnarray}
    A[k > 0] &=& 2\,\Re{V[k]} = 0 \\
    B[k > 0] &=& -2\,\Im{V[k]} = \dfrac{4V_0}{k\uppi}\begin{cases}
        1, & k\text{ непарно} \\
        0, & k\text{ парно}
    \end{cases},
\end{eqnarray}
добија се исти коначни резултат као и \ref{eq:\ID.resenje1}.