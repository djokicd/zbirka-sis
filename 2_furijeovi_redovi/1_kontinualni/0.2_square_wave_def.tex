\PID \label{z:pravougani_po_def}
\begin{slikaDesno}{fig/V0unipolar_vi.pdf}
Нека је дат напон облика биполарне поворке симетричних правоугаоних импулса, периода $T$ и амплитуде $V_0$, 
на свом основном периоду изразом
$
v(t) = 
\begin{cases}
    -V_0 &, -\dfrac{T}{2} < t < 0 \\[2mm]
    V_0  &, 0 < t < \dfrac{T}{2}
\end{cases},
$ као што је илустровано на слици. Одредити развој овога напона у његов тригонометријски Фуријеов ред.
\end{slikaDesno}
%
\RESENJE
У општем случају, тригонометријски облик Фуриеовог реда дат је изразом \eqref{eq:trig_ctfs}. Нарочито, члан 
$A[0]$ представља средњу вредност сигнала, док чланови $A[k]$ и $B[k]$ представљају парне односно непарне 
компоненте тог сигнала респективно. Приметимо да је дати сигнал непаран, па су на основу тога средња вредност 
тог сигнала као и његове парне компоненте сви једнаки нули, $A[k] = 0, k \in \mathbb N_0$. Преостаје да се 
израчунају непарне компоненте што се може обавити по дефиницији према изразу \eqref{eq:ctfs_B_def} из додатка: 
\begin{eqnarray}
    B[k] = \dfrac{2}{T}\int_{-T/2}^{T/2} x(t) \sin(k\upomega_0 t) \de t, 
\end{eqnarray}
где су замењени $T_{\rm F} = T$, односно $\upomega_{\rm F} = \upomega_0 = \dfrac{2\uppi}{T}$. Будући да је 
подинтегрална величина парна функција даље се  може писати\footnote{Користи се таблични 
интеграл $\int \sin(x)\,\de x = -\cos(x) + C$.}
\begin{align}
    B[k] &= \dfrac{4}{T} \int_0^{T/2} V_0 \sin(k\upomega_0 t) \de t 
         = \dfrac{4}{T} \int_{t = 0}^{T/2} V_0 \sin(k\upomega_0 t) \dfrac{ \de(k \upomega_0 t) }{k \upomega_0} 
         \\
         &= \dfrac{4V_0}{k \underbrace{\upomega_0 T}_{2\uppi}} \left( 1 - \cos\left( k \underbrace{\upomega_0 \dfrac{T}{2}}_{\pi} \right) \right)
         = \dfrac{2V_0}{k\uppi} ( 1 - (-1)^k )
\end{align}
Резултат се може дискутовати раздвајањем на парне и непарне хармонике будући да је 
$1 - (-1)^k = 
\begin{cases}
    2 &, k \text{ непарно} \\
    0 &, k \text{ парно}
\end{cases}$. \vspace*{1mm}
Коначно закључујемо да дати сигнал има само непарне хармонике, и то 
\begin{equation}
B[k] = 
\begin{cases}
    \dfrac{4V_0}{k\uppi} &, k \text{ непарно} \\[2mm]
    0 &, k \text{ парно}
\end{cases}.
\end{equation}