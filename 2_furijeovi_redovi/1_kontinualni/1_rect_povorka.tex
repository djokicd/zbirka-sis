\PID \label{z:rect_povorka}
За сигнал описан у задатку \refz{snaga_pwm} познато је 
$D = 25\%$ и $f = 1\unit{kHz}$. Одредити средњу снагу сигнала $v(t)$.
Одредити развоj овог сигнала у комплексан Фуриjеов ред, $V[k]$, на основном периоду $T$.
\\[2mm]

\RESENJE  Развој у Фуријеов ред се може потражити по дефиницији применом 
аналитичке релације 
$\DS V[k] = \int_{\langle T \rangle} v(t) \ee^{-\jj k \upomega_{\rm F} t} \, \de t$, 
поступком\footnote{Користи се резултат $\int e^{kx} \, \de x = \frac{1}{k} e^x + C$}
\begin{align}
    V[k] = \dfrac{1}{T} \int_{0}^T v(t) \ee^{-\jj k \upomega_{\rm 0} t} \, \de t 
         = \dfrac{1}{T} \int_{0}^{DT}  V_{\rm m} \ee^{-\jj k \frac{2\uppi}{T} t} \, \de t
         = - \dfrac{1}{T} \dfrac{V_{\rm m}}{\jj k \frac{2\uppi}{T}} 
         \ee^{-\jj k \frac{2\uppi}{T} t}\bigg|_{t = 0}^{t = DT}
         = \dfrac{V_{\rm m}}{T} \dfrac{
            1
            -
            \ee^{-\jj k 2\uppi D}
         }{\jj k \frac{2\uppi}{T}} 
\end{align}
Добијени облик може се поједноставити примедбом 
$\sin(x) = \dfrac{\ee^{\jj x} - \ee^{-\jj x}}{\jj 2} = \ee^{\jj x} \dfrac{1 - \ee^{-\jj 2x}}{\jj 2}$,
односно, 
$\dfrac{1 - \ee^{-\jj 2x}}{\jj 2} = \ee^{-\jj x} \sin(x) $,
одакле се може писати
\begin{align}
    V[k] = V_{\rm m}D 
     \underbrace{
     \dfrac{
        1
        -
        \ee^{-\jj 2 (k \uppi D) }
     }{\jj 2 }}_{ = \ee^{-\jj k \uppi D} \sin(k\uppi D) }
     \cdot 
     \dfrac{1}{k \uppi D} 
     = V_{\rm m} D \dfrac{\sin(k\uppi D)}{k\uppi D} \ee^{-\jj k \uppi D} 
     = V_{\rm m} D \sinc(k D) \ee^{-\jj k \uppi D}. \label{eq:\ID.form}
\end{align}

\textbf{II начин.} Исти резултат може  се одредити и помоћу трансформација поворке Диракових импулса. 
Приметимо да је $v(t) = V_{\rm m} \int_{-\infty}^t (\III_T(t) - \III_T(t - DT))\, \de t$. Применом особина 
померања у времену\footnote{
   $\FS{x(t-\uptau)} = \FS{x(t)} \ee^{-\jj k \upomega_{\rm 0} \uptau}$, $\uptau = \const$
}
и интеграције\footnote{
   $\FS{ \int_{\uptau = -\infty}^t x(\uptau) \, \de \uptau } = \dfrac{1}{\jj k \upomega_0} X[k]$ 
} Фуријеовог реда континуалног сигнала, уз резултат задатка \refz{dirak_povorka} добија се 
\begin{equation}
   \FS{v(t)} = V_{\rm m} \dfrac{1}{\jj k \upomega }  \left( \dfrac{1}{T} - \dfrac{\ee^{-\jj k \upomega D T}}{T} \right)
             = \dfrac{V_{\rm m}}{\uppi k} \dfrac{1 - \ee^{-\jj k \upomega D T}}{\jj 2}, \qquad \qquad \upomega T = 2\uppi,
\end{equation}
што је исти облик који је добијен и као \eqref{eq:\ID.form}, па се истим поступком долази до истог коначног резултата.

Добијени резултат може се одредити и помоћу таблице, којом приликом ће се члан 
$\ee^{-\jj k \uppi D}$ појавити услед кашњења у времену уноса у таблици доступног као \reft{T:ctfs:rect}. 
