\begin{slikaDesno}[1.2]{fig/formal_ckt.pdf}
    \PID 
    У колу са слике позната jе струjа струjног генера-
    тора у дата облику 
    $i_{\rm G}(t) = I_{\rm m} (1 + \cos(\upomega_0 t) \sin^2(\upomega_0 t))$,
    где су $I_{\rm m} = 1\unit{mA}$ и $\upomega_0 = 200\uppi \unit{\dfrac{rad}{s}}$.
    (а) Одредити развоj струjе $i_{\rm G}$ у Фуриjеов ред на основном периоду.
    (б) Израчунати средње снаге отпорника $R_1$ и $R_2$. У 
    колу jе успостављен сложенопериодичан режим.
\end{slikaDesno}

\REZULTAT

(а) Тражени развој је 
$I[k] = I_{\rm m} \left(
    \updelta[k] - \dfrac{1}{8}\left(
        \updelta[k-3] - \updelta[k-1] - \updelta[k+1] + \updelta[k+3]
    \right)
\right)$. (б) Снаге отпорника су 
$P_{R_1} \approx 100\unit{mW}$, и 
$P_{R_2} \approx \dfrac{1}{16}\unit{mW}$.