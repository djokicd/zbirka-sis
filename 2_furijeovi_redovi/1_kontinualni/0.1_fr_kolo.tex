\begin{slikaDesno}[0.833]{fig/fr_kolo.pdf}
    \PID
    Дат jе континуалан сигнал 
    $$ 
    x(t) = 1 - 2 \sin(\upomega_0 t) + 3 \cos(\upomega_0 t) 
    - 4 \cos(4 \upomega_0 t),$$ где jе $\upomega_0$ константа. 
    Израчунати коефицијенте развоjа тог сигнала у (а) тригонометриjски и (б) комплексни Фуриjеов ред, 
    на основном периоду. (в) У колу са слике познато jе $R_1 = 10R_2 = 5\unit{k\Omega}$, $L \to \infty$, 
    а струjа струjног генератора jе $i_{\rm G} = 1 \unit{mA} \cdot x(t)$. Израчунати ефективне вредности напона на отпорницима 
    $V_{R_1}$ и $V_{R_2}$.
\end{slikaDesno}

\RESENJE

Општа форма развоја у тригонометријски Фуријеов ред је 
\begin{equation}
    x(t) = A[0] + \sum_{k = 1}^{\infty} A[k] \cos(k\upomega_0 t) + \sum_{k = 1}^{\infty} B[k] \sin(k\upomega_0 t),
    \label{eq:trig_ctfs}
\end{equation}
па се чланови тригонометријског Фуријеовог реда могу директно препознати идентификацијом. Поређењем са формом датом у поставци задатка има се:
\begin{eqnarray}
    x(t) = 
    \underbrace{1}_{A[0]} 
    \underbrace{-2}_{B[{\color{blue}1}]} \sin({\color{blue} 1} \upomega_0 t) 
    \underbrace{+3}_{A[{\color{blue}1}]} \cos({\color{blue} 1} \upomega_0 t) 
    \underbrace{-4}_{A[{\color{blue}4}]} \cos({\color{blue} 4} \upomega_0 t),
\end{eqnarray}
односно: 
\begin{eqnarray}
    A[0] = 1, A[1] = 3, A[4] = -4 &\Rightarrow& A[k] = \updelta[k] + 3\updelta[k-1] - 4\updelta[k-4]\\
              B[1] = -2           &\Rightarrow& B[k] = -2\updelta[k-1]
\end{eqnarray}

(б) Коефицијенти комплексног Фуријеовог реда могу се изразити преко тригонометријских помоћу веза
$X[0] = A[0], X[k > 0] = \dfrac{A[k] - \jj B[k]}{2}, X[k < 0] = X^\ast[-k]$, па је 
\begin{eqnarray}
    && X[0] = 1 \\ 
    && X[1] = \dfrac{A[1] - \jj B[1]}{2} = \dfrac{3 + \jj 2}{2} \Rightarrow  X[-1] = \dfrac{3 - \jj 2}{2}  \\
    && X[4] = -2 \Rightarrow X[-4] = -2 \\
\end{eqnarray}
Па се коначно може записати 
\begin{equation}
    X[k] = - 2\updelta[k + 4] + \dfrac{3 - \jj 2}{2}\updelta[k+1] 
    + \updelta[k] 
    + \dfrac{3 + \jj 2}{2}\updelta[k-1] 
    - 2\updelta[k - 4].
\end{equation}

(б) Ефективна вредност напона отпорника може се изразити као 
\begin{eqnarray}
    V_R = \sqrt{ \dfrac{1}{T} \int_{\langle T \rangle} v_R^2  \, \de t  }
        = \sqrt{ \dfrac{1}{T} \int_{\langle T \rangle} R^2 i_R^2  \, \de t  }
        = R \sqrt{ \, \overline{i_R^2} },
\end{eqnarray}
Снага сигнала струје отпорника $\overline{i^2}$ може се рачунати помоћу Парсевалове теореме. Струја
отпорника $R_1$ једнака је струји генератора па је 
\begin{eqnarray}
    I_{R_1}[k] = 1\unit{mA} \cdot X[k]
\end{eqnarray}
па се снага тог сигнала налази као 
\begin{equation}
    \overline{i_{R_1}^2} = 1 \unit{mA^2} \sum_{k = -\infty}^{\infty} |X[k]|^2 = \dfrac{31}{2} \unit{mA^2}.
\end{equation}
па је онда $V_{R_1} = 5 \sqrt{\dfrac{31}{2}} \unit{V} \approx 19,7 \unit{V}$. За отпорник $R_2$, пошто је $L \to \infty$, само 
стална компонента струје генератора се успоставља на калему, односно, спектрални садржај струје отпорника 
$R_2$ исти је као и спектрални садржај струје отпорника $R_1$ са изузетком сталне компоненте која је код 
њега једнака нули. Снага сигнала струје тог отпорника се онда може потражити као
\begin{equation}
    \overline{i_{R_2}^2} = 1 \unit{mA^2} \sum_{ \substack{k = -\infty\\k\neq 0}}^{\infty} |X[k]|^2 = \dfrac{29}{2} \unit{mA^2}.
\end{equation}
па је онда $V_{R_2} = \dfrac{1}{2} \sqrt{ \dfrac{29}{2} } \unit{V} \approx 1,9 \unit{V}$.